\documentclass[11pt]{amsart}
%\pagestyle{empty} 
\setlength{\topmargin}{-0.5in} % usually -0.25in
\addtolength{\textheight}{1.2in} % usually 1.25in
\addtolength{\oddsidemargin}{-0.95in}
\addtolength{\evensidemargin}{-0.95in}
\addtolength{\textwidth}{1.9in} %\setlength{\parindent}{0pt}

\newcommand{\normalspacing}{\renewcommand{\baselinestretch}{1.1}\tiny\normalsize}
\normalspacing

% macros
\usepackage{amssymb,xspace}
\usepackage[final]{graphicx}
\usepackage[pdftex,colorlinks=true]{hyperref}
\usepackage{fancyvrb}
\usepackage{tikz}

\newtheorem*{lem*}{Lemma}

\newcommand{\bb}{\mathbf{b}}
\newcommand{\bc}{\mathbf{c}}
\newcommand{\bs}{\mathbf{s}}
\newcommand{\bu}{\mathbf{u}}
\newcommand{\bv}{\mathbf{v}}
\newcommand{\bx}{\mathbf{x}}
\newcommand{\by}{\mathbf{y}}

\newcommand{\bbf}{\mathbf{f}}

\newcommand{\CC}{{\mathbb{C}}}
\newcommand{\RR}{{\mathbb{R}}}
\newcommand{\eps}{\epsilon}
\newcommand{\ZZ}{{\mathbb{Z}}}
\newcommand{\ZZn}{{\mathbb{Z}}_n}
\newcommand{\NN}{{\mathbb{N}}}
\newcommand{\ip}[2]{\mathrm{\left<#1,#2\right>}}

\renewcommand{\Re}{\operatorname{Re}}
\renewcommand{\Im}{\operatorname{Im}}

\newcommand{\Log}{\operatorname{Log}}

\newcommand{\grad}{\nabla}

\newcommand{\ds}{\displaystyle}

\newcommand{\Matlab}{\textsc{Matlab}\xspace}

\newcommand{\prob}[1]{\bigskip\noindent\textbf{#1.} }
\newcommand{\pts}[1]{(\emph{#1 pts})}

\newcommand{\probpts}[2]{\prob{#1} \pts{#2} \quad}
\newcommand{\ppartpts}[2]{\textbf{(#1)} \pts{#2} \quad}
\newcommand{\epartpts}[2]{\medskip\noindent \textbf{(#1)} \pts{#2} \quad}

\newcommand{\boxy}[1]{\fbox{\Huge \strut \hspace{#1mm} \strut}}
\newcommand{\boxycontent}[2]{\fbox{\quad {\large #2} {\Huge \strut \hspace{#1mm} \strut}}}


\begin{document}
\hfill \Large Name:\underline{\phantom{Ed Bueler really really long long long name}}
\medskip

\scriptsize \noindent Math 302 Differential Equations (Bueler) \hfill Wednesday 18 October 2023
\medskip

\LARGE\centerline{\textbf{Midterm Exam}}

\smallskip
\begin{quote}
\large
\textbf{In-class.  No book, notes, electronics, calculator, internet access, or communication with other people.  Please write your solution in the box if provided.  100 points possible.  \underline{65 minutes} maximum!}
\end{quote}

\normalsize

\thispagestyle{empty}

% 2.2 #6 essentially
\probpts{1}{6}  Solve the differential equation by separation of variables: \quad $\ds \frac{dy}{dx} - 2 x y^2 = 0$
\vfill

\hfill $\boxycontent{80}{$y(x)=$}$

% Quiz 1 #2
\probpts{2}{6}  Verify that $y(t) = 1 / (c - t)$ is a one-parameter family of solutions of the differential equation \quad $\ds \frac{dy}{dt} = y^2$.
\vfill


\clearpage
\newpage
% 2.4 #3 essentially
\probpts{3}{15}  Determine whether the differential equation is exact.  If it is exact, solve it and write the solution in the box: \quad {\huge \strut}$\ds \left(8 y^3 - 4x\right) \frac{dy}{dx} = 5x + 4y$
\vfill

\hfill $\boxy{90}$


\clearpage
\newpage
% 2.3 #29
\probpts{4}{15}  Solve the initial value problem, assuming $L,R,E,i_0$ are constants:

\medskip
\hspace{19mm} $\ds L\frac{di}{dt} + R i = E$, \quad $i(0)=i_0$
\vfill

\hfill $\boxycontent{90}{$i(t)=$}$


\clearpage
\newpage
% update of Quiz 3 #4
\prob{5}  \ppartpts{a}{8}  Find the general solution: \quad $\ds y'' - y' - 12 y = 0$
\vfill

\hfill $\boxycontent{80}{$y(x)=$}$

\bigskip
\epartpts{b}{8}  Show that your general solution in part \textbf{(a)} is built from a fundamental set of solutions.
\vfill


\clearpage
\newpage
% Quiz 2 #1,2
\prob{6}  \ppartpts{a}{5}  The direction field of the following differential equation is shown: \quad $\ds \frac{dx}{dt} = 1 + t x$

For each of the following points, sketch an approximate solution curve.

\medskip
\begin{itemize}
\item $x(0)=-1$

\medskip
\item $x(-2)=2$
\end{itemize}

\vspace{-10mm}
\hfill \includegraphics[width=0.7\textwidth]{figs/xtdirfield.png}

\bigskip
\epartpts{b}{8}  Find the general solution of the differential equation in part \textbf{(a)}.  You may write the solution in integral form if you do not know how to do an integral.
\vfill

\hfill $\boxycontent{90}{$x(t)=$}$


\clearpage
\newpage
\prob{7}  \ppartpts{a}{5}  Newton's law of cooling/warming says that the rate of change of an object's temperature is proportional to the difference between its temperature and a constant ambient temperature.  Write this differential equation, a model for the temperature $T(t)$.  (\emph{Hint.  This model has two constants.})

\vspace{5mm}

\hfill $\boxy{90}$ \phantom{lkadjsf asdf adskfj}

\bigskip
\epartpts{b}{5}  Find the general solution of the differential equation in part \textbf{(a)}.
\vfill

\hfill $\boxycontent{90}{$T(t)=$}$

\bigskip
\epartpts{c}{5}  A thermometer---\emph{it is the object here}---is initially at room temperature $20^\circ$ C.  It is put in boiling water at $100^\circ$ C.  After 2 seconds the thermometer reads $60^\circ$ C.  What is the constant of proportionality in the above model?
\vfill
\hfill $\boxy{60}$


\clearpage
\newpage
% like 4.3 #16
\prob{8}  \ppartpts{a}{7}  Find the general solution of this third-order, constant-coefficient, homogeneous, and linear differential equation: \quad {\LARGE \strut} {\large $\ds y''' + y' = 0$}
\vfill

\hfill $\boxycontent{80}{$y(x)=$}$

\bigskip
\epartpts{b}{7}  The general solution in part \textbf{(a)} is a linear combination of three specific functions $f_1(x),f_2(x),f_3(x)$.  State these functions and then compute and simplify the Wronskian.
\vfill

\hfill $\boxycontent{60}{$W(f_1,f_2,f_3)=$}$


\bigskip
\probpts{Extra Credit}{3}  Find the general solution on $(0,\infty)$: \quad $x y'' - y' = 0$
\vspace{2.0in}


\clearpage
\newpage
\thispagestyle{empty}
\begin{center}
\small
\textsc{blank space (full page)}
\end{center}
\vfill

\end{document}
