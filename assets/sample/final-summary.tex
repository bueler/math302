\documentclass[12pt]{article}
\usepackage[top=1in, bottom=0.75in, left=1in, right=1in, headheight=1in, headsep=6pt]{geometry}

% Fonts.
\usepackage{mathptmx}
%\usepackage[scaled=0.86]{helvet}
%\renewcommand{\emph}[1]{\textsf{\textbf{#1}}}

% Misc packages.
\usepackage{amsmath,amssymb,latexsym}
\usepackage{graphicx}
\usepackage{array}
\usepackage{xcolor}
\usepackage{multicol}
\usepackage{tabularx,colortbl}
\usepackage{enumitem}
\usepackage[colorlinks=true]{hyperref}

% Paragraph spacing
\parindent 0pt
\parskip 6pt plus 1pt
\def\tableindent{\hskip 0.5 in}
\def\ts{\hskip 1.5 em}

\usepackage{fancyhdr}
\pagestyle{fancy}
\lhead{MATH F302 UX1 (Spring 2019)}
\rhead{23 April 2019}
\cfoot{}


\begin{document}

\phantom{k}
\vspace{2mm}
\Large\centerline{\textbf{Summary of the Final Exam}} \normalsize

\bigskip
The Final Exam must be taken

\medskip
\centerline{\textbf{Tuesday April 30} \quad or \quad \textbf{Wednesday May 1} \quad or \quad \textbf{Thursday May 2}}

The format: 150 minutes, 135 points, no book, no notes, no calculator.  It is just like a Midterm Exam, but 50\% longer.  There is a cover sheet, seven pages of actual exam, and finally three pages of tables and blank space. 

The Final Exam is comprehensive over the whole course.  Here is a detailed summary of what is on the actual Exam.

\medskip
\begin{itemize}
\item[page 1:]  A constant-coefficient 2nd-order ODE; solve by \S4.3 methods.  Then a \S2.2 separation of variables 1st-order ODE problem.
\item[page 2:]  A three-part problem based on the ideas in \S2.1, with a supplied  direction field.  The third part asks for a sketch of two steps of the Euler method (\S2.6).
\item[page 3:]  First is an initial value problem for a linear, 1st-order ODE; solve by \S2.3 methods.  Second is a problem requiring that you know the definition of the Laplace transform (from \S7.1; the definition is \emph{not} in a Table) and requiring that you know how to do integration-by-parts.
\item[page 4:]  Use the Laplace transform to solve a 2nd-order ODE IVP.  (Review partial fractions.  Also practice using a table of Laplace transforms to find $y(t)$ once you have found $Y(s)$, that is, to do the inverse Laplace transform as in \S7.2.)
\item[page 5:]  In the first problem, \emph{verify} that a given formula satisfies a certain 2nd-order ODE IVP; see the problems in \S4.1.  In the second problem, convert a scalar 2nd-order ODE to a 1st-order system; see \S4.10 and the slides for \S5.3.  Also an Extra Credit problem about using \texttt{ode45}.
\item[page 6:]  A question like example 1 or 2 in \S8.2 where, as promised in the \S8.2 slides, I supply the eigenvalues and eigenvectors.  Also an Extra Credit problem about the matrix exponential.
\item[page 7:]  A straightforward \S6.2 computation of a power series solution of a 2nd-order ODE IVP.  The second part is easy if you understand that you know the radius of convergence in advance of computing the series itself.  (See the beginning of \S6.2.)
\item[pages 8--9:]  Here are four tables you have seen before.  The first three appeared on Midterm 2.
    \begin{enumerate}
    \item \emph{Brief Table of Integrals}
    \item Table of \emph{Maclaurin Series} from page 239 of the textbook
    \item Table 4.4.1 of \emph{Trial Particular Solutions} from \S4.4 of the textbook
    \item \emph{Table of Laplace Transforms}, as appeared on Quiz 9 and Quiz 10
    \end{enumerate}
\item[page 10:]  Blank space for your work (if needed).
\end{itemize}
\end{document}
