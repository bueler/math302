\documentclass[12pt]{article}
\usepackage[top=0.9in, bottom=0.9in, left=0.9in, right=1.1in]{geometry}

\usepackage{graphicx,color,enumitem}
\usepackage{amsmath,amsthm,amsbsy}
\usepackage{palatino}

\usepackage{tikz}
\usepackage[colorlinks=true]{hyperref}

%% Setup aproblem environment, 
%% aproblem items
%% subproblems environment
%% subproblem items
\makeatletter
\newcounter{probcount}
\newcounter{subprobcount}
\newlength\probsep
\newlength\pshrinking
\newif\iffirstprob

\newenvironment{aproblems}%
  {\ifhmode\unskip\par\fi\setcounter{probcount}{0}\probsep\parskip
  \sbox\@tempboxa{\textbf{9.}}\pshrinking\wd\@tempboxa\advance\pshrinking\labelsep
  \let\hproblem\aproblem
  \advance\linewidth -\pshrinking
  \advance\@totalleftmargin\pshrinking
  \advance\leftskip\pshrinking}%
  {\ifhmode\unskip \par\fi\advance\leftskip-\pshrinking}%

\newcommand{\aproblem}{%
  \setcounter{subprobcount}{0}%
  \stepcounter{probcount}%
  \def\@currentlabel{\arabic{probcount}}%
  \ifhmode
    \unskip \par
  \fi
%  \addpenalty{-4000}%
  \iffirstprob\else\addvspace\probsep\fi
  \firstprobfalse
  \hskip -\labelwidth\hskip -\labelsep 
  \hbox to\labelwidth{\hss\textbf{\arabic{probcount}.}}\hskip\labelsep
}%

\newcommand{\subprob}{\item\def\@currentlabel{\arabic{probcount}\alph{\thelistlabel}}}
\newcommand{\skipproblem}{\stepcounter{probcount}}


%% The following commands put defined left and right headers on the top, and a page number
%% on the bottom of all pages beyond page 1
\usepackage{fancyhdr}
\pagestyle{fancy}
\fancyfoot[C]{\ifnum \value{page} > 1\relax\thepage\fi}
\fancyhead[L]{\ifx\@doclabel\@empty\else\@doclabel\fi}
\fancyhead[R]{\ifx\@docdate\@empty\else\@docdate\fi}
\headheight 15pt
\def\doclabel#1{\gdef\@doclabel{#1}}
\def\docdate#1{\gdef\@docdate{#1}}
\makeatother

%% General formatting parameters
\parindent 0pt
\parskip 6pt plus 1pt

\newcommand{\epart}[1]{\noindent \textbf{(#1)} \,}

\doclabel{Math F302 UX1: Mini-Project 3}
\docdate{due \emph{at 11pm} on Thursday 7 March 2019}

\begin{document}
\renewcommand{\d}{\displaystyle}

\strut
\centerline{{\Large \textsc{Energy is just a technique}}}

\smallskip

\centerline{{\Large \textsc{for solving differential equations}}}

\medskip
\small
\begin{quote}
\textbf{Instructions.}  \emph{Write your solutions neatly on separate sheets of paper, or start a new document on a computer.  Include your name in the upper right on each sheet.  Clearly label the parts, i.e.~``\emph{(a)}'', ``\emph{(b)}'', \dots following the headings below.  At each step, show your work neatly.  (Full credit requires both evidence of clear understanding \emph{and} clear writing.)  Your final document, which should \emph{not} be your first draft, should be at most four pages.}

\emph{Submit your PDF by uploading using the link (to a Google Form) on the ``Week 8'' page at the main course page} \href{https://bueler.github.io/math302/index.html}{\texttt{bueler.github.io/math302}}.  \emph{If you wrote on paper, scan or photograph your completed document.  (If you use a phone then make the lighting good and line things up carefully!  Use a scanner app.  In any case generate an easily-readable PDF!)}
\end{quote}

\normalsize
\bigskip


\textbf{Energy.}  It is common to talk about natural gas or oil as ``energy.''   We say batteries use chemistry to store ``energy'' which they then deliver as electrical ``energy.''  People who know no physics or math happily discuss the ``energy industry'' and the ``politics of renewable energy''.

But the idea of energy in this sense is new, younger than the United States of America, and it is arose in physics before entering everyday language:

\small

\begin{quotation}
\noindent \emph{In 1807, Thomas Young was possibly the first to use the term ``energy'' \dots in its modern sense.  Gustave-Gaspard Coriolis described ``kinetic energy'' in 1829 in its modern sense, and in 1853, William Rankine coined the term ``potential energy.''  The law of conservation of energy was also first postulated in the early 19th century \dots}

\hfill \url{https://en.wikipedia.org/wiki/Energy}
\end{quotation}

\normalsize
As a conserved quantity that cannot be created or destroyed, but only transformed, ``energy'' arose as a way of partly solving the differential equations (DEs) of motion.  From the point of view of Newtonian and classical mechanics, \emph{energy} is a \emph{first integral}, a partial solution, of the famous second-order DE
\begin{equation}
    F=ma.  \label{Newton2nd}
\end{equation}
In fact, the energy appears when you use a kind of integrating-factor technique for certain second-order DEs, so in that sense it is comparable to the technique you know for solving first-order linear equations.

Our textbook does not discuss energy, which is a bit odd.  Most textbook treatments of DEs include it, but on the other hand energy is often not a used to fully solve the DE.  The energy technique helps understand the DE by identifying a \emph{conserved quantity}, namely the total energy which is a sum of \emph{kinetic} and \emph{potential} energy.  This Mini-Project introduces energy, makes sense of these technical terms, and fixes the book's omission.

\medskip
\textbf{The first integral.}  We consider second-order, possibly-nonlinear DEs of the form
\begin{equation}
y'' = f(y) \label{basicform}
\end{equation}
Note that $y''$ and $y$ appear in equation \eqref{basicform} but $y'$ does not.  The function $f(y)$ may be linear or nonlinear.

\renewcommand{\labelenumi}{\textbf{(\alph{enumi})}}
\begin{enumerate}
\item To start, find the general solution of \eqref{basicform} if
    $$f(y) = Ky$$
where $K$ is any real constant.  (\emph{Hint.}  There are multiple cases, but each is familiar.)
\item Find the general solution of \eqref{basicform} if $f(y)$ is the linear function
    $$f(y) = -ky + g$$
where $k$ and $g$ are positive real constants.  (\emph{Hint.}  Easy \S 4.3 and 4.4 problem.)
\item FIXME
\end{enumerate}

FIXME

\end{document}
