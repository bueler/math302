\documentclass[12pt]{article}
\usepackage[top=0.9in, bottom=0.9in, left=0.9in, right=1.1in]{geometry}

\usepackage{graphicx,color,enumitem}
\usepackage{amsmath,amsthm,amsbsy}
\usepackage{palatino}
\usepackage{mdframed}

\usepackage{tikz}
\usepackage[colorlinks=true,urlcolor={blue}]{hyperref}

\mdfdefinestyle{exerpartSty}{}
\theoremstyle{definition}
\newmdtheoremenv[style=exerpartSty,linewidth=0.5pt, topline=false, bottomline=false, rightline=false,%
leftmargin=0pt, innerleftmargin=0.4em, rightmargin=0pt, innerrightmargin=0pt, innertopmargin=-5pt ,%
innerbottommargin=3pt, splittopskip=\topskip, splitbottomskip=0.3\topskip, %
skipabove=1.3\topsep]{exerpart}{Part}

\renewcommand*{\theexerpart}{(\alph{exerpart})}
\newcommand{\eps}{\epsilon}

%% The following commands put defined left and right headers on the top, and a page number
%% on the bottom of all pages beyond page 1
\makeatletter
\usepackage{fancyhdr}
\pagestyle{fancy}
\fancyfoot[C]{\ifnum \value{page} > 1\relax\thepage\fi}
\fancyhead[L]{\ifx\@doclabel\@empty\else\@doclabel\fi}
\fancyhead[R]{\ifx\@docdate\@empty\else\@docdate\fi}
\headheight 15pt
\def\doclabel#1{\gdef\@doclabel{#1}}
\def\docdate#1{\gdef\@docdate{#1}}
\makeatother

%% General formatting parameters
\parindent 0pt
\parskip 6pt plus 1pt

\newcommand{\epart}[1]{\noindent \textbf{(#1)} \,}

\doclabel{Math F302 UX1: Mini-Project 5}
\docdate{due \emph{at 11pm} on Monday 22 April 2019}

\begin{document}
\renewcommand{\d}{\displaystyle}

\strut
\centerline{{\Large \textsc{One problem, many methods}}}

\medskip
\centerline{{\large a review assignment}}

\medskip
\small
\renewcommand{\baselinestretch}{1.0}
\begin{quote}
\textbf{Instructions.}  Please write your solutions neatly on paper, or start a new document on a computer.  Include your name in the upper right on each sheet.  Clearly label the parts of your solutions following the headings below.    Your final document, which should \emph{not} be your first draft, should be at most five pages.  Submit your PDF by uploading using the Google Form link on the ``Week 14'' page at

     \centerline{\href{https://bueler.github.io/math302/index.html}{\texttt{bueler.github.io/math302}}}  

If you wrote on paper, please scan or photograph your completed document and make the lighting good.  Also line things up carefully, and/or use a scanner app.  In any case \emph{please generate an easily-readable PDF.}

Because of how this Mini-Project is structured, the final answer in each part, i.e.~the value of $y(2)$, is not worth many points!  What you need to do is show understanding and correct application of each method.  Please show your work neatly.  Full credit requires evidence of clear understanding \emph{and} clear writing.
\end{quote}

\normalsize
\bigskip
\renewcommand{\baselinestretch}{1.1}

\textbf{One problem}.  Though many real-world ODEs are nonlinear, a large fraction of this course is about linear ODEs of second order.  The same is true of the problems on the Final Exam.  This Mini-Project is a review of the methods we know for such problems.  There is no new content.

Consider this ODE IVP for an unknown function $x(t)$:
\begin{equation}
x''+4x'+5x=0, \quad x(0)=0, \, x'(0)=10 \label{homo}
\end{equation}
We have many methods for solving this problem.  In parts A--E of this Mini-Project you are asked to use five different methods to compute, or approximate, $x(2)$.  Because this answer is the same in each case, you can check your work.

Also consider this closely-related non-homogeneous variation of the same problem:
\begin{equation}
x''+4x'+5x=5\sin t, \quad x(0)=0, \, x'(0)=10 \label{nonhomo}
\end{equation}
Parts G--I ask you to compute $x(2)$ for this problem using three different methods.

\medskip
\textbf{Many solution methods}.  The emphasis of your solutions, and of my grading, must be on whether you show clear understanding of the steps of the method, and whether those steps are correct.  Please consistently write ``$x(t)$'' for the solution, that is, with independent variable $t$ and dependent variable $x$.

\renewcommand{\labelenumi}{\Alph{enumi}.}
\begin{enumerate}
\item Use methods from \S4.3, \emph{auxiliary equation} methods, to solve \eqref{homo}.  Compute $x(2)$.
\item Use methods from Chapter 7, \emph{Laplace transform} methods, to solve \eqref{homo}.  Compute $x(2)$.
\item Use methods from Chapter 6, \emph{power series} methods, to solve \eqref{homo}.  In particular, start your solution with
    $$x(t) = \sum_{n=0}^\infty c_n t^n.$$
Note that you can use $x(0)$ and $x'(0)$ to immediately find $c_0$ and $c_1$.  Then determine the recurrence relation which (potentially) generates all of the coefficients $c_k$.  Use it to determine the values of $c_2,\dots,c_6$.  Your solution is now approximated by a polynomial of degree 6, i.e.~$x(t)\approx p(t)$.  What is your approximation of $x(2)$?  (\emph{It will not be close to the result from the other parts.})  Backing-up, what is your approximation of $x(1)$?  (\emph{This should be reasonably close to $x(1)$ computed from the solutions in A and B.})
\item FIXME system
\item FIXME numerical improved Euler
\item FIXME classify using 5.1.2 methods into under/over/critically
\item FIXME \eqref{nonhomo} \S4.4
\item FIXME \eqref{nonhomo} Chapter 7
\item FIXME \eqref{nonhomo} numerical \texttt{ode45}
\end{enumerate}
\end{document}
