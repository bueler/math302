\documentclass[12pt]{article}
\usepackage[top=1in, bottom=0.75in, left=1in, right=1in, headheight=1in, headsep=6pt]{geometry}

% Fonts.
\usepackage{mathptmx}
\usepackage[scaled=0.86]{helvet}
\renewcommand{\emph}[1]{\textsf{\textbf{#1}}}

% Misc packages.
\usepackage{amsmath,amssymb,latexsym}
\usepackage{graphicx}
\usepackage{array}
\usepackage{xcolor}
\usepackage{multicol}
\usepackage{tabularx,colortbl}
\usepackage{enumitem}
\usepackage[colorlinks=true]{hyperref}

% Paragraph spacing
\parindent 0pt
\parskip 6pt plus 1pt
\def\tableindent{\hskip 0.5 in}
\def\ts{\hskip 1.5 em}

\usepackage{fancyhdr}
\pagestyle{fancy}
\lhead{\large\sf\textbf{MATH F302 UX1 (Spring 2019)}}
\rhead{\large\sf\textbf{Regarding Proctoring}}

\newcommand{\localhead}[1]{\par\smallskip\textbf{#1}\nobreak\\}%
\def\heading#1{\localhead{\large\emph{#1}}}
\def\subheading#1{\localhead{\emph{#1}}}

\newenvironment{clist}%
{\bgroup\parskip 0pt\begin{list}{$\bullet$}{\partopsep 4pt\topsep 0pt\itemsep -2pt}}%
{\end{list}\egroup}%


\begin{document}

\phantom{foo}
%\bigskip
\cfoot{}


\heading{Proctored Quizzes and Exams}
The 11 Quizzes, 2 Midterm Exams, and Final Exam for this course are all \emph{proctored}.  This means there must be human supervisor to monitor the Quiz/Exam to maintain academic integrity according to standards established by the University of Alaska Fairbanks.

The proctor must be approved by UAF eCampus.  I will not use virtual proctoring.

I will design the Quizzes so that a well-prepared student can easily complete them in only half an hour.  Similarly the Midterm Exams will only have an hour of material while the Final will have two hours of material.  However, I want to make sure that you have enough time to finish these assessments.  Here are the official durations for proctored assessments:
\begin{itemize}
\item Quizzes must be completed in 45 minutes.
\item Midterm Exams must be completed in 90 minutes (1.5 hours).
\item the Final Exam must be completed in 150 minutes (2.5 hours).
\end{itemize}

The Quizzes and Exams can only be take on the dates listed in the schedule:

\centerline{\href{https://bueler.github.io/math302/schedule.pdf}{\tt bueler.github.io/math302/schedule.pdf}}

Note there are two-day periods for the Quizzes and three-day periods for the Exams.  If absolutely necessary, on an individual basis this may be modified to an earlier (not later) date.  Please contact me as soon as you know of a conflict.  If you are outside of Fairbanks then you will need to find a proctor right away, in the first week of class.

\heading{How to set it up}
Setting-up proctoring is the responsibility of the student!  (Quizzes and Exams cannot be made-up due to last-minute proctoring/testing center hour issues.)  See the website:

\medskip
\centerline{\href{https://ecampus.uaf.edu/current-students/arrange-a-proctored-exam/}{\tt ecampus.uaf.edu/current-students/arrange-a-proctored-exam/}}

\begin{itemize}
\item \textbf{Students in the Fairbanks region:}  If you are located in Fairbanks, please use the UAF eCampus Exam Center in Bunnell 131.  They are open 8am--5pm and they will have your exam ready for you.  No appointment is necessary but give yourself enough time to complete your exam before closing.
\item \textbf{Students outside of Fairbanks:}  See the instructions at the above website.  You must identify a proctor at least two weeks in advance.  Don't assume that the next Quiz/Exam will be scheduled just because the last one was; you need to ``request'' each assessment.  Other UA campuses may require appointment times; you may not be able to just drop-in.
\end{itemize}


\heading{First Quiz and Midterm}  The first Quiz is on January 23--24.  You must set up a proctor for a 45 minute period sometime during these two days.  It will cover only sections 1.1 and 1.2 or the textbook.

Midterm 1 is February 19--21.  You must set up a proctor for a 90 minute period sometime during these three days.  Topics/sections for Midterm 1 will be announced.

\vfill
\hfill \scriptsize document version: \today \normalsize

\end{document}
