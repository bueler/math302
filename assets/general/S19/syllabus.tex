\documentclass[12pt]{article}
\usepackage[top=1in, bottom=0.75in, left=1in, right=1in, headheight=1in, headsep=6pt]{geometry}

% Fonts.
\usepackage{mathptmx}
\usepackage[scaled=0.86]{helvet}
\renewcommand{\emph}[1]{\textsf{\textbf{#1}}}

% Misc packages.
\usepackage{amsmath,amssymb,latexsym}
\usepackage{graphicx}
\usepackage{array}
\usepackage{xcolor}
\usepackage{multicol}
\usepackage{tabularx,colortbl}
\usepackage{enumitem}
\usepackage[colorlinks=true]{hyperref}

% Paragraph spacing
\parindent 0pt
\parskip 6pt plus 1pt
\def\tableindent{\hskip 0.5 in}
\def\ts{\hskip 1.5 em}

\usepackage{fancyhdr}
\pagestyle{fancy}
\lhead{\large\sf\textbf{MATH F302 UX1 (Spring 2019)}}
\rhead{\large\sf\textbf{Syllabus}}

\newcommand{\localhead}[1]{\par\smallskip\textbf{#1}\nobreak\\}%
\def\heading#1{\localhead{\large\emph{#1}}}
\def\subheading#1{\localhead{\emph{#1}}}

\newenvironment{clist}%
{\bgroup\parskip 0pt\begin{list}{$\bullet$}{\partopsep 4pt\topsep 0pt\itemsep -2pt}}%
{\end{list}\egroup}%


\begin{document}

\phantom{foo}
%\bigskip
\cfoot{}

\heading{Instructor: Ed Bueler}

\quad \begin{tabularx}{\textwidth}{lX}
email        & \texttt{elbueler@alaska.edu} \\
office       & Chapman 306C \\
office hours \phantom{jfxdsd} & \href{http://bueler.github.io/OffHrs.htm}{\tt bueler.github.io/OffHrs.htm}
\end{tabularx}

\bigskip

\heading{Essentials}

\quad \begin{tabularx}{\textwidth}{lX}
\emph{Course Information} & \hspace{-3mm} \begin{tabular}[t]{ll}
                  number:  & MATH F302 \\
                  title:   & Differential Equations \\
                  section: & UX1 \qquad ($=$ online course)\\
                  credits: & $3.0$ \\
                  CRN:     & 34998 \\
                  \end{tabular} \\
 & \\
\emph{Prerequisite}      & Grade of at least C- in MATH 253 Calculus III or equivalent. \\
 & \\
\emph{Detailed Schedule} & \href{https://bueler.github.io/math302/schedule.pdf}{\tt bueler.github.io/math302/schedule.pdf} \\
 & \\
\emph{Websites} & \hspace{-3mm} \begin{tabular}[t]{ll}
                  \href{https://bueler.github.io/math302/}{\tt bueler.github.io/math302} \phantom{sdfjaldsj adslfj} & main course page \\
                  \href{https://piazza.com/uaf/spring2019/math302ux1/home}{\tt piazza.com/uaf/spring2019/math302ux1/home} & course communication \\
                  \href{https://webassign.net/}{\tt webassign.net} \, $\longleftarrow$ class key: \texttt{uaf 3982 1786} & homework \\
                  \href{https://classes.alaska.edu/}{\tt classes.alaska.edu} \, (Blackboard) & for grades \textsl{and solutions}
                  \end{tabular} \\
 & \\
\emph{Proctoring} & \hspace{-3mm} \begin{tabular}[t]{ll}
                  \href{https://bueler.github.io/math302/proctoring.pdf}{\tt bueler.github.io/math302/proctoring.pdf \phantom{s}} & see first \\
                  \href{https://ecampus.uaf.edu/current-students/arrange-a-proctored-exam/}{\tt ecampus.uaf.edu/current-students/} & setting up a proctor \\
                  \href{https://ecampus.uaf.edu/current-students/arrange-a-proctored-exam/}{\quad \tt arrange-a-proctored-exam} & 
                  \end{tabular} \\
 & \\
\emph{Required Text}     & \textsl{A First Course in Differential Equations with Modeling Applications}, 11th ed., Dennis G. Zill, 2018 (ISBN-13:\, \texttt{978-1337604994}) \\
 & \\
\emph{Optional Text}     & \textit{Student Solutions Manual for Zill's A First Course in Differential Equations with Modeling Applications}, 11th ed.~(ISBN-13:\, \texttt{978-1305965737}) \\
    & (Solutions to selected odd-numbered exercises; available on Amazon) \\
 & \\
\emph{Course Materials}  & $\bullet$\, Reliable internet access is required for any online course. \\
                         & $\bullet$\, Your UA e-mail address (\texttt{yourname@alaska.edu}) must work. \\
                         & $\bullet$\, A WebAssign Access Code is required.  (Bundled with required text.) \\
                         & $\bullet$\, A scanner/tablet/smartphone with software/app for scanning documents \\
                         & \quad and making PDFs out of them and/or writing on PDFs. \\
                         & $\bullet$\, A printer may be useful to print assignments or lecture notes, but it is \\
                         & \quad not required. \\
 & 
\end{tabularx}

\vfill

\newpage
\cfoot{\thepage}
\strut

\heading{Description and Course Goals}
Most laws of nature used in science take the form of differential equations.  So do many of the models used in engineering, finance, and the social sciences.  In fact, differential equations are the single most-important reason why students in technical majors are expected to learn calculus.

Differential equations describe smoothly-changing functions (solutions) using either ordinary or partial derivatives.  This course is about \textsl{ordinary differential equations}.  Partial differential equations are covered in MATH 421 Applied Analysis, which is a sequel to this course.

We will mostly use the derivatives, integrals, and series from calculus I (MATH 251) and II (MATH 252).  Some content from calculus III (MATH 253), e.g.~multivariable visualization, is also used.

Here is the catalog description:
\begin{quote}
\textsl{Nature and origin of differential equations, first order equations and solutions, linear differential equations with constant coefficients, systems of equations, power series solutions, operational methods, and applications.}
\end{quote}

A passing grade from this course indicates that you are able to:

\begin{clist}
\item understand the language of ordinary differential equation (ODE) initial value problems,
\item verify that a given function solves a differential equation,
\item use and construct basic models based on differential equations,
\item use well-known methods for generating solutions to common first-order ODEs,
\item use series and Laplace transform methods to solve certain linear ODEs,
\item understand and apply well-known numerical methods to solve initial value problems,
\item describe and understand linear systems of ODEs and their matrix exponential solution.
\end{clist}


\heading{Detailed Schedule}
The schedule of topics and due dates will be kept up-to-date at

\smallskip
\centerline{\href{https://bueler.github.io/math302/schedule.pdf}{\tt bueler.github.io/math302/schedule.pdf}}

You should consult this schedule frequently and routinely.  It is tentative; warning of significant changes will be given well in advance.

In order to allow you flexibility in your own schedule, all homework assignments will be available well in advance, with due dates each week.  Quizzes and exams will, however, only be available on the dates indicated on the schedule.


\heading{Course Communication}
We will use Piazza for public communication including announcements, questions, and discussions (among students and/or with the instructor).  Go here to sign up and get started:

\smallskip
\centerline{\href{https://piazza.com/uaf/spring2019/math302ux1}{\tt piazza.com/uaf/spring2019/math302ux1}}

When you sign up you will provide an email address.  It should be an address you check frequently; I expect you to receive all such messages in a timely manner.  (Your UA-generated email address will also be used for all private communication, for instance when I return graded work.)


\newpage
\strut

\heading{Online Instructional Methods}
This is an online course but it is neither independent study nor self-paced.  There will be regular due dates for completing all assignments and assessments (below).

Videos, links, and other material are posted in weekly modules at

\smallskip
\centerline{\href{https://bueler.github.io/math302/}{\tt bueler.github.io/math302}}

The videos by the instructor, and links to videos elsewhere, attempt to substitute for interactive in-person lecture.  Thoughtful viewing of this material is essential for success.  Students must actively seek explanations and examples for the challenging material in this course!


\heading{Assignments and assessments}
Weekly assignments consist of online WebAssign homework and written mini-project homework.  Student performance is assessed by proctored weekly quizzes and proctored exams.

\subheading{WebAssign homework} 
Online WebAssign homework will typically be due twice each week.  These exercise are generally routine; they give you immediate feedback on correctness and understanding.

\begin{clist}
\item Go to WebAssign \href{https://www.webassign.net/}{\tt www.webassign.net} and sign in with the class key: \quad \texttt{uaf 3982 1786}
\item You will need a WebAssign Access Code.  Texts purchased from the UAF bookstore include one.  WebAssign can be used in a short ``trial'' period.
\item You get 5 chances to get a problem correct. 
\item Each assignment is due at 11 pm on the date stated on the course \href{https://bueler.github.io/math302/schedule.pdf}{schedule}. 
\item No WebAssign scores are dropped.
\end{clist}

You are encouraged to use your text and a calculator to help solve WebAssign problems, but \textsl{you will need to draft calculations on paper}.  Use of sophisticated tools (e.g., Wolfram Alpha) will often undermine the benefit of the homework, and may leave you unprepared for the quizzes and exams.

\subheading{Mini-project homework}
Roughly every two weeks there will be a Mini-project.  Each Mini-project will be found in that week's module on the \href{https://bueler.github.io/math302/}{main course page}.  They will be distributed as PDFs.  You will get a link to a Google Form for submitting (attaching) your solution, which also needs to be a PDF.

This written homework includes more-challenging exercises, often regarding real situations modeled by differential equations.  You can use online resources.  You are encouraged to work with others if that is possible, but the solution you submit needs to be your own.  Because your solution will be read by another human being, presentation matters.  Your writing should be legible and you must show all relevant work; points will be deducted for poor presentation.  Generally you should prepare a first draft and then submit a final version for grading.

The Week 1 Mini-project is a review assignment covering high points in calculus.

\subheading{Quizzes}
Each 45 minute quiz will cover the material taught in the the previous week.  Quizzes are equally weighted, and are given under testing conditions.  Thus a proctor is needed.  Books, notes, and calculators are \emph{not} allowed.  \textsl{The lowest quiz grade will be dropped.}  Solutions to quizzes will be posted on Blackboard.

The quizzes will be held on the dates on the \href{https://bueler.github.io/math302/schedule.pdf}{schedule}.  Each quiz must be scheduled with a proctor during a fixed two-day period.  Quizzes cannot be made up except with a documented excused absence.

Performance on quizzes is the best predictor of exam performance.

\subheading{Exams}
There are two Midterm exams and a Final exam.  You will have 90 minutes to complete each Midterm and 150 minutes for the Final.  These must be taken during the three-day periods shown on the \href{https://bueler.github.io/math302/schedule.pdf}{schedule}; these must be scheduled with a proctor.  Solutions to the exams will be posted on Blackboard.


\heading{Grades}
Grades are determined as follows.
 
\begin{multicols}{2}

\begin{tabular}{|l|c|}
\hline
Webassign homework& 20\%\\
\hline
Mini-project homework & 10\% \\
\hline
Quizzes& 20\% \\
\hline
Midterm 1 & 15\% \\
\hline
Midterm 2 & 15\%  \\
\hline
Final Exam& 20\% \\
\hline
\textsl{total} & 100\%\\
\hline
\end{tabular}

\vskip 25pt

Letter grades will be assigned on this scale, which is a guarantee; I reserve the right to lower thresholds. 

\begin{tabular}{llll}
A+ & 97--100\% \quad\strut & C+ & 77--79\% \\
A & 93--96\% &  C & 70--76\% \\
A- & 90--92\% & C- & not given \\
B+ & 87--89\% & D+ & 67--69\% \\
B &  83--86\% & D & 63--66\% \\
B- & 80-82\% & D- & 60--62\% \\
 & & F  & $<$ 60\%
\end{tabular}
\end{multicols}


\vspace{-0.3in}

\heading{Office Hours}
I will hold office hours as shown at

\smallskip
\centerline{\href{http://bueler.github.io/OffHrs.htm}{\tt bueler.github.io/OffHrs.htm}}

These office hours will be both in-person and using Piazza.  if you are on campus you can meet me in my office, but otherwise I will be watching my computer for notifications.


\heading{Blackboard}
Only \textsl{grades} and \textsl{solutions to Mini-Projects, Quizzes, and Exams} will be available on BlackBoard (\href{https://classes.alaska.edu/}{\tt classes.alaska.edu}).  There is a link to Blackboard on the \href{https://bueler.github.io/math302/}{main course page}.


\heading{Tutoring and Resources}
UAF Math Services (\href{http://www.uaf.edu/dms/mathlab/}{\texttt{uaf.edu/dms/mathlab}}) offers the following tutoring:
\begin{clist}
	\item Free online tutoring.  Schedule an appointment at\, \href{https://fairbanks.go-redrock.com/}{\texttt{fairbanks.go-redrock.com}}.
	\item Walk-in tutoring, with no appointment needed, at the Math and Stat Lab, Chapman Building Room 305.  See\, \href{https://uaf.edu/dms/mathlab/math-and-stat-lab-schedul-1/}{the schedule at \texttt{uaf.edu/dms/mathlab}}\, for times and availability.  Only a subset of the tutors can help with MATH 302.
	\item Free one-on-one (or small group) tutoring in Chapman Building Room 201.  Schedule an appointment at\, \href{https://fairbanks.go-redrock.com/}{\texttt{fairbanks.go-redrock.com}}.
\end{clist}

Additional services:
\begin{clist}
	\item Student Support Services may offer free tutoring to students who qualify for their program.
	\item ASUAF may offer private tutoring for a small fee (based on student income).
\end{clist}


\newpage
\strut

\heading{Rules and Policies}
\vskip -20pt
\subheading{Participation}
Class participation is mandatory.  Students who stop participating in the course will be withdrawn.  Examples of inadequate participation include, but are not limited to:

\begin{clist}
\item not completing or not turning in \textbf{three} homework assignments (WebAssign or written)
\item repeatedly failing quizzes or exams
\end{clist}

\subheading{Disability Services}
The Office of Disability Services (ODS) implements the Americans with Disabilities Act to ensure that UAF students have equal access to the campus and course materials.  I will work with ODS (208 Whitaker, 474-5655) to provide reasonable accommodations to students with disabilities.

\subheading{Student Protections and Services}
Every qualified student is welcome in this class.  I am happy to work with you, ODS, Military and Veteran Services, Rural Student Services, etc.~to find reasonable accomodations.  Students at this university are protected against sexual harassment and discrimination (Title IX), and minors have additional protections. \textit{As required,} if I notice or am informed of \textit{certain types} of misconduct, then I am required to report it to the appropriate authorities.  For more information on your rights as a student and the resources available to you, please go to \href{https://www.uaf.edu/handbook/}{\texttt{www.uaf.edu/handbook}}.

\subheading{Incomplete Grade} 
An incomplete (I) grade will only be given in DMS courses if the student has completed the majority (normally all but the last three weeks) of a course with a grade of C or better, but for personal reasons beyond his/her control has been unable to complete the course during the regular term. Negligence or indifference are not acceptable reasons for the granting of an incomplete grade. 

\subheading{Late Withdrawals} 
A withdrawal after the deadline (currently 9 weeks into the semester) from a DMS course will normally be granted only in cases where the student is performing satisfactorily (i.e., C or better) in a course, but has exceptional reasons, beyond his/her control, for being unable to complete the course. These exceptional reasons should be detailed in writing to the instructor, department head and dean.

\subheading{No Early Final Examinations}
Final examinations for DMS courses shall not be held earlier than the date and time published in the official term schedule. Normally, a student will not be allowed to take a final exam early. Exceptions can be made by individual instructors, but should only be allowed in exceptional circumstances and in a manner which doesn't endanger the security of the exam.

\subheading{Academic Dishonesty}
Academic dishonesty, including cheating and plagiarism, will not be tolerated.  It is a violation of the Student Code of Conduct and will be punished according to UAF procedures.

\vfill
\hfill \scriptsize syllabus version: \today \normalsize

%\hrulefill \heading{FIXME: DON'T BELIEVE ANYTHING FROM HERE ON}

\end{document}
