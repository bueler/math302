\documentclass[12pt]{article}
\usepackage[top=1in, bottom=0.75in, left=1in, right=1in, headheight=1in, headsep=6pt]{geometry}

% Fonts.
\usepackage{mathptmx}
\usepackage[scaled=0.86]{helvet}
\renewcommand{\emph}[1]{\textsf{\textbf{#1}}}

% Misc packages.
\usepackage{amsmath,amssymb,latexsym}
\usepackage{graphicx}
\usepackage{array}
\usepackage{xcolor}
\usepackage{multicol}
\usepackage{tabularx,colortbl}
\usepackage{enumitem}
\usepackage[colorlinks=true]{hyperref}

% Paragraph spacing
\parindent 0pt
\parskip 6pt plus 1pt
\def\tableindent{\hskip 0.5 in}
\def\ts{\hskip 1.5 em}

\usepackage{fancyhdr}
\pagestyle{fancy}
\lhead{\large\sf\textbf{MATH F302 UX1 (Spring 2019)}}
\rhead{\large\sf\textbf{Syllabus}}

\newcommand{\localhead}[1]{\par\smallskip\textbf{#1}\nobreak\\}%
\def\heading#1{\localhead{\large\emph{#1}}}
\def\subheading#1{\localhead{\emph{#1}}}

\newenvironment{clist}%
{\bgroup\parskip 0pt\begin{list}{$\bullet$}{\partopsep 4pt\topsep 0pt\itemsep -2pt}}%
{\end{list}\egroup}%


\begin{document}

\phantom{foo}
%\bigskip
\cfoot{}

\heading{Instructor: Ed Bueler}

\quad \begin{tabularx}{\textwidth}{lX}
email        & \texttt{elbueler@alaska.edu} \\
office       & Chapman 306C \\
office hours \phantom{jfxdsd} & \href{http://bueler.github.io/OffHrs.htm}{\tt bueler.github.io/OffHrs.htm}
\end{tabularx}

\bigskip\bigskip

\heading{Essentials}

\quad \begin{tabularx}{\textwidth}{lX}
\emph{Course Information} & \hspace{-3mm} \begin{tabular}[t]{ll}
                  number:  & MATH F302 \\
                  title:   & Differential Equations \\
                  section: & UX1 \\
                  credits: & $3.0$ \\
                  CRN:     & 34998 \\
                  \end{tabular} \\
 & \\
\emph{Prerequisite}      & A grade of C- or better in MATH 253 Calculus III or equivalent is required. \\
 & \\
\emph{Detailed Schedule} & \href{https://bueler.github.io/math302/schedule.pdf}{\tt bueler.github.io/math302/schedule.pdf} \\
 & \\
\emph{Websites} & \hspace{-3mm} \begin{tabular}[t]{ll}
                  \href{https://bueler.github.io/math302/}{\tt bueler.github.io/math302} \phantom{sdfjaldsj adslfj} & main course page \\
                  \href{https://piazza.com/uaf/spring2019/math302ux1/home}{\tt piazza.com/uaf/spring2019/math302ux1/home} & course communication \\
                  \href{https://webassign.net/}{\tt webassign.net} \, $\longleftarrow$ class key: \texttt{uaf 3982 1786} & homework \\
                  \href{https://classes.alaska.edu/}{\tt classes.alaska.edu} \, (Blackboard) & only for grades
                  \end{tabular} \\
 & \\
\emph{Required Text}     & \textsl{A First Course in Differential Equations with Modeling Applications}, 11th ed., Dennis G. Zill, 2018 (ISBN-13:\, \texttt{978-1337604994}) \\
 & \\
\emph{Optional Text}     & \textit{Student Solutions Manual for Zill's A First Course in Differential Equations with Modeling Applications}, 11th ed.~(ISBN-13:\, \texttt{978-1305965737}) \\
    & (Solutions to selected odd-numbered exercises; available on Amazon) \\
 & \\
\emph{Course Materials}  & $\bullet$\, Reliable internet access is required for any online course. \\
                         & $\bullet$\, A WebAssign Access Code is required.  (It is bundled with the required \\
                         & \quad text above.) \\
                         & $\bullet$\, A scanner or smartphone with software/app for scanning documents \\
                         & \quad and making PDFs out of them. \\
                         & $\bullet$\, A printer is useful to print off lecture notes, but \textsl{not} required. \\
 & 
\end{tabularx}

\vfill

\newpage
\cfoot{\thepage}
\strut

\heading{Description, Course Goals \& Student Learning Outcomes}
Most laws of nature used in science take the form of differential equations, as do many of the models used in engineering, finance, and the social sciences.  Differential equations describe smoothly-changing functions (solutions) using either ordinary or partial derivatives.  This course is about \textsl{ordinary differential equations} so we mostly use the derivatives, integrals, and series covered in calculus I (MATH 251) and II (MATH 252).  However, multivariable visualization from calculus III (MATH 253) is also important.  Partial differential equations are covered in MATH 421 Applied Analysis, which is a sequel to this course.

Here is the catalog description:
\begin{quote}
\textsl{Nature and origin of differential equations, first order equations and solutions, linear differential equations with constant coefficients, systems of equations, power series solutions, operational methods, and applications.}
\end{quote}

A passing grade from this course indicates that you are able to:

\begin{clist}
\item understand the language of ordinary differential equation (ODE) initial value problems,
\item verify that a given function solves a differential equation,
\item use basic models based on differential equations,
\item use well-known methods for generating solutions to many first-order ODEs,
\item use series and Laplace transform methods to solve certain linear ODEs,
\item understand and apply well-known numerical methods to solve ODE initial value problems,
\item describe and understand linear systems of ODEs and their matrix exponential solution.
\end{clist}

\heading{Online Instructional Methods}
This is an online course but it is neither independent study nor self-paced.  There will be regular due dates for completing online and written homework, taking proctored weekly quizzes, and taking proctored midterm and final exams.

Videos, links, and other material are posted in weekly modules at

\smallskip
\centerline{\href{https://bueler.github.io/math302/}{\tt bueler.github.io/math302}}

The videos by the instructor, and links to videos elsewhere, attempt to substitute for interactive in-person lecture.  Thoughtful viewing of this material is essential for success, and students must take an active role in finding explanations and examples for the challenging material in this course.


\heading{Detailed Schedule}
The schedule of due dates will be kept up-to-date at

\smallskip
\centerline{\href{https://bueler.github.io/math302/schedule.pdf}{\tt bueler.github.io/math302/schedule.pdf}}

You should consult this detailed schedule frequently and routinely.  The schedule is tentative but warning of significant changes will be given well in advance.  In order to allow you flexibility in your own schedule, all assignments will be available well in advance, with due dates each week.  However the quizzes and exams will only be available on the dates indicated on the schedule.


\newpage
\strut

\heading{Course Communication}
We will use Piazza for announcements, questions, and discussions between students and the instructor.  Go here to sign up and get started:

\smallskip
\centerline{\href{https://piazza.com/uaf/spring2019/math302ux1}{\tt piazza.com/uaf/spring2019/math302ux1}}

When you sign up for Piazza, you will provide an email address.  This should be an address you check with some regularity; I expect you to receive all messages emailed to this account.

\heading{Office Hours and Communication}
I will have office hours on the schedule at

\smallskip
\centerline{\href{http://bueler.github.io/OffHrs.htm}{\tt bueler.github.io/OffHrs.htm}}

FIXME These office hours are both in-person and using online tools

\heading{Blackboard}
Only \textsl{grades} are available on BlackBoard (\href{https://classes.alaska.edu/}{\tt classes.alaska.edu}), which you can also access via the main course website.


\heading{Evaluation and Grades}
Grades are determined as follows.  (Each component of the grade is discussed below.)
 
\begin{multicols}{2}

\begin{tabular}{|c|c|}
\hline
Webassign Homework& 20\%\\
\hline
Mini-projects & 10\% \\
\hline
Quizes& 20\% \\
\hline
Midterm 1 & 15\% \\
\hline
Midterm 2 & 15\%  \\
\hline
Final Exam& 20\% \\
\hline
total& 100\%\\
\hline
\end{tabular}

\vskip 50pt

Letter grades will be assigned according to the following scale.
This scale is a guarantee; I reserve the right to lower the thresholds. 

\def\sts{\hskip 0.5em}
\strut\hbox to\hsize{\vbox{\halign{#\hfil\sts&#\hfil\ts&#\hfil\sts&#
\hfil\ts&#\hfil\sts&#\hfil   \cr
A+ & 97--100\% & C+ & 77--79\% & F  & $<$ 60\%\cr

A & 93--96\% &  C & 70--76\%&&\cr
A- & 90--92\% & C- & not given&&\cr
B+ & 87--89\% & D+ & 67--69\%&&\cr
B &  83--86\% & D & 63--66\%&&\cr
B- & 80-82\% & D- & 60--62\%&&\cr
}}\hfil}
\end{multicols}

\subheading{WebAssign} 
Online WebAssign homework will be assigned multiple times each week.  These problems consist of more routine exercises and and allow you to receive immediate feedback on correctness.  You are welcome to use your text and a calculator to help solve these problems, but the use of more sophisticated tools (e.g., Wolfram Alpha) will undermine the benefit to you of the homework, and may leave you unprepared for the quizzes and exams.

\begin{clist}
\item Go to WebAssign \href{https://www.webassign.net/}{\tt www.webassign.net} and sign in with the class key \quad \texttt{uaf 3982 1786}
\item You will need a WebAssign Access Code.  Texts purchased from the UAF bookstore include one.  WebAssign can be used in a short ``trial'' period.
\item You get 5 chances to get a problem correct. 
\item Each assignment is due at 11 pm on the date stated on the course \href{https://bueler.github.io/math302/schedule.pdf}{schedule}. 
\item Each WebAssign assignment is equally weighted. No scores are dropped.
\end{clist}


\subheading{Mini-projects}
FIXME consists of more challenging or interesting exercises, and allow you to practice presenting  a solution suitable for reading by another human being. You are encouraged to work with others to solve these problems.  However, when you write up your final solutions, you need to do so on your own.

Presentation matters. You must show all relevant work and your writing should be legible.  Points will be deducted for poor presentation.  You should prepare a first draft of your solutions, and then a final edition to be graded.


\heading{Quizzes}
The quiz will cover the material taught in the classes held since the previous quiz; specific topics can be found in the schedule on the course website.  Quizzes are equally weighted, and are given under testing conditions; books, notes, and calculators are not allowed.  Performance on the quizzes is a better indicator of exam performance, and how well you are learning the course material, than homework which may be done with the input of tutors/friends/internet/etc.

Quizzes cannot be made up except with a documented excused absence.  \emph{The lowest quiz grade will be dropped.}  Solutions to quizzes will be posted on the course webpage.


\heading{Midterms}
There are two midterm exams this semester, to be held on the dates in the schedule on the course website.


\heading{Final Exam}
The cumulative final exam will be held at the day/time listed in the online schedule. A make-up final exam will be given only in extenuating circumstances, for documented and excused reasons at the discretion of the instructors.


\heading{Tutoring and Resources}
UAF Math Services (\href{http://www.uaf.edu/dms/mathlab/}{\texttt{uaf.edu/dms/mathlab}}) offers the following tutoring:
\begin{clist}
	\item Free online tutoring.  Schedule an appointment at\, \href{https://fairbanks.go-redrock.com/}{\texttt{fairbanks.go-redrock.com}}.
	\item Walk-in tutoring, with no appointment needed, at the Math and Stat Lab, Chapman Building Room 305.  See\, \href{https://uaf.edu/dms/mathlab/math-and-stat-lab-schedul-1/}{the schedule at \texttt{uaf.edu/dms/mathlab}}\, for times and availability.  Only a subset of the tutors can help with MATH 302.
	\item Free one-on-one (or small group) tutoring in Chapman Building Room 201.  Schedule an appointment at\, \href{https://fairbanks.go-redrock.com/}{\texttt{fairbanks.go-redrock.com}}.
\end{clist}

Additional services:
\begin{clist}
	\item Student Support Services may offer free tutoring to students who qualify for their program.
	\item ASUAF may offer private tutoring for a small fee (based on student income).
\end{clist}


\newpage
\strut

\heading{Rules and Policies}
\vskip -20pt
\subheading{Participation}
Class participation is mandatory.  Students who stop participating in the course will be withdrawn.  Examples of inadequate participation include, but are not limited to:

\begin{clist}
\item not completing or not turning in \textbf{three} homework assignments (WebAssign or written)
\item repeatedly failing quizzes or exams with no attempt at remediation
\end{clist}

\subheading{Disability Services}
The Office of Disability Services (ODS) implements the Americans with Disabilities Act to ensure that UAF students have equal access to the campus and course materials.  I will work with ODS (208 Whitaker, 474-5655) to provide reasonable accommodations to students with disabilities.

\subheading{Student Protections and Services}
Every qualified student is welcome in this class.  I am happy to work with you, ODS, Military and Veteran Services, Rural Student Services, etc.~to find reasonable accomodations.  Students at this university are protected against sexual harassment and discrimination (Title IX), and minors have additional protections. \textit{As required,} if I notice or am informed of \textit{certain types} of misconduct, then I am required to report it to the appropriate authorities.  For more information on your rights as a student and the resources available to you, please go to \href{https://www.uaf.edu/handbook/}{\texttt{www.uaf.edu/handbook}}.

\subheading{Incomplete Grade} 
An incomplete (I) grade will only be given in DMS courses if the student has completed the majority (normally all but the last three weeks) of a course with a grade of C or better, but for personal reasons beyond his/her control has been unable to complete the course during the regular term. Negligence or indifference are not acceptable reasons for the granting of an incomplete grade. 

\subheading{Late Withdrawals} 
A withdrawal after the deadline (currently 9 weeks into the semester) from a DMS course will normally be granted only in cases where the student is performing satisfactorily (i.e., C or better) in a course, but has exceptional reasons, beyond his/her control, for being unable to complete the course. These exceptional reasons should be detailed in writing to the instructor, department head and dean.

\subheading{No Early Final Examinations}
Final examinations for DMS courses shall not be held earlier than the date and time published in the official term schedule. Normally, a student will not be allowed to take a final exam early. Exceptions can be made by individual instructors, but should only be allowed in exceptional circumstances and in a manner which doesn't endanger the security of the exam.

\subheading{Academic Dishonesty}
Academic dishonesty, including cheating and plagiarism, will not be tolerated.  It is a violation of the Student Code of Conduct and will be punished according to UAF procedures.

\vfill
\hfill \scriptsize syllabus version: \today \normalsize

%\hrulefill \heading{FIXME: DON'T BELIEVE ANYTHING FROM HERE ON}

\end{document}
