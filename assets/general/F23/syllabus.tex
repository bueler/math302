\documentclass[12pt]{article}
\usepackage[top=1in, bottom=0.75in, left=1in, right=1in, headheight=1in, headsep=6pt]{geometry}

% Fonts.
\usepackage{mathptmx}
\usepackage[scaled=0.86]{helvet}
\renewcommand{\emph}[1]{\textsf{\textbf{#1}}}

% Misc packages.
\usepackage{amsmath,amssymb,latexsym}
\usepackage{graphicx}
\usepackage{array}
\usepackage{xcolor}
\usepackage{multicol}
\usepackage{tabularx,colortbl}
\usepackage{enumitem}
\usepackage[colorlinks=true]{hyperref}

% Paragraph spacing
\parindent 0pt
\parskip 6pt plus 1pt
\def\tableindent{\hskip 0.5 in}
\def\ts{\hskip 1.5 em}

\usepackage{fancyhdr}
\pagestyle{fancy}
\lhead{\large\sf\textbf{MATH F302 Differential Equations (Fall 2023)}}
\rhead{\sf version \today}

\newcommand{\localhead}[1]{\par\smallskip\textbf{#1}\nobreak\\}%
\def\heading#1{\localhead{\large\emph{#1}}}
\def\subheading#1{\localhead{\emph{#1}}}

\newenvironment{clist}%
{\bgroup\parskip 0pt\begin{list}{$\bullet$}{\partopsep 4pt\topsep 0pt\itemsep -2pt}}%
{\end{list}\egroup}%


\begin{document}

\phantom{foo}
\begin{center}
{\LARGE\sf \textbf{Syllabus}}
\end{center}
%\bigskip
\cfoot{}

\heading{Instructor: Ed Bueler}

\quad \begin{tabularx}{\textwidth}{lX}
email        & \href{mailto:elbueler@alaska.edu}{\texttt{elbueler@alaska.edu}} \\
office       & Chapman 306C \\
office hours \phantom{jfxdsd} & \href{http://bueler.github.io/OffHrs.htm}{\tt bueler.github.io/OffHrs.htm}
\end{tabularx}

\bigskip

\heading{Essentials}

\quad \begin{tabularx}{\textwidth}{lX}
\emph{Course Information} & \hspace{-3mm} \begin{tabular}[t]{l}
                  MATH F302 Differential Equations (3.0 credits) \\
                  CRN:\, 72953 (section 901) \\
                  time:\, 10:30--11:30am \\
                  room:\, Gruening 408 \\
                  \end{tabular} \\
 & \\
\emph{Prerequisite}      & Grade of at least C- in MATH 253 Calculus III or equivalent. \\
 & \\
\emph{Websites} & \hspace{-3mm} \begin{tabular}[t]{ll}
                  \href{https://bueler.github.io/math302/}{\tt bueler.github.io/math302} \phantom{sdfjaldsj adslfj} & main course page \\
                  \href{https://canvas.alaska.edu/courses/16214}{\tt canvas.alaska.edu/courses/16214} \, (Canvas) & for grades \textsl{and solutions}
                  \end{tabular} \\
 & \\
\emph{Required Text}     & \textsl{A First Course in Differential Equations with Modeling Applications}, 11th ed., Dennis G. Zill, 2018 (ISBN-13:\, \texttt{978-1337604994}) \\
 & \\
\emph{\underline{Optional} Text}     & \textit{Student Solutions Manual for Zill's A First Course in Differential Equations with Modeling Applications}, 11th ed.~(ISBN-13:\, \texttt{978-1305965737}) \\
    & (Solutions to selected odd-numbered exercises; available on Amazon) \\
 & \\
\emph{Course Materials}  & $\bullet$\, A scanner/tablet/smartphone with software/app for scanning documents \\
                         & \quad and making PDFs out of them and/or writing on PDFs. \\
                         & $\bullet$\, A printer may be useful to print assignments or lecture notes, but it is \\
                         & \quad not required. \\
 & 
\end{tabularx}

\vfill

%\newpage
\cfoot{\thepage}
%\strut

\heading{Description}
Most physical laws of nature take the form of differential equations.  So do many of the models used in engineering, finance, and the social sciences.  Differential equations describe smoothly-changing functions (solutions) using either ordinary or partial derivatives.  This course is about \textsl{ordinary differential equations}.  Partial differential equations are covered in MATH 421 Applied Analysis, which is a sequel to this course.

Differential equations are the single most-important reason why students in technical majors are expected to learn calculus.  We will mostly use the derivatives, integrals, and series from calculus I (MATH 251) and II (MATH 252).  Some content from calculus III (MATH 253), especially multivariable visualization, is also used.


\bigskip\bigskip
\clearpage\newpage
\heading{Course Goals and Outcomes}
Here is the catalog description:
\begin{quote}
\textsl{Nature and origin of differential equations, first order equations and solutions, linear differential equations with constant coefficients, systems of equations, power series solutions, operational methods, and applications.}
\end{quote}

A passing grade from this course indicates that you are able to:

\begin{clist}
\item understand the language of ordinary differential equation (ODE) initial value problems,
\item verify that a given function solves a differential equation,
\item use and construct basic models based on differential equations,
\item use well-known methods for generating solutions to common first-order ODEs,
\item use series and Laplace transform methods to solve certain linear ODEs,
\item understand linear systems of ODEs and their matrix-exponential solutions, and
\item understand and apply well-known numerical methods to solve initial value problems.
\end{clist}


\heading{Detailed Schedule}
The schedule of topics and due dates will be kept up-to-date at

\smallskip
\centerline{\href{https://bueler.github.io/math302/assets/general/F23/schedule.pdf}{\tt bueler.github.io/math302/assets/general/F23/schedule.pdf}}

You should consult this schedule frequently and routinely.  It is tentative and subject to change.


\heading{Assignments and assessments}
Each textbook section we cover will correspond to a Homework assignment.  These are graded for completion only; see below.  Student performance is assessed by six in-class Quizzes and two in-class Exams.


\subheading{Quizzes}
FIXME

Each 45 minute quiz will cover the material taught in the the previous week.  Quizzes are equally weighted, and are given under testing conditions.  Thus a proctor is needed.  Books, notes, and calculators are \emph{not} allowed.  \textsl{The lowest quiz grade will be dropped.}  Solutions to quizzes will be posted on Blackboard.

The quizzes will be held on the dates on the \href{https://bueler.github.io/math302/assets/general/F23/schedule.pdf}{schedule}.  Each quiz must be scheduled with a proctor during a fixed two-day period.  Quizzes cannot be made up except with a documented excused absence.

Performance on quizzes is the best predictor of exam performance.

\subheading{Exams}
FIXME

There are two Midterm exams and a Final exam.  You will have 90 minutes to complete each Midterm and 150 minutes for the Final.  These must be taken during the three-day periods shown on the \href{https://bueler.github.io/math302/assets/general/F23/schedule.pdf}{schedule}; these must be scheduled with a proctor.  Solutions to the exams will be posted on Blackboard.


\heading{Grades}
FIXME

Grades are determined as follows.
 
\begin{multicols}{2}

\begin{tabular}{|l|c|}
\hline
Homework & 20\%\\
\hline
Quizzes& 20\% \\
\hline
Midterm Exam & 15\%  \\
\hline
Final Exam& 20\% \\
\hline
\textsl{total} & 100\%\\
\hline
\end{tabular}

\vskip 25pt

Letter grades will be assigned on this scale, which is a guarantee; I reserve the right to lower thresholds. 

\begin{tabular}{llll}
A+ & 97--100\% \quad\strut & C+ & 77--79\% \\
A & 93--96\% &  C & 70--76\% \\
A- & 90--92\% & C- & not given \\
B+ & 87--89\% & D+ & 67--69\% \\
B &  83--86\% & D & 63--66\% \\
B- & 80-82\% & D- & 60--62\% \\
 & & F  & $<$ 60\%
\end{tabular}
\end{multicols}


\vspace{-0.3in}

\heading{Office Hours}
I will hold office hours in Chapman 306C as shown at

\smallskip
\centerline{\href{http://bueler.github.io/OffHrs.htm}{\tt bueler.github.io/OffHrs.htm}}


\heading{Tutoring and Resources}
UAF Math Services (\href{http://www.uaf.edu/dms/mathlab/}{\texttt{uaf.edu/dms/mathlab}}) offers the following tutoring:
\begin{clist}
	\item Free online tutoring.
	\item Walk-in tutoring, with no appointment needed, at the Math and Stat Lab, Chapman Building Room 305.  See the schedule at \, \href{http://www.uaf.edu/dms/mathlab/}{\texttt{uaf.edu/dms/mathlab}}\, for times and availability.  Only a subset of the tutors can help with MATH 302.
	\item Free one-on-one (or small group) tutoring in Chapman Building Room 201.
\end{clist}

Additional services:
\begin{clist}
	\item Student Support Services may offer free tutoring to students who qualify for their program.
	\item ASUAF may offer private tutoring for a small fee (based on student income).
\end{clist}


\newpage
\strut

\heading{Rules and Policies}
\vskip -20pt
\subheading{Participation}
Class participation is mandatory.  Students who stop participating in the course will be withdrawn.  Examples of inadequate participation include, but are not limited to:

\begin{clist}
\item not completing or not turning in \textbf{three} homework assignments (WebAssign or written)
\item repeatedly failing quizzes or exams
\end{clist}

\subheading{Disability Services}
The Office of Disability Services (ODS) implements the Americans with Disabilities Act to ensure that UAF students have equal access to the campus and course materials.  I will work with ODS (208 Whitaker, 474-5655) to provide reasonable accommodations to students with disabilities.

\subheading{Student Protections and Services}
Every qualified student is welcome in this class.  I am happy to work with you, ODS, Military and Veteran Services, Rural Student Services, etc.~to find reasonable accomodations.  Students at this university are protected against sexual harassment and discrimination (Title IX), and minors have additional protections. \textit{As required,} if I notice or am informed of \textit{certain types} of misconduct, then I am required to report it to the appropriate authorities.  For more information on your rights as a student and the resources available to you, please go to \href{https://www.uaf.edu/handbook/}{\texttt{www.uaf.edu/handbook}}.

\subheading{Incomplete Grade} 
An incomplete (I) grade will only be given in DMS courses if the student has completed the majority (normally all but the last three weeks) of a course with a grade of C or better, but for personal reasons beyond his/her control has been unable to complete the course during the regular term. Negligence or indifference are not acceptable reasons for the granting of an incomplete grade. 

\subheading{Late Withdrawals} 
A withdrawal after the deadline (currently 9 weeks into the semester) from a DMS course will normally be granted only in cases where the student is performing satisfactorily (i.e., C or better) in a course, but has exceptional reasons, beyond his/her control, for being unable to complete the course. These exceptional reasons should be detailed in writing to the instructor, department head and dean.

\subheading{No Early Final Examinations}
Final examinations for DMS courses shall not be held earlier than the date and time published in the official term schedule. Normally, a student will not be allowed to take a final exam early. Exceptions can be made by individual instructors, but should only be allowed in exceptional circumstances and in a manner which doesn't endanger the security of the exam.

\subheading{Academic Dishonesty}
Academic dishonesty, including cheating and plagiarism, will not be tolerated.  It is a violation of the Student Code of Conduct and will be punished according to UAF procedures.

\vfill
\hfill \scriptsize syllabus version: \today \normalsize

\end{document}
