\documentclass[12pt]{article}

% Layout.
\usepackage[top=1.2in, bottom=0.75in, left=1in, right=1in, headheight=1.0in, headsep=0pt]{geometry}

% Fonts.
\usepackage{mathptmx}
\usepackage[scaled=0.86]{helvet}
\renewcommand{\emph}[1]{\textsf{\textbf{#1}}}

% TiKZ.
\usepackage{tikz, pgfplots}
\usetikzlibrary{calc}
\pgfplotsset{my style/.append style={axis x line=middle, axis y line=middle, xlabel={$x$}, ylabel={$y$}}}
\pgfplotsset{compat=1.16}

% Misc packages.
\usepackage{amsmath,amssymb,latexsym,bm,array,multicol,enumitem}
\usepackage{graphicx}
\usepackage{xcolor}

% Commands to set various header/footer components.
\makeatletter
\def\doctitle#1{\gdef\@doctitle{#1}}
\doctitle{Use {\tt\textbackslash doctitle\{MY LABEL\}}.}
\def\docdate#1{\gdef\@docdate{#1}}
\docdate{Use {\tt\textbackslash docdate\{MY DATE\}}.}
\def\doccourse#1{\gdef\@doccourse{#1}}
\let\@doccourse\@empty
\def\docscoring#1{\gdef\@docscoring{#1}}
\let\@docscoring\@empty
\def\docversion#1{\gdef\@docversion{#1}}
\let\@docversion\@empty
\makeatother

% Headers and footers layout.
\makeatletter
\usepackage{fancyhdr}
\pagestyle{fancy}
\fancyhf{} % Clears all headers/footers.
\lhead{\emph{\@doctitle\hfill\@docdate} \medskip
\ifnum \value{page} > 1\relax\else\\
\emph{Name: \rule{3.5in}{1pt}\hfill \@docscoring}
\fi}

\rfoot{\emph{\@docversion}}
\lfoot{\emph{\@doccourse}}
\cfoot{\emph{\thepage}}
\renewcommand{\headrulewidth}{0pt}%
\makeatother

% Paragraph spacing
\parindent 0pt
\parskip 6pt plus 1pt

% A problem is a section-like command. Use \problem{5} for a problem worth 5 points.
\newcounter{probcount}
\newcounter{subprobcount}
\setcounter{probcount}{0}

\newcommand{\problem}[1]{%
\par
\addvspace{4pt}%
\setcounter{subprobcount}{0}%
\stepcounter{probcount}%
\makebox[0pt][r]{\emph{\arabic{probcount}.}\hskip1ex}\emph{[#1 points]}\hskip1ex}

\newcommand{\thesubproblem}{\emph{\alph{subprobcount}.}}

% Subproblems are an enumerate-like environment with a consistent
% numbering scheme.  Use \begin{subproblems}\item...\item...\end{subproblems}
\newenvironment{subproblems}{%
\begin{enumerate}[itemindent=8pt,leftmargin=0pt]%
\setcounter{enumi}{\value{subprobcount}}%
\renewcommand{\labelenumi}{\emph{\textsl{\alph{enumi})}} \,}%
}%
{%
\setcounter{subprobcount}{\value{enumi}}%
\end{enumerate}%
}

% Blanks for answers in normal and math mode.
\newcommand{\blank}[1]{\rule{#1}{0.75pt}}
\newcommand{\mblank}[1]{\underline{\hspace{#1}}}
\def\emptybox(#1,#2){\framebox{\parbox[c][#2]{#1}{\rule{0pt}{0pt}}}}

% Misc.
\renewcommand{\d}{\displaystyle}
\newcommand{\ds}{\displaystyle}
\newcommand{\threeopts}{{\small \hspace{-6mm} $\begin{matrix} \text{\textsc{converges}} \\ \text{\textsc{absolutely}} \end{matrix}$ \qquad\qquad $\begin{matrix} \text{\textsc{converges}} \\ \text{\textsc{conditionally}} \end{matrix}$ \qquad\qquad \textsc{diverges}} \bigskip}

\newcommand{\ba}{\mathbf{a}}
\newcommand{\bb}{\mathbf{b}}
\newcommand{\bc}{\mathbf{c}}
\newcommand{\bi}{\mathbf{i}}
\newcommand{\bj}{\mathbf{j}}
\newcommand{\bk}{\mathbf{k}}
\newcommand{\bn}{\mathbf{n}}
\newcommand{\br}{\mathbf{r}}
\newcommand{\bu}{\mathbf{u}}
\newcommand{\bv}{\mathbf{v}}
\newcommand{\bw}{\mathbf{w}}

\newcommand{\bT}{\mathbf{T}}

\newcommand{\grad}{\nabla}
\newcommand{\Div}{\nabla\cdot}

\newcommand{\ip}[2]{\mathrm{\left<#1,#2\right>}}
\newcommand{\vv}[2]{\mathrm{\left<#1,#2\right>}}
\newcommand{\vvv}[3]{\mathrm{\left<#1,#2,#3\right>}}


\doctitle{Math 302 Differential Equations: \,{\large Quiz 2}}
\docdate{27 September, 2023}
\doccourse{}
\docversion{}
\docscoring{\fbox{{\LARGE \strut}\hspace{0.8in} / 25}}

\begin{document}
25 minutes maximum.  No aids (book, calculator, etc.) are permitted.  Show all work and use proper notation for full credit.  Answers should be in reasonably-simplified form.  25 points possible.

% like 2.1 #3ac (not assigned)
\problem{4}  
The direction field of the following differential equation is shown: \quad $\ds \frac{dy}{dx} = 1 - x y$.

\noindent For each of the following points,

\noindent sketch an approximate solution curve.
\begin{subproblems}
\item $y(0)=0$
\item $y(2)=2$
\end{subproblems}

\vspace{-35mm}
\hfill \includegraphics[width=0.55\textwidth]{figs/dirfield3.png}

% like 2.3 example #7
\problem{4} 
The same differential equation as in the previous question is \emph{linear}: \quad $\ds \frac{dy}{dx} = 1 - x y$.

\noindent Find the general solution.  You may write the solution in integral form if you do not know how to do an integral.
\vfill

\clearpage\newpage
% exactly 2.1 #7 (assigned)
\problem{4}  Find the general solution by separation of variables:
	$$\frac{dy}{dx} = e^{3x+2y}  \hspace{5.5in}$$
\vfill

% exactly 2.4 #13 (assigned)
\problem{5}  Determine whether this differential equation is exact.  If it is exact, solve it:
    $$x \frac{dy}{dx} = 2 x e^x - y + 6 x^2  \hspace{5.0in}$$
\vfill


\clearpage\newpage
\problem{8}  For each of the following initial value problems, determine if the differential equation is separable, linear, or exact.  Then find the solution.

\begin{subproblems}
% exactly 2.3 #31 (not assigned)
\item \quad $\ds x \frac{dy}{dx} + y = 4x + 1, \quad y(1)=8$
\vfill

% simpler than 2.2 #17 (assigned), with added initial value
\item \quad $\ds\frac{dP}{dt} = - P^2, \quad P(0)=3$
\vfill
\end{subproblems}


\clearpage\newpage
% 2.2 example 3
\noindent \emph{Extra Credit. [1 point]} \, The following differential equation (DE) is separable: \quad $\ds \frac{dy}{dx}=y^2-4$.

\noindent Suppose that we want to solve the initial value problem for this DE with $y(3)=2$.  Explain very briefly why the separable (separation of variables) technique breaks down immediately.  Then write down the solution anyway, without any calculation.
\vspace{2.0in}

\noindent \hrule

\bigskip
\centerline{\footnotesize \textsc{extra space}}
\vfill
\end{document}