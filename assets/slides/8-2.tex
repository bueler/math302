\documentclass[urlcolor=blue,dvipsnames]{beamer}

\usepackage[utf8]{inputenc}
\usepackage{fancybox,fancyvrb}
\usepackage{environ,xspace,empheq}

\usepackage{tikz}
\usetikzlibrary{arrows.meta,decorations.markings,decorations.pathreplacing,fadings,positioning}

\hypersetup{colorlinks,linkcolor=,urlcolor=cyan}

\beamertemplatenavigationsymbolsempty
\setbeamertemplate{footline}[frame number]
\usetheme{Pittsburgh}

%\makeatletter
%\newcommand{\tinytiny}{\@setfontsize{\tinytiny}{4pt}{4pt}}
%\makeatother

\newcommand\enumnum[1]{{\renewcommand{\insertenumlabel}{#1}%
      \usebeamertemplate{enumerate item} \,}}

\newcommand{\bA}{\mathbf{A}}
\newcommand{\bF}{\mathbf{F}}
\newcommand{\bK}{\mathbf{K}}
\newcommand{\bX}{\mathbf{X}}

\newcommand{\grad}{\nabla}
\newcommand{\ih}{\boldsymbol{\hat{\textbf{\i}}}}
\newcommand{\jh}{\boldsymbol{\hat{\textbf{\j}}}}
\newcommand{\vF}{\boldsymbol{\vec{\textbf{F}}}}
\newcommand{\Matlab}{\textsc{Matlab}\xspace}
\newcommand{\Octave}{\textsc{Octave}\xspace}


\title{8.2 Homogeneous linear systems \\ of first-order ODEs}

\subtitle{a lesson for MATH F302 Differential Equations}

\author{Ed Bueler, Dept.~of Mathematics and Statistics, UAF}

\date{\tiny \today}


\begin{document}
\setbeamertemplate{itemize item}{$\bullet$}
\setbeamertemplate{itemize subitem}{$\circ$}
\renewcommand{\thefootnote}{{\color{green} \arabic{footnote}}}

\begin{frame}
\titlepage

\centerline{\tiny for textbook: \, D. Zill, \emph{A First Course in Differential Equations with Modeling Applications}, 11th ed.}
%\color{green!40!blue}
\end{frame}


\begin{frame}{homogeneous linear systems of ODEs}

\begin{itemize}
\item consider
    $$\bX' = \bA \bX$$
\item which means
    $$\frac{d}{dt} \begin{pmatrix} x_1 \\ x_2 \\ \vdots \\ x_n \end{pmatrix} = \begin{pmatrix}
a_{11} & a_{12} & \dots & a_{1n} \\
a_{21} & a_{22} &       & a_{2n} \\
 \vdots&        & \ddots& \vdots \\
a_{n1} & a_{n2} & \dots & a_{nn}
\end{pmatrix}     \begin{pmatrix} x_1 \\ x_2 \\ \vdots \\ x_n \end{pmatrix}$$
    \begin{itemize}
    \item the solution is a list of functions $x_1(t),\dots,x_n(t)$ which we combine into a vector $\bX(t)$
    \end{itemize}
\item in sections 8.2 and 8.4 we assume $\bA$ is a matrix of constants
\end{itemize}
\end{frame}


\begin{frame}{how do you solve the simplest ODEs?}

\begin{itemize}
\item \emph{ODE 1}.  how do you solve for $y(t)$?:
    $$y'=my$$
    \begin{quote}
    \textbf{answer:}  the solution is an exponential $y(x) = c e^{mt}$
    \end{quote}
\item \emph{ODE 2}.  how do you solve for $y(t)$ if $a_i$ are constant?:
    $$a_2 y'' + a_1 y' + a_0 y = 0$$
    \begin{quote}
    \textbf{answer:}  try an exponential
        $$y(x) = e^{mt}$$
    and get an auxiliary equation which will determine $m$,
        $$a_2 m^2 + a_1 m + a_0=0$$
    \end{quote}
\end{itemize}
\end{frame}


\begin{frame}{how do you solve the simplest ODEs?}

\begin{itemize}
\item \emph{ODE 3}.  how do you solve for $\bX(t)$ if $\bA$ is a constant matrix?:
    $$\bX' = \bA \bX$$
    \begin{quote}
    \textbf{answer:}  try an exponential times a vector
        $$\bX(t) = \bK e^{\lambda t}$$
    and get an auxiliary eqn which will determine $\lambda$,
        $$\text{\textbf{[what goes here?]}}$$
    \end{quote}
\item $\lambda$ is just an unknown scalar like $m$ in ODE 2
\end{itemize}
\end{frame}


\begin{frame}{X}

\begin{itemize}
\item X
\end{itemize}
\end{frame}


\begin{frame}{X}

\begin{itemize}
\item X
\end{itemize}
\end{frame}


\begin{frame}{X}

\begin{itemize}
\item X
\end{itemize}
\end{frame}


\begin{frame}{X}

\begin{itemize}
\item X
\end{itemize}
\end{frame}


\begin{frame}{X}

\begin{itemize}
\item X
\end{itemize}
\end{frame}


\begin{frame}{X}

\begin{itemize}
\item X
\end{itemize}
\end{frame}

\begin{frame}{expectations}

\begin{itemize}
\item just watching this video is \emph{not} enough!
     \begin{itemize}
     \item see ``found online'' videos and stuff at

     \centerline{\href{https://bueler.github.io/math302/week14.html}{\tt \color{cyan} bueler.github.io/math302/week14.html}}
     \item \emph{read} \S8.2
         \begin{itemize}
         \item you \alert{are responsible} for the ``distinct real eigenvalues'' and the ``complex eigenvalues'' cases
         \item you are not responsible for the ``repeated eigenvalues'' case
         \end{itemize}
     \item \emph{do} the WebAssign exercises for section 8.2
     \end{itemize}
\end{itemize}
\end{frame}

\end{document}

