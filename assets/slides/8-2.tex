\documentclass[urlcolor=blue,dvipsnames]{beamer}

\usepackage[utf8]{inputenc}
\usepackage{fancybox,fancyvrb}
\usepackage{environ,xspace,empheq}

\usepackage{tikz}
\usetikzlibrary{arrows.meta,decorations.markings,decorations.pathreplacing,fadings,positioning}

\hypersetup{colorlinks,linkcolor=,urlcolor=cyan}

\beamertemplatenavigationsymbolsempty
\setbeamertemplate{footline}[frame number]
\usetheme{Pittsburgh}

%\makeatletter
%\newcommand{\tinytiny}{\@setfontsize{\tinytiny}{4pt}{4pt}}
%\makeatother

\newcommand\enumnum[1]{{\renewcommand{\insertenumlabel}{#1}%
      \usebeamertemplate{enumerate item} \,}}

\newcommand{\bA}{\mathbf{A}}
\newcommand{\bF}{\mathbf{F}}
\newcommand{\bI}{\mathbf{I}}
\newcommand{\bK}{\mathbf{K}}
\newcommand{\bM}{\mathbf{M}}
\newcommand{\bX}{\mathbf{X}}
\newcommand{\bZ}{\mathbf{Z}}


\newcommand{\grad}{\nabla}
\newcommand{\ih}{\boldsymbol{\hat{\textbf{\i}}}}
\newcommand{\jh}{\boldsymbol{\hat{\textbf{\j}}}}
\newcommand{\vF}{\boldsymbol{\vec{\textbf{F}}}}
\newcommand{\Matlab}{\textsc{Matlab}\xspace}
\newcommand{\Octave}{\textsc{Octave}\xspace}


\title{8.2 Homogeneous linear systems \\ of first-order ODEs}

\subtitle{a lesson for MATH F302 Differential Equations}

\author{Ed Bueler, Dept.~of Mathematics and Statistics, UAF}

\date{\tiny \today}


\begin{document}
\setbeamertemplate{itemize item}{$\bullet$}
\setbeamertemplate{itemize subitem}{$\circ$}
\renewcommand{\thefootnote}{{\color{green} \arabic{footnote}}}

\begin{frame}
\titlepage

\centerline{\tiny for textbook: \, D. Zill, \emph{A First Course in Differential Equations with Modeling Applications}, 11th ed.}
%\color{green!40!blue}
\end{frame}


\begin{frame}{homogeneous linear systems of ODEs}

\begin{itemize}
\item consider
    $$\bX' = \bA \bX$$
\item which means
    $$\frac{d}{dt} \begin{pmatrix} x_1 \\ x_2 \\ \vdots \\ x_n \end{pmatrix} = \begin{pmatrix}
a_{11} & a_{12} & \dots & a_{1n} \\
a_{21} & a_{22} &       & a_{2n} \\
 \vdots&        & \ddots& \vdots \\
a_{n1} & a_{n2} & \dots & a_{nn}
\end{pmatrix}     \begin{pmatrix} x_1 \\ x_2 \\ \vdots \\ x_n \end{pmatrix}$$
    \begin{itemize}
    \item in sections 8.2 and 8.4 we assume $\bA$ is a matrix of constants
    \item the solution is a list of functions $x_1(t),\dots,x_n(t)$ which we combine into a vector $\bX(t)$
    \end{itemize}
\end{itemize}
\end{frame}


\begin{frame}{how do you solve the simplest ODEs?}

\begin{itemize}
\item \emph{ODE 1}.  how do you solve for $y(t)$?:
    $$y'=3y$$
    \begin{quote}
    \textbf{answer:}  the solution is an exponential $y(x) = c\, e^{3t}$
    \end{quote}
\item \emph{ODE 2}.  how do you solve for $y(t)$ if $p$ and $q$ are constant?:
    $$y'' + p y' + q y = 0$$
    \begin{quote}
    \textbf{answer:}  try an exponential
        $$y(x) = e^{mt}$$
    and get an auxiliary equation to determine $m$:
\begin{align*}
m^2 e^{mt} + p m e^{mt} + q e^{mt} &= 0 \\
m^2 + p m + q &= 0
\end{align*}
    \end{quote}
\end{itemize}
\end{frame}


\begin{frame}{how do you solve the simplest ODEs?}

\begin{itemize}
\item \emph{ODE 3}.  how do you solve for $\bX(t)$ if $\bA$ is a constant matrix?:
    $$\bX' = \bA \bX$$
    \begin{quote}
    \textbf{answer:}  try an exponential times a vector
        $$\bX(t) = \bK e^{\lambda t}$$
    and get an auxiliary equation to determine $\lambda$:
        $$\text{\textbf{[what equation goes here?]}}$$
    \end{quote}
\item $\lambda$ is an unknown \emph{scalar}, like $m$ before
\item $\bK$ is an unknown \emph{vector}
\end{itemize}
\end{frame}


\begin{frame}{the eigenvalue equation for a system}

    $$\bX' = \bA \bX$$

\bigskip
\begin{itemize}
\item try \quad $\bX(t) = \bK e^{\lambda t}$:
    \begin{align*}
         \text{left side}&  & \bX' &= \bK \lambda e^{\lambda t} \\
        \text{right side}&  & \bA\bX &= \bA \bK e^{\lambda t}
    \end{align*}
\item so system becomes
\begin{align*}
    \bK \lambda e^{\lambda t} &= \bA \bK e^{\lambda t} \\
    \bK \lambda &= \bA \bK
\end{align*}
\item the \emph{eigenvalue equation}:
    $$\boxed{\bA \bK = \lambda \bK}$$
\end{itemize}
\end{frame}


\begin{frame}{the last slide}

\begin{itemize}
\item \alert{the last slide is the main idea}
\item write it out yourself and understand it!
\end{itemize}
\end{frame}


\begin{frame}{meaning of the eigenvalue equation}

\begin{itemize}
\item eigenvalue equation is analogous to auxiliary equation:

\bigskip
\begin{tabular}{l|c|c}
               & scalar & system \\\hline
ODE            & \quad $y''+py'+qy=0$ \quad & $\bX' = \bA\bX \LARGE\strut$ \\ \hline
trial solution & $y(t) = e^{mt}$            & \quad $\bX(t) = \bK e^{\lambda t} \LARGE\strut$ \\ \hline
equation       & $m^2+pm+q=0$               & $\boxed{\bA \bK = \lambda \bK} \LARGE\strut$
\end{tabular}

\bigskip
\item in the eigenvalue equation we are seeking \emph{both}
    \begin{itemize}
    \item \emph{eigenvalues}: the exponential rates $\lambda$
    \item \emph{eigenvectors}: the ``constants'' $\bK$
    \end{itemize}

\medskip
\item ``eigen'' means ``characteristic of'' or ``property of''
    \begin{itemize}
    \item the eigenvalues of a matrix $\bA$ are the most characteristic numbers to associate to $\bA$
    \end{itemize}
\end{itemize}
\end{frame}


\begin{frame}{forms of the eigenvalue equation}

\begin{itemize}
\medskip
\item don't forget the ODE problem we started with: \quad $\bX' = \bA \bX$

\bigskip
\item different forms of the eigenvalue equation:
\begin{align*}
\lambda \bK e^{\lambda t} &= \bA \bK e^{\lambda t} & &\text{substitute $\bX=\bK e^{\lambda t}$} \\
\bA \bK &= \lambda \bK\hspace{10mm} & &\text{eigenvalue equation} \\
(\bA - \lambda \bI)\bK &= 0 & &\text{because $\bI\bK = \bK$} \\
\det(\bA - \lambda \bI) &= 0 & &\text{because we want \emph{nonzero} solutions $\bK$}
\end{align*}

    \begin{itemize}
    \item $\bI$ is the identity matrix
    \item \emph{fact:} a linear equation $\bM \bZ = 0$ has a nonzero solution only if the determinant of $\bM$ is zero: \qquad $\det \bM=0$
    \end{itemize}
\end{itemize}
\end{frame}


\begin{frame}{like \#1 in \S8.2}

\begin{itemize}
\item \emph{example 1.}  find the general solution of the system:
\begin{align*}
\frac{dx}{dt} &= 4x + 3y \hspace{70mm} \\
\frac{dy}{dt} &= x + 2y
\end{align*}
\end{itemize}

\vspace{45mm}
\end{frame}


\begin{frame}{like \#9 in \S8.2}

\begin{itemize}
\item \emph{example 2.}  find the general solution of the system:
$$\bX' = \begin{pmatrix} 1 & 7 & 0 \\ 0 & -2 & 0 \\ 1 & 6 & 4 \end{pmatrix} \bX \hspace{70mm}$$
\end{itemize}

\vspace{50mm}
\end{frame}


\begin{frame}{scalar 2nd-order $\iff$ system of 2 eqns}

\noindent \emph{example 3.}
\begin{itemize}
\item find the general solution of the 2nd-order scalar ODE:
$$y''+3y'+2y=0 \hspace{80mm}$$

\vspace{7mm}
\item convert above to a system:

\vspace{10mm}
\item find the general solution of the system:
$$\bX' = \begin{pmatrix} 0 & 1 \\ -2 & -3 \end{pmatrix} \bX \hspace{70mm}$$

\vspace{12mm}
\end{itemize}
\end{frame}


\begin{frame}[fragile]
\frametitle{help from a machine}

\begin{itemize}
\item \alert{once you know what you want you can get it fast by machine!}
\item find eigenvalues and eigenvectors in \Matlab/\Octave:
\begin{Verbatim}[fontsize=\small]
>> [V,D] = eig(A)    % diagonal of D has eigenvalues
                     % columns of V are eigenvectors
\end{Verbatim}
\item FIXME  need to show how to get "nice" eigenvalues
\end{itemize}
\end{frame}


\begin{frame}{like \#13 in \S8.2}

\begin{itemize}
\item \emph{example 4.}  solve the initial value problem:
$$\bX' = \begin{pmatrix} 4 & 3 \\ 1 & 2 \end{pmatrix} \bX, \qquad \bX(0) = \begin{pmatrix} 5 \\ -1 \end{pmatrix} \hspace{40mm}$$
\end{itemize}

\vspace{70mm}
\end{frame}


\begin{frame}{like \#35 in \S8.2}

\begin{itemize}
\item \emph{example 5.}  find the general solution of the system:
$$\bX' = \begin{pmatrix} 5 & -1 \\ 5 & 1 \end{pmatrix} \bX \hspace{70mm}$$
\end{itemize}

\vspace{70mm}
\end{frame}


\begin{frame}{X}

\begin{itemize}
\item FIXME  on quizzes/exams I can supply eigenstuff
\item FIXME on quizzes/exams I will stick to distinct real case
\end{itemize}
\end{frame}


\begin{frame}[fragile]
\frametitle{the main idea}

\begin{itemize}
\item the ODE system
    $$\alert{\bX' = \bA \bX}$$
has solutions
    $$\alert{\bX(t) = \bK e^{\lambda t}}$$
where $\lambda$ is an eigenvalue of $\bA$ and $\bK$ is an eigenvector of $\bA$,
    $$\alert{\bA\bK=\lambda \bK}$$
and you can get $\lambda$ by-hand via
    $$\det(\bA - \lambda \bI) = 0$$
\item in the modern world you don't find eigenvalues by hand:
\begin{Verbatim}[fontsize=\small]
>> [V,D] = eig(A)    % diagonal of D has eigenvalues
                     % columns of V are eigenvectors
\end{Verbatim}
\end{itemize}
\end{frame}


\begin{frame}{expectations}

\begin{itemize}
\item just watching this video is \emph{not} enough!
     \begin{itemize}
     \item see ``found online'' videos and stuff at

     \centerline{\href{https://bueler.github.io/math302/week14.html}{\tt \color{cyan} bueler.github.io/math302/week14.html}}
     \item \emph{read} \S8.2
         \begin{itemize}
         \item you \alert{are responsible} for the ``distinct real eigenvalues'' and the ``complex eigenvalues'' cases
         \item you are not responsible for the ``repeated eigenvalues'' case
         \end{itemize}
     \item \emph{do} the WebAssign exercises for section 8.2
     \end{itemize}
\end{itemize}
\end{frame}

\end{document}

