\documentclass[dvipsnames]{beamer}

\usepackage[utf8]{inputenc}
\usepackage{fancybox,fancyvrb}
\usepackage{environ}
\usepackage{tikz}

\beamertemplatenavigationsymbolsempty
\setbeamertemplate{footline}[frame number]
\usetheme{Pittsburgh}

\newcommand\enumnum[1]{{\renewcommand{\insertenumlabel}{#1}%
      \usebeamertemplate{enumerate item} \,}}

\newcommand{\grad}{\nabla}
\newcommand{\ih}{\boldsymbol{\hat{\textbf{\i}}}}
\newcommand{\jh}{\boldsymbol{\hat{\textbf{\j}}}}
\newcommand{\vF}{\boldsymbol{\vec{\textbf{F}}}}
\newcommand{\Matlab}{\textsc{Matlab}}


\title{4.2 Reduction of order}

\subtitle{a lesson for MATH F302 Differential Equations}

\author{Ed Bueler, Dept.~of Mathematics and Statistics, UAF}

\date{\tiny \today}


\begin{document}
\setbeamertemplate{itemize item}{$\bullet$}
\setbeamertemplate{itemize subitem}{$\circ$}


\begin{frame}
\titlepage

\centerline{\tiny for textbook: \, D. Zill, \emph{A First Course in Differential Equations with Modeling Applications}, 11th ed.}
%\color{green!40!blue}
\end{frame}


\begin{frame}{a 2nd-order example}

\begin{itemize}
\item start with an example for which the methods of section 4.3 \emph{do not} work, but where we \emph{can} see the solution anyway
\item \emph{example 1}: find the general solution
    $$x y'' + y' = 0$$
\end{itemize}

% solve by:
%  x u' + u = 0
%  du/u = -dx/x
%  ln|u| = - ln|x| + c
%  y' = u = A/x
%  y = A ln|x| + c_1
% fund set:  {1,ln|x|}
\vspace{50mm}
\end{frame}


\begin{frame}{the \S4.3 rule needing explanation}

\begin{itemize}
\item the next example is one we \emph{do} know how to solve, but I will use a rule which requires justification
\item \emph{example 2}:  find the general solution using \S4.3 rules
    $$y'' - 6 y' + 9 y = 0$$
\end{itemize}

\vspace{50mm}
\end{frame}


\begin{frame}{reducing the order 1: first illustration}

\begin{itemize}
\item \emph{reduction of order} is a technique:
    \begin{itemize}
    \item substitute $y(x) = u(x) y_1(x)$
    \item derive a DE for $u$ which has no zeroth-order term
    \item solve a first-order equation for $w=u'$
    \end{itemize}
\item key understanding: the \alert{purpose} is to \alert{find another linearly-independent solution} given you have $y_1(x)$
\item \emph{example 3}:  we know $y_1(x)=e^{3x}$ is a solution; find another
    $$y'' - 6 y' + 9 y = 0 \hspace{70mm}$$
\end{itemize}

\vspace{50mm}
\end{frame}


\begin{frame}{reducing the order 2: general case}

suppose $y_1(x)$ is a solution to 2nd-order homogeneous DE
    $$y'' + P(x) y' +  Q(x) y \stackrel{\ast}{=} 0,$$
and we seek another solution of the form $y(x) = u(x) y_1(x)$:
\begin{itemize}
\item compute $y' = u' y_1 + u y_1'$ and $y'' = u'' y_1 + 2 u' y_1' + u y_1''$
    $$\hspace{-70mm} \text{\footnotesize  \emph{check it}: } \quad y'' = $$
\item substitute into $\ast$:
    $$(u'' y_1 + 2 u' y_1' + u y_1'') + P (u' y_1 + u y_1') + Q u y_1 = 0$$
\item group by derivatives on $u$:
    $$y_1 u'' + (2 y_1' + P y_1) u' + ({\color{Green} y_1'' + P y_1' + Q y_1}) u = 0$$
\item term in {\color{Green} green} is zero (\emph{why?}) so $u$ solves
    $$y_1 u'' + (2 y_1' + P y_1) u' = 0$$
\end{itemize}
\end{frame}


\begin{frame}{reducing the order 3: a first-order equation}

\begin{itemize}
\item we are seeking a solution of the form $y = u y_1$, and $u$ solves
    $$y_1 u'' + (2 y_1' + P y_1) u' = 0$$
\item there is \alert{no zeroth-order term} so we can solve it
\item the equation is first-order and separable for $w=u'$:
\begin{gather*}
y_1 w' + (2 y_1' + P y_1) w = 0 \\
\frac{dw}{dx} = -\frac{(2 y_1' + P y_1) w}{y_1} \\
\frac{dw}{w} = - \left(2 \frac{y_1'}{y_1} + P\right)\, dx \\
\int \frac{dw}{w} = - 2 \int \frac{y_1'(x)}{y_1(x)}\,dx - \int P(x)\,dx
\end{gather*}
\end{itemize}
\end{frame}


\begin{frame}{reducing the order 4: the second solution}

\begin{itemize}
\item continuing:
\begin{gather*}
\int \frac{dw}{w} = - 2 \int \frac{y_1'(x)}{y_1(x)}\,dx - \int P(x)\,dx \\
\ln|w(x)| = - 2 \ln|y_1(x)| - \int P(x)\,dx + C \\
w(x) = c_1 \frac{e^{-\int P(x)\,dx}}{y_1(x)^2}
\end{gather*}
\item recall $u'=w$; thus integrating again gives
    $$u(x) = c_1 \int \frac{e^{-\int P(x)\,dx}}{y_1(x)^2}\,dx + c_2$$
\item the second solution is \alert{the new part of $y=uy_1$}:
    $$y_2(x) = y_1(x) \int \frac{e^{-\int P(x)\,dx}}{y_1(x)^2}\,dx$$
\end{itemize}
\end{frame}


\begin{frame}{example 4}

\begin{itemize}
\item in this example we tell a complete story: we \emph{guess} a first solution and then derive a second one by reduction of order
\item \emph{example 4}: find the general solution (for $x>0$)
    $$x^2 y'' + 5 x y' + 4 y = 0$$
\end{itemize}

\vspace{60mm}
\end{frame}


\begin{frame}{example 4, finished}

\vspace{60mm}

\hfill $y(x) = c_1 x^{-2} + c_2 x^{-2} \ln x$
\end{frame}


\begin{frame}{differential equations \emph{must be} hard}

\begin{itemize}
\item differential equations \emph{must be} hard, and sometimes impossible

\bigskip
\begin{proof} The simple (separable) DE
    $$\frac{dy}{dx} = f(x)$$
has solution
    $$y(x) = \int f(x)\,dx.$$
Integration is hard, for example, try $f(x)=e^{-x^2}$ in the above.

But the set of possible problems in DEs is \emph{bigger} than the set of hard integrals, so DEs must be hard.
\end{proof}
\end{itemize}
\end{frame}


\begin{frame}{exercise 17 in \S4.2}

\begin{itemize}
\item you can use reduction order on nonhomogeneous problems, if you are careful in understanding where you are going
\item \emph{exercise 17}:  given that $y_1=e^{-2x}$ solves the homogeneous equation $y''-4y=0$, find the general solution of the nonhomogeneous equation
    $$y'' - 4y = 2$$
\end{itemize}

\vspace{50mm}
\end{frame}


\begin{frame}{exercise 17, finished}

\vspace{60mm}

\hfill $y(x) = c_1 e^{-2x} + c_2 e^{2x} - \frac{1}{2}$
\end{frame}


\begin{frame}{expectations}

\begin{itemize}
\item just watching this video is \emph{not} enough!
     \begin{itemize}
     \item see ``found online'' videos at

     \centerline{\href{https://bueler.github.io/math302/week7.html}{\tt \color{cyan} bueler.github.io/math302/week7.html}}
     \item \emph{read} \S 4.2 in the textbook
     \item \emph{do} the WebAssign exercises for \S 4.2
     \end{itemize}
\item note example 4 is a \emph{Cauchy-Euler} type of differential equation
     \begin{itemize}
     \item covered in \S 4.7 \dots which we will skip
     \end{itemize}
\item to do reduction of order on a quiz or exam you have a \alert{choice}
\item do you
     \begin{enumerate}
     \item memorize \fbox{$y_2(x) = y_1(x) \int \frac{e^{-\int P(x)\,dx}}{y_1(x)^2}\,dx$}?

\medskip
     \item or substitute $y(x) = u(x) y_1(x)$ and see how it comes out?
     \end{enumerate}
\end{itemize}
\end{frame}

\end{document}

