\documentclass[urlcolor=blue,dvipsnames]{beamer}

\usepackage[utf8]{inputenc}
\usepackage{fancybox,fancyvrb}
\usepackage{environ,xspace,empheq}

\usepackage{tikz}
\usetikzlibrary{arrows.meta,decorations.markings,decorations.pathreplacing,fadings,positioning}

\hypersetup{colorlinks,linkcolor=,urlcolor=cyan}

\beamertemplatenavigationsymbolsempty
\setbeamertemplate{footline}[frame number]
\usetheme{Pittsburgh}

%\makeatletter
%\newcommand{\tinytiny}{\@setfontsize{\tinytiny}{4pt}{4pt}}
%\makeatother

\newcommand\enumnum[1]{{\renewcommand{\insertenumlabel}{#1}%
      \usebeamertemplate{enumerate item} \,}}

\newcommand{\grad}{\nabla}
\newcommand{\ih}{\boldsymbol{\hat{\textbf{\i}}}}
\newcommand{\jh}{\boldsymbol{\hat{\textbf{\j}}}}
\newcommand{\vF}{\boldsymbol{\vec{\textbf{F}}}}
\newcommand{\Matlab}{\textsc{Matlab}\xspace}
\newcommand{\Octave}{\textsc{Octave}\xspace}


\title{3.3 Systems of first-order ODEs \\ are models of everything}

\subtitle{a lesson for MATH F302 Differential Equations}

\author{Ed Bueler, Dept.~of Mathematics and Statistics, UAF}

\date{\tiny \today}


\begin{document}
\setbeamertemplate{itemize item}{$\bullet$}
\setbeamertemplate{itemize subitem}{$\circ$}
\renewcommand{\thefootnote}{{\color{green} \arabic{footnote}}}

\begin{frame}
\titlepage

\centerline{\tiny for textbook: \, D. Zill, \emph{A First Course in Differential Equations with Modeling Applications}, 11th ed.}
%\color{green!40!blue}
\end{frame}

\newcommand{\LL}[1]{\mathcal{L}\left\{#1\right\}}
\newcommand{\LLi}[1]{\mathcal{L}^{-1}\left\{#1\right\}}


\begin{frame}{first-order systems}

\begin{itemize}
\item a \emph{system of two first-order equations}:
\begin{align*}
\frac{dx}{dt} &= f(t,x,y) \\
\frac{dy}{dt} &= g(t,x,y)
\end{align*}

\vspace{-2mm}
    \begin{itemize}
    \item the solution is the pair of functions $x(t),y(t)$
    \item we say system is \emph{coupled} if $f$ depends on $y$ or $g$ depends on $x$
    \end{itemize}
\item we have already started to see such systems
    \begin{itemize}
    \item see \href{https://bueler.github.io/math302/assets/slides/5-3.pdf}{slides for \S5.3/4.10} (and \href{https://expl.ai/VSJTFRC}{video})
    \end{itemize}
\item $f$ and $g$ can be any formulas; here's a silly example:

\vspace{-3mm}
\scriptsize
\begin{align*}
\frac{dx}{dt} &= t^5 + x^6 + y^7 \\
\frac{dy}{dt} &= \arctan(y + \sin(x + \cos(t)))
\end{align*}
\normalsize
\end{itemize}
\end{frame}


\begin{frame}{easily-solvable example}

\begin{itemize}
\item \emph{example 1.}  find the general solution to
\begin{align*}
\frac{dx}{dt} &= - 2 x\\
\frac{dy}{dt} &= x - y
\end{align*}
\end{itemize}

\noindent \emph{solution.}

\vspace{45mm}
\hfill \tiny \emph{special properties: (1) one-way coupled and (2) linear and (3) homogeneous}
\end{frame}


\begin{frame}{system can be any size}

\begin{itemize}
\item systematic notation for two equations:
\footnotesize
\begin{align*}
\frac{dx_1}{dt} &= g_1(t,x_1,x_2) \\
\frac{dx_2}{dt} &= g_2(t,x_1,x_2)
\end{align*}
\normalsize
\item system of $n$ equations:
\alert{
\begin{align*}
\frac{dx_1}{dt} &= g_1(t,x_1,x_2,\dots,x_n) \\
\frac{dx_2}{dt} &= g_2(t,x_1,x_2,\dots,x_n) \\
                &\qquad \vdots \\
\frac{dx_n}{dt} &= g_n(t,x_1,x_2,\dots,x_n)
\end{align*}
}
    \vspace{-2mm}
    \begin{itemize}
    \item solution is set of $n$ functions $x_1(t),x_2(t),\dots,x_n(t)$
    \item in \href{http://fluidityproject.github.io/}{practical, modern fluids simulations $n\ge 10^6$}
    \item such systems are also \href{https://pybullet.org/wordpress/}{the physics in video games}
    \end{itemize}
\end{itemize}
\end{frame}


\begin{frame}{\emph{most} math models are systems of DEs}

\begin{itemize}
\item systems of ODEs are \alert{common}
\item \dots because most real things involve

    \begin{itemize}
    \item \alert{many parts} \hfill $x_1,\dots,x_n$

\medskip
    \item \alert{changing in time} \hfill $\frac{dx_i}{dt}=g_i(\dots)$

\medskip
    \item \alert{interacting with each other} \hfill $g_i$ depends on $x_j$
    \end{itemize}

\bigskip
\item everything is modeled this way:
    \begin{enumerate}
    \item the galaxy
    \item your body
    \item this double-pendulum fidget spinner

which can be yours for only \$98
    \end{enumerate}
\end{itemize}

\vspace{-15mm}
\hfill \includegraphics[width=0.35\textwidth]{figs/double-pendulum-fidget}
\end{frame}


\begin{frame}{radioactive decay series}

\begin{itemize}
\item read about it in \S3.3

    \begin{itemize}
    \item often one-way coupled
    \item simple cases can be as easy/solvable as example 1
    \end{itemize}
\end{itemize}

\bigskip
\begin{center}
\includegraphics[width=0.45\textwidth]{figs/thorium232-decay}
\end{center}
\end{frame}


\begin{frame}{connected tanks}

\begin{itemize}
\item \emph{example 2.}  Three 100 gallon tanks have brine solutions and are connected as shown.  The tanks are always full.  $x_1(t),x_2(t),x_3(t)$ pounds of salt are in each tank, respectively.
    \begin{itemize}
    \item[(a)] What equations must hold for the flow rates $a,b,c,d,e,f$?
    \item[(b)] Suppose $a=2,d=4,e=5$ in gal/min.  Compute $b,c,f$.
    \item[(c)] Write a first-order ODE system for $x_1(t),x_2(t),x_3(t)$.
    \end{itemize}
\end{itemize}

\begin{center}
\includegraphics[width=0.7\textwidth]{figs/three-tanks}
\end{center}
\end{frame}


\begin{frame}{connected tanks, cont.}

\noindent \emph{solution.}

\vspace{45mm}

\hfill \begin{minipage}{0.55\textwidth}
\includegraphics[width=\textwidth]{figs/three-tanks}

\scriptsize
\hfill given $a=2,d=4,e=5$
\end{minipage}
\end{frame}


\begin{frame}{higher order equations become systems}

\begin{itemize}
\item \alert{any} individual (a.k.a.~\emph{scalar}) ODE can be turned into a first-order system
\item for example, a damped nonlinear pendulum for $\theta(t)$:
    $$m \ell \theta'' + \beta \theta' + mg \sin\theta = 0$$
becomes this system:
\begin{align*}
x_1' &= x_2 \\
x_2' &= -\left(\frac{\beta}{m\ell}\right) x_2 - \left(\frac{g}{\ell}\right) \sin(x_1)
\end{align*}
    \begin{itemize}
    \item \alert{I did almost nothing here!}
    \item just name $\theta$ as $x_1$ and name $\theta'$ as $x_2$
    \item \small solve for the derivative because that is the standard form
    \end{itemize}
\end{itemize}
\end{frame}


\begin{frame}{a 4th order ODE as a system}

\begin{itemize}
\item \emph{example 3.}  write the following fourth-order ODE as a first-order system:
    $$y^{(4)} - 4 y''' + 7 y'' + 10 y' - y = \sin(3t)$$
\end{itemize}

\noindent \emph{solution.}

\vspace{45mm}
\end{frame}


\begin{frame}{hares and lynx}

\begin{itemize}
\item consider this ``Lotka-Volterra'' model
\begin{align*}
\frac{dx}{dt} &= 0.7 x - 1.3 xy \\
\frac{dy}{dt} &= xy - y
\end{align*}

\vspace{-2mm}
    \begin{itemize}
    \item $x(t)$ is the number of prey
    \item $y(t)$ is the number of predators
    \item constants merely representative \dots but signs important
    \end{itemize}
\end{itemize}

\bigskip
\hfill \includegraphics[width=0.6\textwidth]{figs/hares-lynx}
\end{frame}


\begin{frame}[fragile]
\frametitle{like \S3.3 \#11}

\begin{itemize}
\item \emph{example 4.}  solve numerically for $0\le t\le 60$:
\begin{align*}
\frac{dx}{dt} &= 0.7 x - 1.3 xy & x(0)=1\\
\frac{dy}{dt} &= xy - y  & y(0)=1
\end{align*}
\end{itemize}

\vspace{-2mm}
\noindent \emph{solution.}
\begin{Verbatim}[fontsize=\small]
>> f = @(t,z) [0.7*z(1)-1.3*z(1)*z(2); z(1)*z(2)-z(2)];
>> [tt,zz] = ode45(f,0:.1:60,[1;1]);
>> plot(tt,zz),  xlabel t
>> legend('prey','predators')
\end{Verbatim}

\bigskip
using:

$z_1=x$

$z_2=y$

\vspace{-22mm}
\hfill \includegraphics[width=0.45\textwidth]{figs/lotka-time}
\end{frame}


\begin{frame}[fragile]
\frametitle{\emph{phase plane}: a different view}

\begin{itemize}
\item a different view is to plot $x=z_1$ versus $y=z_2$

\begin{Verbatim}[fontsize=\small]
>> figure(2)
>> plot(zz(:,1),zz(:,2),'k')   % curve in black
>> xlabel('x(t)  prey'),  ylabel('y(t)  predators')
\end{Verbatim}

\bigskip
\mbox{\includegraphics[width=0.44\textwidth]{figs/lotka-time}\quad \includegraphics[width=0.46\textwidth]{figs/lotka-phase}}
\item we will get back to this view in Chapter 8
\end{itemize}
\end{frame}


\begin{frame}{beyond: PDEs and pattern generation}

\small
\begin{itemize}
\item consider this system of ODEs ($\phi,\kappa$ constants):
\scriptsize
\begin{align*}
\frac{du}{dt} &= -uv^2+\phi(1-u) \\
\frac{dv}{dt} &= uv^2-(\phi+\kappa)v
\end{align*}
\small
\item it is a model\footnote{\tiny J.~E.~Pearson (1993). \emph{Complex patterns in a simple system}, Science, 261, 189--192} of a reaction between two chemicals $u$ and $v$, similar to Lotka-Volterra (predator-prey) system
\item add diffusion:

\vspace{-10mm}
\scriptsize
\begin{align*}
\frac{\partial u}{\partial t} &= D_u \left(\frac{\partial^2 u}{\partial x^2} + \frac{\partial^2 u}{\partial y^2}\right) -uv^2+\phi(1-u) \\
\frac{\partial v}{\partial t} &= D_v \left(\frac{\partial^2 v}{\partial x^2} + \frac{\partial^2 v}{\partial y^2}\right) + uv^2-(\phi+\kappa)v
\end{align*}
\small
% in biology the way you get many patterns (e.g.~spots on a cheetah) is to have a competition between two chemicals at every location; each location has a first-order ODE system going, and they are linked together by diffusion
\end{itemize}

\mbox{\includegraphics[width=0.23\textwidth]{figs/pattern0}\quad\includegraphics[width=0.23\textwidth]{figs/pattern2}\quad\includegraphics[width=0.23\textwidth]{figs/pattern4}\quad\includegraphics[width=0.23\textwidth]{figs/pattern6}}

\bigskip
\end{frame}



\begin{frame}{ODE systems from circuits}

\begin{itemize}
\item the voltage $v(t)$ and current $i(t)$ in an electrical circuits changes in time
\item each element in a circuit (network) has a little model:

\begin{tabular}{lcc}
resistor & $v = iR$ & \includegraphics[height=10mm,trim=0 3mm 0 -3mm]{figs/circuit-element-R} \\
inductor & $v = L \frac{di}{dt}$ & \includegraphics[height=13mm,trim=0 5mm 0 -5mm]{figs/circuit-element-L} \\
resistor & $v = \frac{q}{C}$ & \includegraphics[height=12mm,trim=0 5mm 0 -5mm]{figs/circuit-element-C}
\end{tabular}

\bigskip \bigskip
\item \emph{Kirchoff's laws} allow you to assemble systems of ODEs from these elements
\item building such models is the heart of electical engineering
\end{itemize}
\end{frame}


\begin{frame}{a linear ODE system for an RLC circuit}

\begin{itemize}
\item I'll do an example, but you are \emph{not} responsible for doing this!
\item \emph{example 5.} construct a system of first-order ODEs for the currents $i_1,i_2,i_3$ in this electical circuit

\bigskip
\includegraphics[width=0.3\textwidth]{figs/new-rlc-circuit}
\end{itemize}

\vspace{30mm}
\end{frame}


\begin{frame}{expectations}

\begin{itemize}
\item just watching this video is \emph{not} enough!
     \begin{itemize}
     \item see ``found online'' videos and stuff at

     \centerline{\href{https://bueler.github.io/math302/week13.html}{\tt \color{cyan} bueler.github.io/math302/week13.html}}
     \item \emph{read} \S3.3
     \item \emph{do} the WebAssign exercises for section 3.3
         \begin{itemize}
         \item what are you actually responsible for?  \alert{be able to do computations like in examples 1--4}
         \item \dots \emph{and} be able to do radioactive decay series examples

             \hspace{4mm} ${\color{blue} \circ}$ read the section!
         \item you are \emph{not} responsible for electrical circuits as in example 5
         \end{itemize}
     \end{itemize}
\end{itemize}
\end{frame}

\end{document}

