\documentclass[urlcolor=blue,dvipsnames]{beamer}

\usepackage[utf8]{inputenc}
\usepackage{fancybox,fancyvrb}
\usepackage{environ,xspace,empheq}

\usepackage{tikz}
\usetikzlibrary{arrows.meta,decorations.markings,decorations.pathreplacing,fadings,positioning}

\hypersetup{colorlinks,linkcolor=,urlcolor=cyan}

\beamertemplatenavigationsymbolsempty
\setbeamertemplate{footline}[frame number]
\usetheme{Pittsburgh}

%\makeatletter
%\newcommand{\tinytiny}{\@setfontsize{\tinytiny}{4pt}{4pt}}
%\makeatother

\newcommand\enumnum[1]{{\renewcommand{\insertenumlabel}{#1}%
      \usebeamertemplate{enumerate item} \,}}

\newcommand{\bA}{\mathbf{A}}
\newcommand{\bC}{\mathbf{C}}
\newcommand{\bF}{\mathbf{F}}
\newcommand{\bI}{\mathbf{I}}
\newcommand{\bK}{\mathbf{K}}
\newcommand{\bM}{\mathbf{M}}
\newcommand{\bX}{\mathbf{X}}
\newcommand{\bZ}{\mathbf{Z}}


\newcommand{\grad}{\nabla}
\newcommand{\ih}{\boldsymbol{\hat{\textbf{\i}}}}
\newcommand{\jh}{\boldsymbol{\hat{\textbf{\j}}}}
\newcommand{\vF}{\boldsymbol{\vec{\textbf{F}}}}
\newcommand{\Matlab}{\textsc{Matlab}\xspace}
\newcommand{\Octave}{\textsc{Octave}\xspace}


\title{8.4 The matrix exponential solves systems}

\subtitle{a lesson for MATH F302 Differential Equations}

\author{Ed Bueler, Dept.~of Mathematics and Statistics, UAF}

\date{\tiny \today}


\begin{document}
\setbeamertemplate{itemize item}{$\bullet$}
\setbeamertemplate{itemize subitem}{$\circ$}
\renewcommand{\thefootnote}{{\color{green} \arabic{footnote}}}

\begin{frame}
\titlepage

\centerline{\tiny for textbook: \, D. Zill, \emph{A First Course in Differential Equations with Modeling Applications}, 11th ed.}
%\color{green!40!blue}
\end{frame}


\begin{frame}{solving the simplest ODEs}

\begin{itemize}
\item \emph{simplest scalar ODE:}
    $$y'=ay \qquad \text{has solution} \qquad y(t) = c e^{at}$$

\bigskip
\item \emph{simplest system of ODEs:}
    $$\bX'=\bA \bX \qquad \text{has solution} \qquad \alert{\bX(t) = e^{\bA t}\bC}$$
    \begin{itemize}
    \item the last formula is \alert{new in \S8.4}
    \end{itemize}
\end{itemize}
\end{frame}


\begin{frame}{what does $e^{\bA t}$ mean?}

\begin{itemize}
\item what does $e^{\bA t}$ mean?
   \begin{itemize}
   \item what does $e^{at}$ mean?
       \begin{itemize}
       \item[$\ast$] what does $e^x$ mean?  \alert{$\longleftarrow$ we know this one!}
\begin{align*}
e^x &= 1 + x + \frac{x^2}{2} + \frac{x^3}{3!} + \frac{x^4}{4!} + \dots \\
    &= \sum_{k=0}^\infty \frac{x^k}{k!}
\end{align*}
       \end{itemize}
   \end{itemize}

\medskip
\item \emph{definition.} if $\bA$ is a square matrix and $t$ is any number then
\begin{align*}
e^{\bA t} &= \bI + \bA t + \bA^2 \frac{t^2}{2} + \bA^3 \frac{t^3}{3!} + \bA^4 \frac{t^4}{4!} + \dots \\
    &= \sum_{k=0}^\infty \bA^k \frac{t^k}{k!}
\end{align*}
   \begin{itemize}
   \item note $\bA^0 = \bI$ makes sense if we believe $x^0=1$
   \item also recall $0!=1$
   \end{itemize}
\end{itemize}
\end{frame}


\begin{frame}{like exercise \#1 in \S8.4}

\begin{itemize}
\item $e^{\bA t} = \bI + \bA t + \bA^2 \frac{t^2}{2} + \bA^3 \frac{t^3}{3!} + \bA^4 \frac{t^4}{4!} + \dots$

\medskip
\item \emph{example 1}.  use the above series to compute $e^{\bA t}$ and $e^{-\bA t}$, in simplified form, if
    $$\bA = \begin{pmatrix} 3 & 0 \\ 0 & -1 \end{pmatrix}     \hspace{80mm}$$
\end{itemize}

\vspace{40mm}
\end{frame}


\begin{frame}{like exercise \#4 in \S8.4}

\begin{itemize}
\item \emph{example 2}.  use the series definition to compute $e^{\bA t}$ and $e^{-\bA t}$, in simplified form, if
    $$\bA = \begin{pmatrix} 0 & 0 & 0 \\ 2 & 0 & 0 \\ -1 & 3 & 0 \end{pmatrix}     \hspace{80mm}$$
\end{itemize}

\vspace{40mm}
\end{frame}


\begin{frame}{like exercise \#3 in \S8.4}

\begin{itemize}
\item \emph{example 3}.  use the series definition to compute $e^{\bA t}$, in simplified form, if
    $$\bA = \begin{pmatrix} -3 & -3 & -3 \\ 1 & 1 & 1 \\ 2 & 2 & 2 \end{pmatrix}     \hspace{80mm}$$
\end{itemize}

\vspace{40mm}
\end{frame}


\begin{frame}{last two were special}

\begin{itemize}
\item \emph{example 2} and \emph{example 3} had unusual matrices with the property that some power was the zero matrix:
    $$\bA^k = \mathbf{0}$$
\item this is not typical; for most $\bA$:
    \begin{itemize}
    \item $e^{\bA t}$ has infinitely-many nonzero terms
    \item the pattern is hard to see
    \end{itemize}

\bigskip
\item but we can show the matrix exponential $e^{\bA t}$ gets it done!
    \begin{itemize}
    \item next slide
    \end{itemize}
\end{itemize}
\end{frame}


\begin{frame}{derivative of $e^{\bA t}$}

\begin{itemize}
\item \emph{fact 1.}
    $$\alert{\frac{d}{dt}\left(e^{\bA t}\right) = \bA e^{\bA t}}$$
\small
\begin{proof}

\vspace{-5mm}
\footnotesize
\begin{align*}
\frac{d}{dt}\left(e^{\bA t}\right) &= \frac{d}{dt}\left(\bI + \bA t + \bA^2 \frac{t^2}{2} + \bA^3 \frac{t^3}{3!} + \bA^4 \frac{t^4}{4!} + \dots\right) \\
    &= \mathbf{0} + \bA + \bA^2 \frac{2t}{2} + \bA^3 \frac{3 t^2}{3!} + \bA^4 \frac{4 t^3}{4!} + \dots \\
    &= \bA \left(\bI + \bA t + \bA^2 \frac{t^2}{2} + \bA^3 \frac{t^3}{3!} + \dots\right) = \bA e^{\bA t}
\end{align*}
\end{proof}
\normalsize

\vspace{-5mm}
\item \emph{fact 2.}  if $\bX(t) = e^{\bA t}\bC$ then $\bX'=\bA \bX$
\small
\begin{proof}
\phantom{foo}
\end{proof}
\normalsize
\item \emph{fact 3.}  if $\bX(t) = e^{\bA t}\bC$ then $\bX(0) = \bC$
\end{itemize}
\end{frame}


\begin{frame}{the matrix exponential solves systems}

\noindent in summary:
\begin{enumerate}
\item for the ODE
    $$\bX' = \bA \bX$$
the general solution is
    $$\bX(t) = e^{\bA t} \bC$$
where $\bC=\left<c_1,c_2,\dots,\right>$ is a vector of unknown constants
\item for the ODE IVP
    $$\bX' = \bA \bX, \qquad \bX(0)=\bC$$
the solution is
    $$\bX(t) = e^{\bA t} \bC$$
\end{enumerate}
\end{frame}


\begin{frame}[fragile]
\frametitle{use a computer \dots}

\begin{itemize}
\item for the ODE IVP
$$\bX' = \bA \bX, \qquad \bX(0)=\bC$$
suppose you want the solution at time $T$:
    $$\bX(T) = e^{\bA T} \bC$$

\medskip
\item with \Matlab/\Octave:
\begin{Verbatim}[fontsize=\small]
>> A = [...];            % enter square matrix A
>> C = [...];            % enter column vector C
>> expm(A*T) * C         % bam! done!
\end{Verbatim}

\medskip
\item \texttt{expm()} computes the matrix exponential
    \begin{itemize}
    \item be careful \dots \texttt{exp()} is \emph{not} what you want
    \end{itemize}
\end{itemize}
\end{frame}


\begin{frame}[fragile]
\frametitle{\dots for fast numbers}

\begin{itemize}
\item \emph{example 4}.  solve the initial value problem for $x(2),y(2),z(2)$:
\begin{align*}
x' &= 2 x - 5 y + z & x(0)&=-2 \\
y' &= - x + y + 3 z & y(0)&=0 \\
z' &= x - 2 y - z   & z(0)&=3
\end{align*}
\end{itemize}

\noindent \emph{solution.}
\begin{Verbatim}[fontsize=\small]
>> A = [2 -5 1; -1 1 3; 1 -2 -1];
>> C = [-2; 0; 3];
>> expm(A*2) * C
ans =
     -1227.9
      68.564
     -381.69
\end{Verbatim}

\medskip
so: \quad $x(2)=-1227.9$, $y(2)=68.564$, $z(2)=-381.69$
\end{frame}


\begin{frame}[fragile]
\frametitle{can you check it?}

\begin{itemize}
\item what tool would help us quickly check previous slide result?
    \begin{itemize}
    \item consider all the tools in the whole course \dots
    \end{itemize}
\item \emph{answer.}  the most general tool is \alert{numerical approximation}

\bigskip
\item \emph{example 4, cont.}  check the solution on the previous slide
\end{itemize}

\noindent \emph{solution.}
\begin{Verbatim}[fontsize=\small]
>> f = @(t,U) [2*U(1)-5*U(2)+U(3); ...
               -U(1)+U(2)+3*U(3); ...
               U(1)-2*U(2)-U(3)];
>> [tt,UU] = ode45(f,[0,2],[-2;0;3]);
>> UU(end,:)
ans =
     -1227.9      68.565      -381.7
\end{Verbatim}
\end{frame}


\begin{frame}{like exercise \#8 in \S8.4}

\begin{itemize}
\item as long as we can compute $e^{\bA t}$ by hand, we can solve an ODE by hand using the matrix exponential
\item \emph{example 5}.  use $\bX(t)=e^{\bA t}\bC$ to find the general solution of the given system
    $$\bX' = \begin{pmatrix} 0 & 0 & 0 \\ 2 & 0 & 0 \\ -1 & 3 & 0 \end{pmatrix} \bX    \hspace{70mm}$$
\end{itemize}

\vspace{40mm}
\end{frame}


\begin{frame}{the two views in \S8.2 and \S8.4}

\begin{itemize}
\item[\S 8.2:] the ODE system $\bX' = \bA \bX$ has solutions
    $$\bX_j(t) = \bK_j e^{\lambda_j t}$$
where $\lambda_j$ is an eigenvalue of $\bA$ and $\bK_j$ is a corresponding eigenvector,
    $$\bA\bK_j=\lambda_j \bK_j$$
and the general solution is
\begin{align*}
\bX(t) &= c_1 \bX_1(t) + \dots + c_n \bX_j(t) \\
       &= c_1 \bK_1 e^{\lambda_1 t} + \dots + c_n \bK_n e^{\lambda_n t}
\end{align*}
\item[\S 8.4:] the ODE system $\bX' = \bA \bX$ has general solution
    $$\bX(t) = e^{\bA t}\bC$$
\end{itemize}
\end{frame}


\begin{frame}{expectations}

\begin{itemize}
\item just watching this video is \emph{not} enough!
     \begin{itemize}
     \item see ``found online'' videos and stuff at

     \centerline{\href{https://bueler.github.io/math302/week14.html}{\tt \color{cyan} bueler.github.io/math302/week14.html}}
     \item \emph{read} \S8.4
     \item \emph{do} the WebAssign exercises for section 8.4
         \begin{itemize}
         \item these exercises are about cases where the matrix exponential reduces to a simple pattern or just a few nonzero terms
         \item like examples 1--3 and 5 above
         \end{itemize}
     \end{itemize}

\bigskip
\item I hope you have learned something from this course!
     \begin{itemize}
     \item perhaps even found it interesting?
     \item note that
         \begin{itemize}
         \item MATH 302 discrete mathematics
         \item MATH 314 linear algebra
         \item MATH 310 numerical analysis
         \end{itemize}
     are courses of more-or-less comparable level to MATH 302 (differential equations)
     \end{itemize}
\end{itemize}
\end{frame}

\end{document}

