\documentclass{beamer}

\usepackage[utf8]{inputenc}
\usepackage{fancybox,fancyvrb}
\usepackage{environ}
\usepackage{tikz}

\beamertemplatenavigationsymbolsempty
\setbeamertemplate{footline}[frame number]
\usetheme{Pittsburgh}

\newcommand\enumnum[1]{{\renewcommand{\insertenumlabel}{#1}%
      \usebeamertemplate{enumerate item} \,}}

\newcommand{\grad}{\nabla}
\newcommand{\ih}{\boldsymbol{\hat{\textbf{\i}}}}
\newcommand{\jh}{\boldsymbol{\hat{\textbf{\j}}}}
\newcommand{\vF}{\boldsymbol{\vec{\textbf{F}}}}


\title{4.1 Higher-order linear equations: \\ first examples and preliminaries}

\subtitle{a lesson for MATH F302 Differential Equations}

\author{Ed Bueler, Dept.~of Mathematics and Statistics, UAF}

\date{\tiny \today}


\begin{document}
\setbeamertemplate{itemize item}{$\bullet$}
\setbeamertemplate{itemize subitem}{$\circ$}


\begin{frame}
\titlepage

\centerline{\tiny for textbook: \, D. Zill, \emph{A First Course in Differential Equations with Modeling Applications}, 11th ed.}
%\color{green!40!blue}
\end{frame}


\begin{frame}{outline}

plan for these slides
\begin{itemize}
\item a bit of review of first-order linear equations (\S2.3)
\item a first look at how to solve constant-coefficient, second-order linear equations (from \S4.3)
\item a whole bunch of new language for higher-order linear equations
    \begin{itemize}
    \item basically, \S4.1 is a lot of new words
    \end{itemize}
\end{itemize}
\end{frame}


\begin{frame}{first-order linear DEs: a review}

\begin{itemize}
\item recall first-order linear DEs:
    $$a_1(x) y' + a_0(x) y = g(x)$$
\item one may divide by the leading coefficient:
    $$y' + P(x) y = f(x)$$

    \begin{itemize}
    \item \alert{this requires leading coefficient $a_1(x)$ to \emph{not} to be zero} on the interval where we are solving
    \end{itemize}
\item special case 1 (\alert{easiest to solve}): constant-coefficient and homogeneous
    $$y' + b y = 0$$

    \begin{itemize}
    \item \emph{homogeneous} means the right-hand side is zero
    \item \emph{constant-coefficient} means $b$ is constant
    \item the solution is (``by inspection''?)
        $$y(x) = A e^{-bx}$$
    \end{itemize}
\end{itemize}
\end{frame}


\begin{frame}{first-order linear review cont.}

\begin{itemize}
\item special case 2: homogeneous (but otherwise general)
    $$y' + P(x) y = 0$$

    \begin{itemize}
    \item now we need an integrating factor $\mu(x) = e^{Q(x)}$ where $Q(x) = \int P(x)\,dx$ is any antiderivative of $P(x)$
    \item multiplying by $\mu$ the equation becomes $\left(\mu(x) y(x)\right)' = 0$
    \item thus
        $$e^{Q(x)} y(x) = A$$
    \item thus the solution is
        $$y(x) = A e^{-Q(x)}$$
    \item homogeneous: \alert{a multiple of a solution is still a solution}
    \end{itemize}
\end{itemize}
\end{frame}


\begin{frame}{first-order linear review cont.$^2$}

\begin{itemize}
\item general \emph{nonhomogeneous} case: first-order linear
    $$y' + P(x) y = f(x)$$

    \begin{itemize}
    \item need same integrating factor; multiplying by $\mu=e^{Q(x)}$ yields $\left(\mu(x) y(x)\right)' = \mu(x) f(x)$
    \item integrate:
        $$e^{Q(x)} y(x) = A + \int_a^x e^{Q(t)} f(t)\,dt$$

        \begin{itemize}
        \item where $Q(x)=\int P(x)\,dx$ is \emph{any} antiderivative of $P(x)$
        \item written to emphasize right side has a free constant $A$
        \end{itemize}
    \item thus the solution is
        $$y(x) = A e^{-Q(x)} +  e^{-Q(x)} \int_a^x e^{Q(t)} f(t)\,dt$$
    \item \alert{solution is the homogeneous solution plus a particular solution}
    \end{itemize}
\end{itemize}
\end{frame}


\begin{frame}{higher-order linear DEs: overview}

for $n$th-order linear equations
\begin{equation*}
    a_n(x) y^{(n)} + a_{n-1}(x) y^{(n-1)} + \dots + a_1(x) y' + a_0(x) y = g(x)
\end{equation*}
\alert{new versions of all four comments in red on the previous slides still apply}
\end{frame}


\begin{frame}{overview cont.}

\begin{equation*}
    a_n(x) y^{(n)} + a_{n-1}(x) y^{(n-1)} + \dots + a_1(x) y' + a_0(x) y \stackrel{\ast}{=} g(x)
\end{equation*}
\begin{enumerate}
\item if $a_n(x)\ne 0$ then we can divide by it:
    $$y^{(n)} + b_{n-1}(x) y^{(n-1)} + \dots + b_1(x) y' + b_0(x) y = f(x)$$
\item easiest case (\S4.3) is homogeneous and constant coefficient
    $$a_n y^{(n)} + a_{n-1} y^{(n-1)} + \dots + a_1 y' + a_0 y = 0$$
\item for the associated homogeneous equation to $\ast$,
\begin{equation*}
    a_n(x) y^{(n)} + a_{n-1}(x) y^{(n-1)} + \dots + a_1(x) y' + a_0(x) y = 0
\end{equation*}
any multiple of, or sum of, solutions is again a solution
\item solutions of $\ast$ are always solutions of the homogeneous equation plus a particular solution
\end{enumerate}
\end{frame}


\begin{frame}{solutions exist}

\begin{theorem}  
\begin{itemize}
\item Consider the linear DE
\begin{equation*}
    a_n(x) y^{(n)} + a_{n-1}(x) y^{(n-1)} + \dots + a_1(x) y' + a_0(x) y = g(x)
\end{equation*}
If the functions $a_j(x)$ and $g(x)$ are continuous on some interval, and if $a_n(x) \ne 0$ on that interval, then solutions exist.
\item Furthermore, if $x_0$ is in that interval then there is exactly one solution which satisfies the initial values
\begin{align*}
y(x_0) &= y_0 \\
y'(x_0) &= y_1 \\
 &\vdots \\
y^{(n-1)}(x_0) &= y_{n-1}
\end{align*}
\end{itemize}
\end{theorem}
\end{frame}


\begin{frame}{linear, homogeneous, constant-coefficient}

\begin{itemize}
\item furthermore, linear DEs which are \alert{homogeneous and constant-coefficient always have exponential solutions}
    \begin{itemize}
    \item you can always find \emph{at least one solution} $y=e^{mx}$
    \item \emph{and} multiples and sums of solutions are solutions
    \end{itemize}
\item \emph{example 1}: solve, by trying $y(x)=e^{mx}$, the equation
    $$y'' + 4 y' - 5 y = 0$$

\vspace{30mm}
\small
\noindent \emph{fundamental set} of solutions:

\medskip
\noindent \emph{general solution}:
\end{itemize}
\end{frame}


\begin{frame}{example 2}

\begin{itemize}
\item \emph{example 2}: solve, by trying $y(x)=e^{mx}$, the equation
    $$y''' + 3 y'' - y' - 3 y = 0$$

\vspace{40mm}
\small
\noindent \emph{fundamental set} of solutions:

\medskip
\noindent \emph{general solution}:
\end{itemize}
\end{frame}


\begin{frame}{linear combination}

\begin{itemize}
\item examples 1 and 2 are from \S 4.3 but they let me illustrate the language introduced in \S 4.1 \alert{$\longleftarrow$ read this section!}
\item for example,
\begin{theorem}
If $y_1(x), y_2(x), \dots, y_n(x)$ solve a linear and homogeneous DE
\begin{equation*}
    a_n(x) y^{(n)} + a_{n-1}(x) y^{(n-1)} + \dots + a_1(x) y' + a_0(x) y = 0
\end{equation*}
then any linear combination
    $$y(x) = c_1 y_1(x) + c_2 y_2(x) + \dots + c_n y_n(x)$$
is also a solution.
\end{theorem}

\item idea: for linear and homogeneous DEs you can form a more general solution from any set of solutions
    \begin{itemize}
    \item see examples 1 and 2
    \end{itemize}        
\end{itemize}
\end{frame}


\begin{frame}{linear dependence and independence}

\begin{itemize}
\item a set of functions $\{f_1(x),\dots,f_n(x)\}$ is \emph{linearly dependent} if you can combine with constants $c_1,\dots,c_n$, \underline{some of which} \underline{are not zero}, and get the zero function:
    $$c_1 f_1(x) + c_2 f_2(x) + \dots + c_n f_n(x) = 0$$
\item a set is \emph{linearly independent} if it is not linearly dependent
\item \emph{example}:
    $$f_1(x) = x^2 + x, \quad f_2(x) = x^2 - x, \quad f_3(x) = 5x$$
are linearly dependent because
    $$1\cdot f_1(x) - 1\cdot f_2(x) - \frac{2}{5}\cdot f_3(x) = 0$$
\end{itemize}
\end{frame}


\begin{frame}{example 3}

\begin{itemize}
\item recall from example 1 that $f_1(x)=e^x$ and $f_2(x)=e^{-5x}$ are solutions to $y'' + 4 y' - 5 y = 0$
\item \emph{example 3}:  Find a solution of the initial value problem
    $$y'' + 4 y' - 5 y = 0, \quad y(0) = 2, \quad y'(0) = -3$$

\vspace{35mm}
\item this calculation works because $\{f_1(x),f_2(x)\}=\{e^x,e^{-5x}\}$ is a linearly-independent set
\end{itemize}
\end{frame}


\begin{frame}{checking linear independence}

\begin{itemize}
\item generally it would require linear algebra thinking to check whether a set of functions is linearly independent
\item \emph{but} there is a determinant to save you from thinking!
\item definition.  given functions $f_1(x),\dots,f_n(x)$ the \emph{Wronskian} is the deteriminant where the rows are derivatives:
    $$W(f_1,\dots,f_n) = \det \left(\begin{bmatrix}
    f_1 & f_2 & \dots & f_n \\
    f_1' & f_2' & \dots & f_n' \\
    \vdots & \vdots & & \vdots \\
    f_1^{(n-1)} & f_2^{(n-1)} & \dots & f_n^{(n-1)}
    \end{bmatrix}\right)$$
\begin{minipage}[t]{0.25\textwidth}
\item \emph{example 4}: find the Wronskian of $\{e^{-3x},e^{-x},e^{x}\}$
\end{minipage}

\vspace{15mm}
\end{itemize}
\end{frame}


\begin{frame}{theorem}

\begin{theorem}  Suppose $\{y_1(x),y_2(x),\dots,y_n(x)\}$ are solutions of a homogeneous linear $n$th-order differential equation on some interval.  Then
\begin{itemize}
\item The set of solutions is linearly-independent if and only if the Wronskian $W(y_1,\dots,y_n)$ is nonzero on the interval.
\item If the Wronskian $W(y_1,\dots,y_n)$ is nonzero at some point on the interval then it is nonzero on the whole interval.
\end{itemize}
\end{theorem}

\end{frame}


\begin{frame}{fundamental set}

definition.  a set of $n$ linearly-independent solutions $\{y_1(x),y_2(x),\dots,y_n(x)\}$ of the homogeneous linear $n$th-order differential equation
\begin{equation*}
    a_n(x) y^{(n)} + a_{n-1}(x) y^{(n-1)} + \dots + a_1(x) y' + a_0(x) y = 0
\end{equation*}
is a \emph{fundamental set of solutions}

\begin{itemize}
\item once you have a fundamental set then the \emph{general solution} of the above DE is
    $$y(x) = c_1 y_1(x) + c_2 y_2(x) + \dots + c_n y_n(x)$$
\item if you have fewer than $n$ solutions, or they are not linearly independent, then the linear combination \emph{is} a solution, but not fully general
\end{itemize}
\end{frame}


\begin{frame}{exercise 25 in \S 4.1}

\begin{itemize}
\item \emph{exercise \#25}:  Verify that the functions form a fundamental set  of solutions on the interval.  Form the general solution.
    $$y''-2y'+5y=0, \qquad \{e^x\cos 2x,e^x\sin 2x\}, \qquad (-\infty,\infty)$$
\end{itemize}

\vspace{50mm}
\end{frame}


\begin{frame}{exercise 27 in \S 4.1}

\begin{itemize}
\item \emph{exercise \#27}:  Verify that the functions form a fundamental set  of solutions on the interval.  Form the general solution.
    $$x^2 y''-6xy'+12y=0, \qquad \{x^3,x^4\}, \qquad (0,\infty)$$
\end{itemize}

\vspace{50mm}
\end{frame}


\begin{frame}{expectations}

\begin{itemize}
\item just watching this video is \emph{not} enough!
     \begin{itemize}
     \item see ``found online'' videos at

     \centerline{\href{https://bueler.github.io/math302/week6.html}{\tt \color{cyan} bueler.github.io/math302/week6.html}}
     \item \emph{read} section 4.1 in the textbook
         \begin{itemize}
         \item know the meaning/definitions of:

\bigskip
             \hspace{-10mm} \begin{tabular}{ll}
             \emph{homogeneous}                     & \emph{linearly independent} \\
             \emph{nonhomogeneous}                  & \emph{Wronskian} \\
             \emph{associated homogeneous equation} & \emph{fundamental set of solutions} \\
             \emph{linear combination}              & \emph{general solution} \\
             \emph{superposition}                   & \emph{particular solution} \\
             \emph{linearly dependent}              & \emph{complementary function}              \end{tabular}

\bigskip
         \item we will soon focus more on nonhomogeneous equations (\S4.4), but the homogeneous case is central for a while (\S4.3 and then \S4.2)
         \item \emph{but} during this course I will not ask questions about ``boundary conditions'' and ``boundary value problems''
         \item there is quite a bit of new language in \S4.1!
         \end{itemize}
     \item \emph{do} the WebAssign exercises for section 4.1
     \end{itemize}
\end{itemize}
\end{frame}

\end{document}

