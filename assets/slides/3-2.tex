\documentclass{beamer}

\usepackage[utf8]{inputenc}
\usepackage{fancybox,fancyvrb}
\usepackage{environ}
\usepackage{tikz}

\beamertemplatenavigationsymbolsempty
\setbeamertemplate{footline}[frame number]
\usetheme{Pittsburgh}

\newcommand\enumnum[1]{{\renewcommand{\insertenumlabel}{#1}%
      \usebeamertemplate{enumerate item} \,}}

\newcommand{\grad}{\nabla}
\newcommand{\ih}{\boldsymbol{\hat{\textbf{\i}}}}
\newcommand{\jh}{\boldsymbol{\hat{\textbf{\j}}}}
\newcommand{\vF}{\boldsymbol{\vec{\textbf{F}}}}


\title{3.2 Nonlinear Models}

\subtitle{a lesson for MATH F302 Differential Equations}

\author{Ed Bueler, Dept.~of Mathematics and Statistics, UAF}

\date{\tiny \today}


\begin{document}
\setbeamertemplate{itemize item}{$\bullet$}
\setbeamertemplate{itemize subitem}{$\circ$}


\begin{frame}
\titlepage

\centerline{\tiny for textbook: \, D. Zill, \emph{A First Course in Differential Equations with Modeling Applications}, 11th ed.}
%\color{green!40!blue}
\end{frame}


\begin{frame}{outline}

\begin{itemize}
\item like 3.1, section 3.2 has modeling problems
    \begin{itemize}
    \item challenging
    \item uses separable linear equations from \S2.2: $\frac{dy}{dx} = g(x) h(y)$
    \end{itemize}
\item plan for these slides:
    \begin{itemize}
    \item start with five standard indefinite integrals
        \begin{itemize}
        \item you did them in calculus I and II
        \item \dots but a reminder is appropriate
        \end{itemize}
    \item two explanations of where the ``logistic equation'' comes from
    \item two exercises from \S3.2
    \end{itemize} 
\end{itemize}
\end{frame}


\begin{frame}{integrals you will need}

\begin{itemize}
\item {\color{blue} integral 1} was already used in \S3.1:
    $$\int \frac{dy}{ay+b} = \hspace{80mm}$$

\vspace{10mm}
\item {\color{blue} integral 2} is sometimes useful:
    $$\int \frac{dx}{x^2+a^2} = \hspace{80mm}$$

\vspace{25mm}
\end{itemize}
\end{frame}


\begin{frame}{integrals, cont.}

\begin{itemize}
\item {\color{blue} integral 3} is the main job in \S3.2:
    $$\int \frac{dz}{z(a-bz)} = \hspace{80mm}$$

\vspace{50mm}
\end{itemize}
\end{frame}


\begin{frame}{integrals, cont.$^2$}

\begin{itemize}
\item {\color{blue} integral 4} was needed for first-order linear equations (\S2.3) and will keep re-appearing in chapter 4 and onward:
    $$\int x^n e^{ax}\,dx = \hspace{80mm}$$

\vspace{35mm}

    \begin{itemize}
    \item[example (a):] $\int x e^x\,dx = $
    
    \bigskip\bigskip
    \item[example (b):] $\int_0^\infty x^2 e^{-x}\,dx = $
    
    \bigskip\bigskip
    \end{itemize}
\end{itemize}
\end{frame}


\begin{frame}{integrals, cont.$^3$}

\begin{itemize}
\item {\color{blue} integral 5} \dots same:
    $$\int e^{at} \cos t\,dt = \hspace{80mm}$$

\vspace{60mm}
\end{itemize}
\end{frame}


\begin{frame}{how to remember integrals}

\begin{itemize}
\item even if you have a good memory, I think it is silly to try to memorize the \emph{results} above
\item but \emph{do}
    \begin{itemize}
    \item try to remember what choices which were made, and \emph{why}
    \item think of integration-by-parts as undoing the product rule
        \begin{itemize}
        \item see Mini-Project 1
        \end{itemize}
    \item remember how to start partial fractions
    \end{itemize}
\end{itemize}
\end{frame}


\begin{frame}{logistic equation: explanation 1}

\small
\begin{quotation}
\noindent Suppose we have a dish with no bacteria, and we plan to supply enough food and water every hour to sustain the needs of $N$ bacteria.  Question: What will happen when we introduce a few bacteria ($P_0 \ll N$)?
\end{quotation}

\normalsize
\begin{itemize}
\item \emph{initial model}:  $P(t)$ is population, $P(0)=P_0$, $k>0$,
    $$\frac{dP}{dt} = k P$$

\vspace{-2mm}
    \begin{itemize}
    \item has solution $P(t) = P_0 e^{kt}$
    \item predicts unlimited exponential growth
    \item at some point (clearly, and in fact!) population growth will be limited by food and water
    \end{itemize}
\item \emph{better model}: make coefficient get smaller as $P$ gets larger
    $$\frac{dP}{dt} = \left(\begin{matrix} \text{coefficient which gets} \\ \text{smaller when $P$ approaches $N$}\end{matrix} \right) P$$
\end{itemize}
\end{frame}


\begin{frame}{explanation 1, cont.}

\begin{itemize}
\item a formula with the desired property:
    $$\boxed{\frac{dP}{dt} = k \left(1-\frac{P}{N}\right) P} \hspace{50mm}$$
\item equivalently with $b=k/N >0$:
    $$\boxed{\frac{dP}{dt} = b P \left(N-P\right)} \hspace{50mm}$$
\item initial model was linear; this one is nonlinear
     \begin{itemize}
     \item but separable!
     \end{itemize}
\item \emph{exercise:} draw phase portrait and typical solutions
\end{itemize}
\end{frame}


\begin{frame}{logistic equation: explanation 2}

\small
\begin{quotation}
\noindent Suppose a population changes by two mechanisms, namely births (rate is proportional to current population) and deaths by conflicts between the members of the population (rate is proportional to number of interactions, modeled as the square of the population).
\end{quotation}

\normalsize
\begin{itemize}
\item write down the DE, with constants $k>0$ and $\ell>0$:
    $$\boxed{\frac{dP}{dt} = k P - \ell P^2}$$
\item equivalently, using $b=\ell$ and $N=k/\ell$, again write as $\frac{dP}{dt} = b P \left(N-P\right)$
\item look at the last three boxed equations
    \begin{itemize}
    \item any one is the \emph{logistic equation}
    \item the three forms are equivalent by renaming variables
    \end{itemize}
\end{itemize}
\end{frame}


\begin{frame}{logistic equation: solution}

\begin{itemize}
\item assuming $b,N$ are positive constants, find the general solution:
    $$\frac{dP}{dt} = b P \left(N-P\right) \hspace{70mm}$$
\end{itemize}

\vspace{50mm}
\end{frame}


\begin{frame}[fragile]

\frametitle{exercise 4 in \S3.2}

\small
\begin{quotation}
\noindent Census data for the United States between 1790 and 1950 are given in the Table on page 102.  (a)  Construct a logistic population model using the data from 1790, 1850, and 1910.  (b)  Construct a table comparing actual census population with the population predicted by the model in part (a).  Compute the absolute error and percentage relative error for each census year. 
\end{quotation}

\normalsize
\begin{itemize}
\item step 1: put the Table in a (Matlab) code

\begin{Verbatim}[fontsize=\scriptsize]
year = 1790:10:1950;    % list of 17 values
pop = [  3.929,   5.308,   7.240,   9.638,  12.866, ...
        17.069,  23.192,  31.433,  38.558,  50.156, ...
        62.948,  75.996,  91.972, 105.711, 122.775, ...
       131.669, 150.697];
\end{Verbatim}

\item \dots and plot it; see result on next slide

\begin{Verbatim}[fontsize=\scriptsize]
plot(year, pop, '.k', 'markersize', 12)
xlabel('census year'),  ylabel('population in millions')
axis([1780 1960 0 160]),  grid on
\end{Verbatim}
\end{itemize}
\end{frame}


\begin{frame}{exercise 4, cont.}

\begin{center}
\includegraphics[width=0.75\textwidth]{figs/uscensus}
\end{center}
\end{frame}


\begin{frame}{exercise 4, cont.$^2$}

\begin{itemize}
\item we are supposed to construct a logistic model using 1790, 1850, 1910 data
\item i.e.~use those years to determine $N,\alpha,b$ in $P(t) = \frac{N}{1 + \alpha e^{-Nbt}}$
\item let $t$ be the number of years after 1790
\item solve 3 equations in 3 unknowns
\small
\begin{align*}
3.929 &= \frac{N}{1 + \alpha}, \quad 23.192 = \frac{N}{1 + \alpha e^{-60 Nb}}, \quad 91.972 = \frac{N}{1 + \alpha e^{-120 Nb}}
\end{align*}
\normalsize
\item this algebra job is hard; I started by clearing denominators and letting $s=e^{-60b}$:
\small
\begin{align*}
3.929 (1+\alpha) - N &= 0 \\
23.192 (1 + \alpha s^N) - N &= 0 \\
91.972 (1 + \alpha s^{2N}) - N &= 0
\end{align*}
\item then I used Newton's method (not shown) to get FIXME $N=197.274, \alpha=49.2096, s=0.962509$
\end{itemize}
\end{frame}


\begin{frame}{exercise 4, cont.$^2$}

\begin{itemize}
\item X
\end{itemize}
\end{frame}


\begin{frame}{exercise 15 in \S3.2}

\begin{itemize}
\item X
\end{itemize}
\end{frame}


\begin{frame}{exercise 15, cont.}

\begin{itemize}
\item X
\end{itemize}
\end{frame}

\begin{frame}{expectations}

\begin{itemize}
\item just watching this video is \emph{not} enough!
     \begin{itemize}
     \item see ``found online'' videos at

     \centerline{\href{https://bueler.github.io/math302/week5.html}{\tt \color{cyan} bueler.github.io/math302/week5.html}}
     \item \emph{read} section 3.2 in the textbook
         \begin{itemize}
         \item \emph{read} example 2 on pages 100--101 regarding a chemical equation
         \item try various exercises from section 3.2
         \item but note I will not put any ``leaking tanks''-type questions (exercises 11--14) on quizzes/exams
         \end{itemize}
     \item \emph{do} the WebAssign exercises for section 3.2
     \end{itemize}

\bigskip
\item up next are second-order linear equations in section 4.1
    \begin{itemize}
    \item we will return to section 3.3 near the end of the course, linking it with material in chapter 8
    \end{itemize}
\end{itemize}
\end{frame}

\end{document}

