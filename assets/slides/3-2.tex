\documentclass{beamer}

\usepackage[utf8]{inputenc}
\usepackage{fancybox,fancyvrb}
\usepackage{environ}
\usepackage{tikz}

\beamertemplatenavigationsymbolsempty
\setbeamertemplate{footline}[frame number]
\usetheme{Pittsburgh}

\newcommand\enumnum[1]{{\renewcommand{\insertenumlabel}{#1}%
      \usebeamertemplate{enumerate item} \,}}

\newcommand{\grad}{\nabla}
\newcommand{\ih}{\boldsymbol{\hat{\textbf{\i}}}}
\newcommand{\jh}{\boldsymbol{\hat{\textbf{\j}}}}
\newcommand{\vF}{\boldsymbol{\vec{\textbf{F}}}}


\title{3.2 Nonlinear Models}

\subtitle{a lesson for MATH F302 Differential Equations}

\author{Ed Bueler, Dept.~of Mathematics and Statistics, UAF}

\date{\tiny \today}


\begin{document}
\setbeamertemplate{itemize item}{$\bullet$}
\setbeamertemplate{itemize subitem}{$\circ$}


\begin{frame}
\titlepage

\centerline{\tiny for textbook: \, D. Zill, \emph{A First Course in Differential Equations with Modeling Applications}, 11th ed.}
%\color{green!40!blue}
\end{frame}


\begin{frame}{outline}

\begin{itemize}
\item like 3.1, section 3.2 has modeling problems
    \begin{itemize}
    \item challenging
    \item uses separable linear equations from \S2.2: $\frac{dy}{dx} = g(x) h(y)$
    \end{itemize}
\item plan for these slides:
    \begin{itemize}
    \item start with five standard indefinite integrals
        \begin{itemize}
        \item you did them in calculus I and II
        \item \dots but a reminder is appropriate
        \end{itemize}
    \item two explanations of where the ``logistic equation'' comes from
    \item two exercises from \S3.2
    \end{itemize} 
\end{itemize}
\end{frame}


\begin{frame}{integrals you will need}

\begin{itemize}
\item {\color{blue} integral 1} was already used in \S3.1:
    $$\int \frac{dy}{ay+b} = \hspace{80mm}$$

\vspace{10mm}
\item {\color{blue} integral 2} is sometimes useful:
    $$\int \frac{dx}{x^2+a^2} = \hspace{80mm}$$

\vspace{25mm}
\end{itemize}
\end{frame}


\begin{frame}{integrals, cont.}

\begin{itemize}
\item {\color{blue} integral 3} is the main job in \S3.2:
    $$\int \frac{dz}{z(a-bz)} = \hspace{80mm}$$

\vspace{50mm}
\end{itemize}
\end{frame}


\begin{frame}{integrals, cont.$^2$}

\begin{itemize}
\item {\color{blue} integral 4} was needed for first-order linear equations (\S2.3) and will keep re-appearing in chapter 4 and onward:
    $$\int x^n e^{ax}\,dx = \hspace{80mm}$$

\vspace{35mm}

    \begin{itemize}
    \item[example (a):] $\int x e^x\,dx = $
    
    \bigskip\bigskip
    \item[example (b):] $\int_0^\infty x^2 e^{-x}\,dx = $
    
    \bigskip\bigskip
    \end{itemize}
\end{itemize}
\end{frame}


\begin{frame}{integrals, cont.$^3$}

\begin{itemize}
\item {\color{blue} integral 5} \dots same:
    $$\int e^{at} \cos t\,dt = \hspace{80mm}$$

\vspace{60mm}
\end{itemize}
\end{frame}


\begin{frame}{how to remember integrals}

\begin{itemize}
\item even if you have a good memory, I think it is silly to try to memorize the \emph{results} above
\item but \emph{do}
    \begin{itemize}
    \item try to remember what choices which were made, and \emph{why}
    \item think of integration-by-parts as undoing the product rule
        \begin{itemize}
        \item see Mini-Project 1
        \end{itemize}
    \item remember how to start partial fractions
    \end{itemize}
\end{itemize}
\end{frame}


\begin{frame}{logistic equation: explanation 1}

\begin{quotation}
\noindent Suppose we have a dish with no bacteria, and we plan to supply enough food and water every hour to sustain the needs of $N$ bacteria.  \textbf{Question}: What will happen when we introduce a few bacteria ($P_0 \ll N$)?
\end{quotation}

\begin{itemize}
\item \emph{initial model}:  $P(t)$ is population, $P(0)=P_0$, $k>0$,
    $$\frac{dP}{dt} = k P$$

\vspace{-2mm}
    \begin{itemize}
    \item has solution $P(t) = P_0 e^{kt}$
    \item predicts unlimited exponential growth
    \item at some point (clearly, and in fact!) population growth will be limited by food and water
    \end{itemize}
\item \emph{better model}: make coefficient get smaller as $P$ gets larger
    $$\frac{dP}{dt} = \left(\begin{matrix} \text{coefficient which gets} \\ \text{smaller when $P$ approaches $N$}\end{matrix} \right) P$$
\end{itemize}
\end{frame}


\begin{frame}{explanation 1, cont.}

\begin{itemize}
\item a formula with the desired property:
    $$\boxed{\frac{dP}{dt} = k \left(1-\frac{P}{N}\right) P} \hspace{50mm}$$
\item equivalently with $b=k/N >0$:
    $$\boxed{\frac{dP}{dt} = b P \left(N-P\right)} \hspace{50mm}$$
\item initial model was linear; this is nonlinear
     \begin{itemize}
     \item but separable!
     \end{itemize}
\item \emph{exercise:} draw phase portrait and typical solutions
\end{itemize}
\end{frame}


\begin{frame}{logistic equation: explanation 2}

\begin{quotation}
\noindent Suppose a population changes by two mechanisms, namely births (rate is proportional to current population) and deaths by conflicts between the members of the population (rate is proportional to number of interactions, modeled as the square of the population).
\end{quotation}

\begin{itemize}
\item write down the DE, with constants $k>0$ and $\ell>0$:
    $$\boxed{\frac{dP}{dt} = k P - \ell P^2}$$
\item equivalently, using $b=\ell$ and $N=k/\ell$, again write as $\frac{dP}{dt} = b P \left(N-P\right)$
\item look at the three boxed equations \dots \emph{any} of them is the \emph{logistic equation}
\end{itemize}
\end{frame}


\begin{frame}{logistic equation: solution}

\begin{itemize}
\item assuming $b,N$ are positive constants, find the general solution:
    $$\frac{dP}{dt} = b P \left(N-P\right) \hspace{70mm}$$
\end{itemize}

\vspace{50mm}
\end{frame}


\begin{frame}{exercise 4 in \S3.2}

\begin{itemize}
\item X
\end{itemize}
\end{frame}


\begin{frame}{exercise X in \S3.2}

\begin{itemize}
\item X
\end{itemize}
\end{frame}


\begin{frame}{expectations}

\begin{itemize}
\item just watching this video is \emph{not} enough!

\item also:
     \begin{itemize}
     \item see ``found online'' videos at

     \centerline{\href{https://bueler.github.io/math302/week5.html}{\tt \color{cyan} bueler.github.io/math302/week5.html}}
     \item \emph{read} section 3.2 in the textbook
     \item \emph{do} the WebAssign exercises for section 3.2
     \end{itemize}
\end{itemize}
\end{frame}

\end{document}

