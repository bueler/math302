\documentclass{beamer}

\usepackage[utf8]{inputenc}
\usepackage{fancybox}
\usepackage{environ}

\usepackage{tikz}

\beamertemplatenavigationsymbolsempty

\title{2.2 Separable Equations}

\subtitle{a lecture for MATH F302 Differential Equations}

\author{Ed Bueler, Dept.~of Mathematics and Statistics, UAF}

\date{Fall 2023}


\usetheme{Pittsburgh}


\begin{document}

\setbeamertemplate{itemize item}{$\bullet$}
\setbeamertemplate{itemize subitem}{$\circ$}


\begin{frame}
\titlepage

\centerline{\tiny for textbook: \, D. Zill, \emph{A First Course in Differential Equations with Modeling Applications}, 11th ed.}
%\color{green!40!blue}
\end{frame}


\begin{frame}{regarding chapters 1 and 2}
\begin{itemize}
\item the purpose of textbook chapter 1 is to sketch the language and meaning of differential equations
\item the purpose of chapter 2 is to actually solve some differential equations \underline{by hand}
    \begin{itemize}
    \item[$\circ$] section 2.2 is the first example: separable equations
    \item[$\circ$] \alert{warning:} by-hand methods can be difficult or impossible on a given differential equation
    \item[$\circ$] however, the examples where we know how to solve are often important in practice
    \end{itemize}
\end{itemize}
\end{frame}


\begin{frame}{separable differential equations}

\begin{definition}
a \emph{separable} differential equation can be put in the form
        $$\boxed{\frac{dy}{dx} = g(x) h(y)}$$
\end{definition}

\begin{itemize}
\item Example.
        $$\frac{dy}{dx} = \frac{y\cos(x)}{1+y^2}$$
%PUT IT IN THIS FORM
\item \textbf{Not} an example.
        $$\frac{dy}{dx} = \cos(x) + y$$
\item Also \textbf{not} an example.
        $$\frac{dy}{dx} = \sin(x + y^2)$$
\end{itemize}
\end{frame}


\begin{frame}{emphasis on \emph{can}}

\begin{definition}
a \emph{separable} differential equation \alert{can} be put in the form
        $$\boxed{\frac{dy}{dx} = g(x) h(y)}$$
\end{definition}

\begin{itemize}
\item Example.
        $$p(y) \frac{dy}{dx} = g(x)$$
\uncover<2>{define $h(y)=1/p(y)$ to make into standard separable form}
\end{itemize}
\end{frame}


\begin{frame}{how to solve separable equations?}

\begin{itemize}
\item move $y$ stuff to left and $x$ to right:
        $$\frac{dy}{dx} = g(x) h(y)$$
    $$\frac{1}{h(y)} \frac{dy}{dx} = g(x)$$
    $$\frac{1}{h(y)}\,dy = g(x)\,dx$$
\item<2> integrate both sides:
    $$\int \frac{1}{h(y)}\,dy = \int g(x)\, dx$$
\end{itemize}
\end{frame}


\begin{frame}{how to solve separable equations?}

\begin{itemize}
\item alternative appearance with $p(y)=1/h(y)$:
        $$p(y) \frac{dy}{dx} = g(x)$$
\item<2> move $y$ stuff to left and $x$ to right:
    $$p(y) \,dy = g(x)\,dx$$
\item<2> integrate both sides:
    $$\int p(y)\,dy = \int g(x)\, dx$$
\end{itemize}
\end{frame}


\begin{frame}{why does it work?}

\begin{itemize}
\item the method works because of the chain rule
\item the integrals you are really doing are both with respect to $x$:
    $$\int p(y(x))\,\frac{dy}{dx}\,dx = \int g(x)\, dx$$
\end{itemize}
\end{frame}


\begin{frame}{how do you finish up?}

\begin{itemize}
\item once you do the integrals
    $$\int p(y)\,dy = \int g(x)\, dx$$
then solve for $y$, if possible, to get an explicit solution
\item if you cannot solve for $y$ then the solution remains implicit
\end{itemize}
\end{frame}


\begin{frame}{example 1}
\begin{itemize}
\item example: find $y(x)$ if
    $$\frac{dy}{dx} = x y^2$$
\end{itemize}

\vspace{50mm}
\end{frame}


\begin{frame}{example 2}
\begin{itemize}
\item some familiar equations are also separable
\item example: find $y(x)$ if
    $$\frac{dy}{dx} = - 5 y$$
\end{itemize}

\vspace{50mm}
\end{frame}


\begin{frame}{example 3}
\begin{itemize}
\item what if there are initial conditions?
\item example: find $z(t)$ if $z(4)=1$ and
    $$z' = \frac{e^{-z}}{t}$$
\end{itemize}

\vspace{50mm}
\end{frame}


\begin{frame}{example 4}
\begin{itemize}
\item you may end up only knowing the solution implicitly
\item example: find $y(x)$ if
    $$\frac{dy}{dx} = \frac{x(1-x)}{y(2+y)}$$
\end{itemize}

\vspace{50mm}
\end{frame}


\begin{frame}{standard expectations}

to learn this material, just listening to a lecture is \emph{not} enough
\begin{itemize}
\item please \emph{read} section 2.2 in the textbook
\item please \emph{do} the Homework for section 2.2
\item search ``separable ODEs'' at YouTube to see more examples
\end{itemize}
\end{frame}

\end{document}

