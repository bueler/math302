\documentclass[urlcolor=blue,dvipsnames]{beamer}

\usepackage[utf8]{inputenc}
\usepackage{fancybox,fancyvrb}
\usepackage{environ,xspace,empheq}
\usepackage{tikz}
\hypersetup{colorlinks,linkcolor=,urlcolor=cyan}

\beamertemplatenavigationsymbolsempty
\setbeamertemplate{footline}[frame number]
\usetheme{Pittsburgh}

\newcommand\enumnum[1]{{\renewcommand{\insertenumlabel}{#1}%
      \usebeamertemplate{enumerate item} \,}}

\newcommand{\grad}{\nabla}
\newcommand{\ih}{\boldsymbol{\hat{\textbf{\i}}}}
\newcommand{\jh}{\boldsymbol{\hat{\textbf{\j}}}}
\newcommand{\vF}{\boldsymbol{\vec{\textbf{F}}}}
\newcommand{\Matlab}{\textsc{Matlab}\xspace}
\newcommand{\Octave}{\textsc{Octave}\xspace}


\title{9.2 Runge-Kutta methods}

\subtitle{a lesson for MATH F302 Differential Equations}

\author{Ed Bueler, Dept.~of Mathematics and Statistics, UAF}

\date{\tiny \today}


\begin{document}
\setbeamertemplate{itemize item}{$\bullet$}
\setbeamertemplate{itemize subitem}{$\circ$}
\renewcommand{\thefootnote}{{\color{green} \arabic{footnote}}}

\begin{frame}
\titlepage

\centerline{\tiny for textbook: \, D. Zill, \emph{A First Course in Differential Equations with Modeling Applications}, 11th ed.}
%\color{green!40!blue}
\end{frame}


\begin{frame}{the Runge-Kutta happy family}

\begin{itemize}
\item these slides describe and test the most-famous \emph{Runge-Kutta} (RK) method, namely \alert{(classical) RK4} which is \alert{order 4}
    \begin{itemize}
    \item RK4 dates to $\sim$1900, \emph{before} invention of electronic computers
    \item there are dozens of useful RK methods of all orders $\ge 1$
    \item Euler's method is the order 1 RK method
    \item improved Euler is an order 2 RK method
    \item $\infty$ly-many methods in the family \dots
    \end{itemize}
\item RK4 was accurate enough for most ODE solutions in science and engineering until $\sim$1960
\item better computers and programming languages allow reliable/debugged implementations of better-than-RK4 methods like \texttt{ode45} in \Matlab/\Octave
    \begin{itemize}
    \item \dots which are \emph{not} a lot more accurate
    \item instead, modern methods like \texttt{ode45} are \emph{adaptive} so the user does not need to choose a step size $h$
    \end{itemize}
\end{itemize}
\end{frame}


\begin{frame}{improved Euler dance}

\begin{itemize}
\item ODE IVP: $\frac{dy}{dt} = t-y^2$, $y(0)=1$
\item show on the direction field: improved Euler with $h=1$
\end{itemize}

\hfill \includegraphics[width=0.8\textwidth]{figs/for-rk4-dir-field}
\end{frame}


\begin{frame}{RK4 dance}

\begin{itemize}
\item ODE IVP: $\frac{dy}{dt} = t-y^2$, $y(0)=1$
\item show on the direction field: RK4 with $h=1$
\end{itemize}

\hfill \includegraphics[width=0.8\textwidth]{figs/for-rk4-dir-field}
\end{frame}


\begin{frame}{RK4 \alert{precise} dance}

\begin{itemize}
\item ODE IVP: $\frac{dy}{dt} = t-y^2$, $y(0)=1$
\item show on the direction field: RK4 with $h=1$
\end{itemize}

\hfill \includegraphics[width=0.8\textwidth]{figs/result-rk4-dir-field}
\end{frame}


\begin{frame}{the RK4 formulas}

\small
\begin{itemize}
\item usually written using four slopes ``$k_i$'' from direction field
\item update the $y$-value by a weighted average of these slopes:

\vspace{-6mm}
\begin{minipage}[t]{0.4\textwidth}
\begin{align*}
k_1 &= f(t_n,y_n) \\
k_2 &= f(t_n+\frac{h}{2},y_n+\frac{h}{2} k_1) \\
k_3 &= f(t_n+\frac{h}{2},y_n+\frac{h}{2} k_2) \\
k_4 &= f(t_n+h,y_n+h k_3)
\end{align*}
\end{minipage}\begin{minipage}[t]{0.45\textwidth}
\vspace{10mm}

$$\implies y_{n+1} = y_n + h\, \frac{k_1 + 2 k_2 + 2 k_3 + k_4}{6}$$
\end{minipage}
\item here are the formulas for improved Euler, written the same way:

\vspace{-6mm}
\begin{minipage}[t]{0.4\textwidth}
\begin{align*}
k_1 &= f(t_n,y_n) \\
k_2 &= f(t_n+h,y_n+h k_1)
\end{align*}
\end{minipage}
\begin{minipage}[t]{0.4\textwidth}
\vspace{2mm}

$$\implies y_{n+1} = y_n + h\, \frac{k_1 + k_2}{2}$$
\end{minipage}
\item Euler written the same way:
$$k_1 = f(t_n,y_n) \qquad \implies y_{n+1} = y_n + h\, k_1$$
\end{itemize}
\end{frame}


\begin{frame}{RK4 scheme in one diagram}

\begin{columns}
\begin{column}{0.35\textwidth}
\small
\begin{itemize}
\item drawing this sketch, or similar, will be an extra credit problem on Midterm 2
\end{itemize}
\end{column}
\begin{column}{0.65\textwidth}
\hfill \includegraphics[width=\textwidth]{figs/rk4diagram}
\end{column}
\end{columns}
\end{frame}


\begin{frame}[fragile]
\frametitle{code \texttt{rk4.m}}

\begin{itemize}
\item code posted at the \href{https://bueler.github.io/math302/codes.html}{Codes tab at the course website}
\end{itemize}

\medskip
\VerbatimInput[fontsize=\scriptsize]{../codes/rk4.m}
\end{frame}


\begin{frame}[fragile]
\frametitle{exercise \#7 in \S9.2}

\begin{itemize}
\item \emph{exercise.}  Use the RK4 method with $h=0.1$ to obtain a four-decimal approximation of the indicated value:
    $$y' = e^{-y}, \quad y(0)=0; \qquad y(0.5)$$
\end{itemize}

\noindent \emph{solution.}
\begin{Verbatim}[fontsize=\footnotesize]
>> f = @(t,y) exp(-y);
>> [tt,yy] = rk4(f,[0,0.5],0,0.1);
>> yy(end)
ans =    0.40547
\end{Verbatim}

\end{frame}


\begin{frame}[fragile]
\frametitle{better version of same exercise}

\begin{itemize}
\item \emph{exercise.}  Use the RK4 method with $h=0.1$ to obtain a four-decimal approximation of the indicated value:
    $$y' = e^{-y}, \quad y(0)=0; \qquad y(0.5)$$
Compute the exact value.  Plot both solutions in good style.
\end{itemize}

\hfill \includegraphics[width=0.45\textwidth]{figs/exer7s9p2}

\vspace{-40mm}
\noindent \emph{solution, cont.}

\vspace{20mm}
\begin{Verbatim}[fontsize=\footnotesize]
>> log(1+0.5)
ans =    0.40547
>> t = 0:.001:0.5;
>> plot(t,log(1+t),'k')
>> hold on,  plot(tt,yy,'ro','markersize',12)
>> xlabel t,  ylabel y,  grid on
>> legend('exact solution','RK4 values')
\end{Verbatim}

\end{frame}


\begin{frame}[fragile]
\frametitle{there are \emph{bad} ODE IVPs out there!}

\begin{itemize}
\item \emph{exercise \#15 in \S9.2.}  for this ODE IVP, find $y(1.4)$:
    $$y' = x^2 + y^3, \quad y(1)=1,$$
\end{itemize}

\hfill \includegraphics[width=0.3\textwidth]{figs/blowuprk4}

\vspace{-20mm}
\noindent \emph{solution.}
\begin{Verbatim}[fontsize=\footnotesize]
>> f = @(x,y) x^2 + y^3;
>> [xx,yy] = rk4(f,[1,1.4],1,0.1);
>> plot(xx,yy)
>> yy
yy =
      1      1.2511      1.6934      2.9425      903.03
>> [xxx,yyy] = ode45(f,[1,1.4],1);
warning:  Solving was not successful.  The iterative integration
loop exited at time t = 1.355695 before the endpoint at
tend = 1.400000 was reached.  This may happen if the stepsize
becomes too small.  Try to reduce the value of 'InitialStep'
and/or 'MaxStep' with the command 'odeset'.
warning: called from ...
\end{Verbatim}

\end{frame}


\begin{frame}[fragile]
\frametitle{when does it blow up?}

\begin{itemize}
\item for this ODE IVP, find $y(1.4)$:\footnote{``find $y(1.4)$'' \emph{is} a trick question \dots never gets that far!} \quad $y' = x^2 + y^3, \quad y(1)=1$
\end{itemize}

\begin{Verbatim}[fontsize=\footnotesize]
>> [xxx,yyy] = ode45(f,1:0.00001:1.35569,1);
>> plot(xxx,yyy)                               <-- PLOT BELOW
>> [xxx,yyy] = ode45(f,1:0.00001:1.35570,1);   <-- GENERATES WARNING
\end{Verbatim}

\hfill \includegraphics[width=0.55\textwidth]{figs/blowupode45}
\end{frame}


\begin{frame}{RK4 and \texttt{ode45} summary}

\begin{itemize}
\item the last example shows one reason nonlinear ODEs are interesting \dots the \emph{problems} can be badly behaved
\item but RK4 is the first of powerful tools to handle nonlinear ODE problems via highly-accurate numerical approximations
\item the black-box \texttt{ode45} is a combination of certain ``RK4'' and ``RK5'' formulas
   \begin{itemize}
   \item the two formulas use the same intermediate locations to get slopes from the direction field
   \item \dots which allows adaptive step size computations
   \item if you really want to know, see the \href{https://www.mathworks.com/help/matlab/ref/ode45.html}{Matlab technical doc page on \texttt{ode45}} and the \href{https://en.wikipedia.org/wiki/Dormand-Prince_method}{wikipedia page on the Dormand-Prince method}
   \end{itemize}
\end{itemize}
\end{frame}


\begin{frame}{expectations}

\begin{itemize}
\item just watching this video is \emph{not} enough!
     \begin{itemize}
     \item see ``found online'' videos and stuff at

     \centerline{\href{https://bueler.github.io/math302/week10.html}{\tt \color{cyan} bueler.github.io/math302/week10.html}}
     \item \emph{read} section 9.2 in the textbook
     \item \emph{do} the WebAssign exercises for section 9.2
     \end{itemize}
\end{itemize}
\end{frame}

\end{document}

