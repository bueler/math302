\documentclass[urlcolor=blue,dvipsnames]{beamer}

\usepackage[utf8]{inputenc}
\usepackage{fancybox,fancyvrb}
\usepackage{environ,xspace,empheq}

\usepackage{tikz}
\usetikzlibrary{arrows.meta,decorations.markings,decorations.pathreplacing,fadings,positioning}

\hypersetup{colorlinks,linkcolor=,urlcolor=cyan}

\beamertemplatenavigationsymbolsempty
\setbeamertemplate{footline}[frame number]
\usetheme{Pittsburgh}

%\makeatletter
%\newcommand{\tinytiny}{\@setfontsize{\tinytiny}{4pt}{4pt}}
%\makeatother

\newcommand\enumnum[1]{{\renewcommand{\insertenumlabel}{#1}%
      \usebeamertemplate{enumerate item} \,}}

\newcommand{\grad}{\nabla}
\newcommand{\ih}{\boldsymbol{\hat{\textbf{\i}}}}
\newcommand{\jh}{\boldsymbol{\hat{\textbf{\j}}}}
\newcommand{\vF}{\boldsymbol{\vec{\textbf{F}}}}
\newcommand{\Matlab}{\textsc{Matlab}\xspace}
\newcommand{\Octave}{\textsc{Octave}\xspace}


\title{8.1 Linear systems of first-order ODEs: \\ the basics}

\subtitle{a lesson for MATH F302 Differential Equations}

\author{Ed Bueler, Dept.~of Mathematics and Statistics, UAF}

\date{\tiny \today}


\begin{document}
\setbeamertemplate{itemize item}{$\bullet$}
\setbeamertemplate{itemize subitem}{$\circ$}
\renewcommand{\thefootnote}{{\color{green} \arabic{footnote}}}

\begin{frame}
\titlepage

\centerline{\tiny for textbook: \, D. Zill, \emph{A First Course in Differential Equations with Modeling Applications}, 11th ed.}
%\color{green!40!blue}
\end{frame}

\newcommand{\LL}[1]{\mathcal{L}\left\{#1\right\}}
\newcommand{\LLi}[1]{\mathcal{L}^{-1}\left\{#1\right\}}


\begin{frame}{first-order systems}

\begin{itemize}
\item we have already seen the most general form of a system of ODEs (\S3.3):
\begin{align*}
\frac{dx_1}{dt} &= g_1(t,x_1,x_2,\dots,x_n) \\
\frac{dx_2}{dt} &= g_2(t,x_1,x_2,\dots,x_n) \\
                &\qquad \vdots \\
\frac{dx_n}{dt} &= g_n(t,x_1,x_2,\dots,x_n)
\end{align*}
     \begin{itemize}
     \item my claim in \S3.3: everything is modeled this way
     \end{itemize}
\item Chapter 8 is a special case: the dependent variables $x_i$ only appear with first powers
\end{itemize}
\end{frame}

\begin{frame}{first-order \emph{linear} systems}

\begin{itemize}
\item a \alert{first-order system of linear ODEs} is
\begin{align*}
\frac{dx_1}{dt} &= a_{11}(t) x_1 + a_{12}(t) x_2 + \dots + a_{1n}(t) x_n + f_1(t) \\
\frac{dx_2}{dt} &= a_{21}(t) x_1 + a_{22}(t) x_2 + \dots + a_{2n}(t) x_n + f_2(t) \\
                &\qquad \vdots \\
\frac{dx_n}{dt} &= a_{n1}(t) x_1 + a_{n2}(t) x_2 + \dots + a_{nn}(t) x_n + f_n(t)
\end{align*}
     \begin{itemize}
     \item above is the \emph{general} or \emph{normal form} of the system
     \item the $a_{ij}(t)$ functions are the \emph{coefficients}
     \item if functions $a_{ij}(t)$ are independent of time then we say it is a \emph{constant-coefficient} system
     \item if all $f_i=0$ then the system is \emph{homogeneous}
     \end{itemize}
\end{itemize}
\end{frame}


\begin{frame}{examples}

\small
\begin{itemize}
\item these examples are from the \href{https://bueler.github.io/math302/assets/slides/3-3.pdf}{\S3.3 slides} and \href{https://drive.explaineverything.com/thecode/XAAUNGS}{video}
\item \emph{instructions:} write in normal form using $x_i(t)$ and identify the coefficients $a_{ij}(t)$ and non-homogeneous functions $f_i(t)$, if any
\item \emph{example 1.}
\begin{align*}
\frac{dx}{dt} &= - 2 x \hspace{70mm} \\
\frac{dy}{dt} &= x - y
\end{align*}
\item \emph{example 2.}
\begin{align*}
\frac{dx_1}{dt} &= -0.04 x_1 + 0.02 x_2 \hspace{60mm} \\
\frac{dx_2}{dt} &= 0.04 x_1 - 0.07 x_2 + 0.03 x_3 \\
\frac{dx_3}{dt} &= 0.05 x_2 - 0.05 x_3
\end{align*}
\end{itemize}
\end{frame}

\begin{frame}{examples, cont.}

\small
\begin{itemize}
\item \emph{instructions:} write in normal form using $x_i(t)$ and identify the coefficients $a_{ij}(t)$ and non-homogeneous functions $f_i(t)$, if any
\item \emph{example 3.}
\begin{align*}
y' &= u \hspace{75mm} \\
u' &= v \\
v' &= w \\
w' &= 4w-7v-10u+y+\sin(3t)
\end{align*}
\end{itemize}

\vspace{40mm}
\end{frame}


\begin{frame}{matrix form}

\begin{itemize}
\item X
\end{itemize}
\end{frame}


\begin{frame}{matrix multiplication}

\begin{itemize}
\item X
\end{itemize}
\end{frame}


\begin{frame}{verify}

\begin{itemize}
\item X
\end{itemize}
\end{frame}


\begin{frame}{linear independent solutions}

\begin{itemize}
\item X
\end{itemize}
\end{frame}


\begin{frame}{expectations}

\begin{itemize}
\item just watching this video is \emph{not} enough!
     \begin{itemize}
     \item see ``found online'' videos and stuff at

     \centerline{\href{https://bueler.github.io/math302/week13.html}{\tt \color{cyan} bueler.github.io/math302/week13.html}}
     \item \emph{read} \S8.1
     \item \emph{do} the WebAssign exercises for section 8.1
     \end{itemize}
\end{itemize}
\end{frame}

\end{document}

