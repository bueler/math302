\documentclass{beamer}

\usepackage[utf8]{inputenc}
\usepackage{fancybox}
\usepackage{environ}
\usepackage{tikz}

\beamertemplatenavigationsymbolsempty
\setbeamertemplate{footline}[frame number]

\title{2.1 Solution Curves (Without a Solution)}

\subtitle{a lesson for MATH F302 Differential Equations}

\author{Ed Bueler, Dept.~of Mathematics and Statistics, UAF}

\date{\tiny \today}


\usetheme{Pittsburgh}


\begin{document}

\setbeamertemplate{itemize item}{$\bullet$}
\setbeamertemplate{itemize subitem}{$\circ$}


\begin{frame}
\titlepage

\centerline{\tiny for textbook: \, D. Zill, \emph{A First Course in Differential Equations with Modeling Applications}, 11th ed.}
%\color{green!40!blue}
\end{frame}


\begin{frame}{meaning of a differential equation}

\begin{itemize}
\item let's start over on the meaning of a (first-order) differential equation:
    $$\frac{dy}{dx} = f(x,y)$$
\item as long as we have a formula for $f$ then we have a clear statement that
    $$\begin{matrix}
    \text{the slope of the} \\
    \text{solution } y 
    \end{matrix} \quad \stackrel{\text{equals}}{=} \quad
    \begin{matrix}
    \text{a known function of} \\
    \text{the location } (x,y)
    \end{matrix}$$
\item this literal reading of the differenential equation means that

\centerline{\alert{we can draw a picture of the differential equation itself}}

\item we can draw the picture whether or not we can do the calculus/algebra to find a solution (formula for) $y(x)$
\end{itemize}
\end{frame}




\begin{frame}{expectations}

to learn this material, just watching this video is \emph{not} enough; also
\begin{itemize}
\item \emph{read} section 2.1 in the textbook
\item \emph{do} the WebAssign exercises for section 2.1
\item see the other ``found online'' videos at bottom of this page:

\centerline{\href{https://bueler.github.io/math302/week2.html}{\tt \color{cyan} bueler.github.io/math302/week2.html}}
\end{itemize}
\end{frame}

\end{document}

