\documentclass[urlcolor=blue,dvipsnames]{beamer}

\usepackage[utf8]{inputenc}
\usepackage{fancybox,fancyvrb}
\usepackage{environ,xspace,empheq}
\usepackage{tikz}
\hypersetup{colorlinks,linkcolor=,urlcolor=cyan}

\beamertemplatenavigationsymbolsempty
\setbeamertemplate{footline}[frame number]
\usetheme{Pittsburgh}

\newcommand\enumnum[1]{{\renewcommand{\insertenumlabel}{#1}%
      \usebeamertemplate{enumerate item} \,}}

\newcommand{\grad}{\nabla}
\newcommand{\ih}{\boldsymbol{\hat{\textbf{\i}}}}
\newcommand{\jh}{\boldsymbol{\hat{\textbf{\j}}}}
\newcommand{\vF}{\boldsymbol{\vec{\textbf{F}}}}
\newcommand{\Matlab}{\textsc{Matlab}\xspace}
\newcommand{\Octave}{\textsc{Octave}\xspace}


\title{7.1 Laplace Transforms \\ (from the definition)}

\subtitle{a lesson for MATH F302 Differential Equations}

\author{Ed Bueler, Dept.~of Mathematics and Statistics, UAF}

\date{\tiny \today}


\begin{document}
\setbeamertemplate{itemize item}{$\bullet$}
\setbeamertemplate{itemize subitem}{$\circ$}
\renewcommand{\thefootnote}{{\color{green} \arabic{footnote}}}

\begin{frame}
\titlepage

\centerline{\tiny for textbook: \, D. Zill, \emph{A First Course in Differential Equations with Modeling Applications}, 11th ed.}
%\color{green!40!blue}
\end{frame}

\newcommand{\LL}[1]{\mathcal{L}\left\{#1\right\}}

\begin{frame}{the definition}

\begin{itemize}
\item the \alert{Laplace transform} of a function $f(t)$ defined on $(0,\infty)$ is
    $$\alert{\boxed{\LL{f(t)} = \int_0^\infty e^{-st} f(t)\,dt}}$$

\vspace{-2mm}
    \begin{itemize}
    \item this is the book's notation
    \end{itemize}
\item the \alert{result} of applying the Laplace transform \alert{is a function of s}, so slightly better notation would be
    $$\LL{f}(s) = \int_0^\infty e^{-st} f(t)\,dt$$
\item a common (and good) way to write it is
    $$F(s) = \int_0^\infty e^{-st} f(t)\,dt$$
\end{itemize}
\end{frame}


\begin{frame}{why do it in a differential equations class?}

\begin{itemize}
\item why do
    $$\LL{f(t)} = \int_0^\infty e^{-st} f(t)\,dt \,?$$
\item because the Laplace transform \alert{converts a linear DE} (in $t$) \alert{into an algebra problem} (in $s$)
    \begin{itemize}
    \item this is especially useful for solving (and thinking about) nonhomogeneous linear DEs
    \end{itemize}
\end{itemize}
\end{frame}


\begin{frame}{practice integrals out to $\infty$}

\begin{itemize}
\item in any case, a Laplace transform is an integral $\int_0^\infty \dots$
\item we need practice!
\item \emph{practice 1}.  compute
    $$\int_2^\infty e^{-3t}\,dt = \hspace{60mm}$$

\vspace{10mm}
\hfill $=\frac{1}{3} e^{-6}$
\item \emph{practice 2}.  compute and sketch
    $$\int_1^\infty \frac{1}{t}\,dt = \hspace{60mm}$$

\vspace{10mm}
\hfill $=+\infty$
\end{itemize}
\end{frame}


\begin{frame}{practice integrals, cont.}

\begin{itemize}
\item \emph{practice 3}.  compute and sketch
    $$\int_0^\infty t e^{-t}\,dt = \hspace{60mm}$$

\vspace{20mm}
\hfill $=+1$

\vspace{10mm}
\end{itemize}
\end{frame}


\begin{frame}{Laplace transforms from the definition}

\begin{itemize}
\item the way it works is to pre-compute the Laplace transforms of a bunch of familiar functions, and then use these to solve DEs
\item \emph{example 1}.  compute $\LL{e^{kt}}$

\vspace{30mm}
\item \emph{example 2}.  compute $\LL{1}$

\vspace{25mm}
\end{itemize}
\end{frame}


\begin{frame}{X}

\begin{itemize}
\item X
\end{itemize}
\end{frame}


\begin{frame}{expectations}

\begin{itemize}
\item just watching this video is \emph{not} enough!
     \begin{itemize}
     \item see ``found online'' videos and stuff at

     \centerline{\href{https://bueler.github.io/math302/week10.html}{\tt \color{cyan} bueler.github.io/math302/week11.html}}
     \item \emph{read} section 7.1 in the textbook
     \item \emph{do} the WebAssign exercises for section 7.1
     \end{itemize}
\end{itemize}
\end{frame}

\end{document}

