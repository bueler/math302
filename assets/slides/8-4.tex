\documentclass[urlcolor=blue,dvipsnames]{beamer}

\usepackage[utf8]{inputenc}
\usepackage{fancybox,fancyvrb}
\usepackage{environ,xspace,empheq}

\usepackage{tikz}
\usetikzlibrary{arrows.meta,decorations.markings,decorations.pathreplacing,fadings,positioning}

\hypersetup{colorlinks,linkcolor=,urlcolor=cyan}

\beamertemplatenavigationsymbolsempty
\setbeamertemplate{footline}[frame number]
\usetheme{Pittsburgh}

%\makeatletter
%\newcommand{\tinytiny}{\@setfontsize{\tinytiny}{4pt}{4pt}}
%\makeatother

\newcommand\enumnum[1]{{\renewcommand{\insertenumlabel}{#1}%
      \usebeamertemplate{enumerate item} \,}}

\newcommand{\bA}{\mathbf{A}}
\newcommand{\bC}{\mathbf{C}}
\newcommand{\bF}{\mathbf{F}}
\newcommand{\bI}{\mathbf{I}}
\newcommand{\bK}{\mathbf{K}}
\newcommand{\bM}{\mathbf{M}}
\newcommand{\bX}{\mathbf{X}}
\newcommand{\bZ}{\mathbf{Z}}


\newcommand{\grad}{\nabla}
\newcommand{\ih}{\boldsymbol{\hat{\textbf{\i}}}}
\newcommand{\jh}{\boldsymbol{\hat{\textbf{\j}}}}
\newcommand{\vF}{\boldsymbol{\vec{\textbf{F}}}}
\newcommand{\Matlab}{\textsc{Matlab}\xspace}
\newcommand{\Octave}{\textsc{Octave}\xspace}


\title{8.4 The matrix exponential solves systems}

\subtitle{a lesson for MATH F302 Differential Equations}

\author{Ed Bueler, Dept.~of Mathematics and Statistics, UAF}

\date{\tiny \today}


\begin{document}
\setbeamertemplate{itemize item}{$\bullet$}
\setbeamertemplate{itemize subitem}{$\circ$}
\renewcommand{\thefootnote}{{\color{green} \arabic{footnote}}}

\begin{frame}
\titlepage

\centerline{\tiny for textbook: \, D. Zill, \emph{A First Course in Differential Equations with Modeling Applications}, 11th ed.}
%\color{green!40!blue}
\end{frame}


\begin{frame}{the simplest ODEs}

\begin{itemize}
\item \emph{simplest scalar ODE:}
    $$y'=ay \qquad \text{has solution} \qquad y(t) = c e^{at}$$

\bigskip
\item \emph{simplest system of ODEs:}
    $$\bX'=\bA \bX \qquad \text{has solution} \qquad \bX(t) = e^{\bA t}\bC$$
\end{itemize}
\end{frame}


\begin{frame}{what does $e^{\bA t}$ mean?}

\begin{itemize}
\item what does $e^{\bA t}$ mean?
   \begin{itemize}
   \item what does $e^{at}$ mean?
       \begin{itemize}
       \item[$\ast$] what does $e^x$ mean?  \alert{we know this one!}
\begin{align*}
e^x &= 1 + x + \frac{x^2}{2} + \frac{x^3}{3!} + \frac{x^4}{4!} + \dots \\
    &= \sum_{k=0}^\infty \frac{x^k}{k!}
\end{align*}
       \end{itemize}
   \end{itemize}

\medskip
\item \emph{definition.} if $\bA$ is a square matrix and $t$ is any number then
\begin{align*}
e^{\bA t} &= \bI + \bA t + \bA^2 \frac{t^2}{2} + \bA^3 \frac{t^3}{3!} + \bA^4 \frac{t^4}{4!} + \dots \\
    &= \sum_{k=0}^\infty \bA^k \frac{t^k}{k!}
\end{align*}
   \begin{itemize}
   \item note $\bA^0 = \bI$ makes sense if we believe $x^0=1$
   \end{itemize}
\end{itemize}
\end{frame}


\begin{frame}{like exercise \#1 in \S8.4}

\begin{itemize}
\item $e^{\bA t} = \bI + \bA t + \bA^2 \frac{t^2}{2} + \bA^3 \frac{t^3}{3!} + \bA^4 \frac{t^4}{4!} + \dots = \sum_{k=0}^\infty \bA^k \frac{t^k}{k!}$

\medskip
\item \emph{example 1}.  use the above series to compute $e^{\bA t}$ and $e^{-\bA t}$, in simplified form, if
    $$\bA = \begin{pmatrix} 3 & 0 \\ 0 & -1 \end{pmatrix}     \hspace{80mm}$$
\end{itemize}

\vspace{40mm}
\end{frame}


\begin{frame}{like exercise \#4 in \S8.4}

\begin{itemize}
\item \emph{example 2}.  use the series definition to compute $e^{\bA t}$ and $e^{-\bA t}$, in simplified form, if
    $$\bA = \begin{pmatrix} x \end{pmatrix}     \hspace{80mm}$$
\end{itemize}

\vspace{50mm}
\end{frame}


\begin{frame}{like exercise \#3 in \S8.4}

\begin{itemize}
\item \emph{example 3}.  use the series definition to compute $e^{\bA t}$ and $e^{-\bA t}$, in simplified form, if
    $$\bA = \begin{pmatrix} x \end{pmatrix}     \hspace{80mm}$$
\end{itemize}

\vspace{50mm}
\end{frame}


\begin{frame}{two views}

\begin{itemize}
\item[\S 8.2:] the ODE system $\bX' = \bA \bX$ has solutions
    $$\bX_j(t) = \bK_j e^{\lambda_j t}$$
where $\lambda_j$ is an eigenvalue of $\bA$ and $\bK_j$ is a corresponding eigenvector,
    $$\bA\bK_j=\lambda_j \bK_j$$
and the general solution is
\begin{align*}
\bX(t) &= c_1 \bX_1(t) + \dots + c_n \bX_j(t) \\
       &= c_1 \bK_1 e^{\lambda_1 t} + \dots + c_n \bK_n e^{\lambda_n t}
\end{align*}
\item[\S 8.4:] the ODE system $\bX' = \bA \bX$ has general solution
    $$\bX(t) = e^{\bA t}\bC$$
\end{itemize}
\end{frame}



\begin{frame}[fragile]
\frametitle{use a computer}

\begin{itemize}
\item for the ODE IVP
$$\bX' = \bA \bX, \qquad \bX(0)=\bC$$
suppose you want $\bX(T)$

\medskip
\item with \Matlab/\Octave do
\begin{Verbatim}[fontsize=\small]
>> A = [...]         % enter square matrix A
>> C = [...]         % enter column vector C
>> expm(A*T) * C     % done!
\end{Verbatim}
\end{itemize}
\end{frame}


\begin{frame}{X}

\begin{itemize}
\item X
\end{itemize}
\end{frame}


\begin{frame}{X}

\begin{itemize}
\item X
\end{itemize}
\end{frame}


\begin{frame}{X}

\begin{itemize}
\item X
\end{itemize}
\end{frame}


\begin{frame}{X}

\begin{itemize}
\item X
\end{itemize}
\end{frame}


\begin{frame}{expectations}

\begin{itemize}
\item just watching this video is \emph{not} enough!
     \begin{itemize}
     \item see ``found online'' videos and stuff at

     \centerline{\href{https://bueler.github.io/math302/week14.html}{\tt \color{cyan} bueler.github.io/math302/week14.html}}
     \item \emph{read} \S8.4
     \item \emph{do} the WebAssign exercises for section 8.4
         \begin{itemize}
         \item these exercises are about cases where the matrix exponential reduces to a simple pattern or just a few nonzero terms
         \item like examples 1--3 above
         \end{itemize}
     \end{itemize}
\end{itemize}
\end{frame}

\end{document}

