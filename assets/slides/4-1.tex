\documentclass{beamer}

\usepackage[utf8]{inputenc}
\usepackage{fancybox,fancyvrb}
\usepackage{environ}
\usepackage{tikz}

\beamertemplatenavigationsymbolsempty
\setbeamertemplate{footline}[frame number]
\usetheme{Pittsburgh}

\newcommand\enumnum[1]{{\renewcommand{\insertenumlabel}{#1}%
      \usebeamertemplate{enumerate item} \,}}

\newcommand{\grad}{\nabla}
\newcommand{\ih}{\boldsymbol{\hat{\textbf{\i}}}}
\newcommand{\jh}{\boldsymbol{\hat{\textbf{\j}}}}
\newcommand{\vF}{\boldsymbol{\vec{\textbf{F}}}}


\title{4.1 Higher-order linear equations: \\ first examples and preliminaries}

\subtitle{a lesson for MATH F302 Differential Equations}

\author{Ed Bueler, Dept.~of Mathematics and Statistics, UAF}

\date{\tiny \today}


\begin{document}
\setbeamertemplate{itemize item}{$\bullet$}
\setbeamertemplate{itemize subitem}{$\circ$}


\begin{frame}
\titlepage

\centerline{\tiny for textbook: \, D. Zill, \emph{A First Course in Differential Equations with Modeling Applications}, 11th ed.}
%\color{green!40!blue}
\end{frame}


\begin{frame}{outline}

plan for these slides
\begin{itemize}
\item a bit of review of first-order linear equations (\S2.3)
\item how to solve constant-coefficient, second-order linear equations (from \S4.3)
\item a whole bunch of new language for higher-order linear equations
    \begin{itemize}
    \item basically, \S4.1 is just a bunch of new language
    \item I like to have some examples to talk about first, thus a bit of \S4.3 first
    \end{itemize}
\end{itemize}
\end{frame}


\begin{frame}{first-order linear DEs: a review}

\begin{itemize}
\item recall most-general form mentioned in \S2.3:
    $$a_1(x) y' + a_0(x) y = g(x)$$
\item you get to practical form by dividing by the leading coefficient:
    $$y' + P(x) y = f(x)$$

    \begin{itemize}
    \item \alert{this requires leading coefficient $a_1(x)$ to \emph{not} to be zero} on the interval where we are solving
    \end{itemize}
\item Case 1 (\alert{easiest to solve}) are constant-coefficient and homogeneous:
    $$y' + b y = 0$$

    \begin{itemize}
    \item \emph{homogeneous} means the right-hand side is zero
    \item we can see the solution is
        $$y(x) = A e^{-bx}$$
    \end{itemize}
\end{itemize}
\end{frame}


\begin{frame}{first-order linear review cont.}

\begin{itemize}
\item Case 2 are homogeneous but otherwise general:
    $$y' + P(x) y = 0$$

    \begin{itemize}
    \item now we need an integrating factor $\mu(x) = e^{Q(x)}$ where $Q(x) = \int P(x)\,dx$ is any antiderivative of $P(x)$
    \item multiplying by $\mu$ the equation becomes $\left(\mu(x) y(x)\right)' = 0$
    \item thus
        $$e^{Q(x)} y(x) = A$$
    \item thus the solution is
        $$y(x) = A e^{-Q(x)} \qquad \text{where } Q=\int P(x)\,dx$$
    \item homogeneous: \alert{a multiple of a solution is still a solution}
    \end{itemize}
\end{itemize}
\end{frame}


\begin{frame}{first-order linear review cont.$^2$}

\begin{itemize}
\item Case 3 are general first-order linear:
    $$y' + P(x) y = f(x)$$

    \begin{itemize}
    \item need same integrating factor as before; multiplying by $\mu=e^{Q(x)}$ the equation becomes $\left(\mu(x) y(x)\right)' = \mu(x) f(x)$
    \item integrate:
        $$e^{Q(x)} y(x) = A + \int_a^x e^{Q(t)} f(t)\,dt$$

        \begin{itemize}
        \item written to emphasize right side has a free constant $A$
        \end{itemize}
    \item thus the solution is
        $$y(x) = A e^{-Q(x)} +  e^{-Q(x)} \int_a^x e^{Q(t)} f(t)\,dt \quad \text{where } Q=\int P(x)\,dx$$
    \item \alert{solution is the homogeneous solution plus a particular solution}
    \end{itemize}
\end{itemize}
\end{frame}


\begin{frame}{higher-order linear DEs}

for $n$th-order linear equations
\begin{equation}
    a_n(x) y^{(n)} + a_{n-1}(x) y^{(n-1)} + \dots + a_1(x) y' + a_0(x) y = g(x),  \label{generallinear}
\end{equation}
\alert{versions all four comments in red on the previous slides still apply:}
\begin{enumerate}
\item if $a_n(x)$ is never zero we can divide by it and write
    $$y^{(n)} + b_{n-1}(x) y^{(n-1)} + \dots + b_1(x) y' + b_0(x) y = f(x)$$
\item easiest case (\S4.3) is homogeneous and constant coefficient
    $$a_n y^{(n)} + a_{n-1} y^{(n-1)} + \dots + a_1 y' + a_0 y = 0$$
\item any multiple of a solution, or sum of solutions, of a homogeneous equation
\begin{equation}
    a_n(x) y^{(n)} + a_{n-1}(x) y^{(n-1)} + \dots + a_1(x) y' + a_0(x) y = 0  \label{homogeneous}
\end{equation}
is again a solution
\item solutions of \eqref{generallinear} are always solutions of the homogeneous equation \eqref{homogeneous} plus a particular solution
\end{enumerate}
\end{frame}


\begin{frame}{X}

\begin{itemize}
\item X
\end{itemize}
\end{frame}

\begin{frame}{X}

\begin{itemize}
\item X
\end{itemize}
\end{frame}

\begin{frame}{X}

\begin{itemize}
\item X
\end{itemize}
\end{frame}

\begin{frame}{X}

\begin{itemize}
\item X
\end{itemize}
\end{frame}


\begin{frame}{expectations}

\begin{itemize}
\item just watching this video is \emph{not} enough!
     \begin{itemize}
     \item see ``found online'' videos at

     \centerline{\href{https://bueler.github.io/math302/week6.html}{\tt \color{cyan} bueler.github.io/math302/week6.html}}
     \item \emph{read} section 4.1 in the textbook
         \begin{itemize}
         \item know the meaning/definitions of:

\bigskip
             \hspace{-10mm} \begin{tabular}{ll}
             \emph{homogeneous}                     & \emph{Wronskian} \\
             \emph{nonhomogeneous}                  & \emph{fundamental set of solutions} \\
             \emph{associated homogeneous equation} & \emph{general solution} \\
             \emph{superposition}                   & \emph{particular solution} \\
             \emph{linearly dependent}              & \emph{complementary function} \\
             \emph{linearly independent}            &
             \end{tabular}

\bigskip
         \item I will not ask questions about ``boundary conditions'' and ``boundary value problems''!
         \item \dots but it is still quite a bit of new language
         \end{itemize}
     \item \emph{do} the WebAssign exercises for section 4.1
     \end{itemize}
\end{itemize}
\end{frame}

\end{document}

