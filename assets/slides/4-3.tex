\documentclass{beamer}

\usepackage[utf8]{inputenc}
\usepackage{fancybox,fancyvrb}
\usepackage{environ}
\usepackage{tikz}

\beamertemplatenavigationsymbolsempty
\setbeamertemplate{footline}[frame number]
\usetheme{Pittsburgh}

\newcommand\enumnum[1]{{\renewcommand{\insertenumlabel}{#1}%
      \usebeamertemplate{enumerate item} \,}}

\newcommand{\grad}{\nabla}
\newcommand{\ih}{\boldsymbol{\hat{\textbf{\i}}}}
\newcommand{\jh}{\boldsymbol{\hat{\textbf{\j}}}}
\newcommand{\vF}{\boldsymbol{\vec{\textbf{F}}}}
\newcommand{\Matlab}{\textsc{Matlab}}


\title{4.3 Homogeneous linear equations \\ with constant coefficients}

\subtitle{a lesson for MATH F302 Differential Equations}

\author{Ed Bueler, Dept.~of Mathematics and Statistics, UAF}

\date{\tiny \today}


\begin{document}
\setbeamertemplate{itemize item}{$\bullet$}
\setbeamertemplate{itemize subitem}{$\circ$}


\begin{frame}
\titlepage

\centerline{\tiny for textbook: \, D. Zill, \emph{A First Course in Differential Equations with Modeling Applications}, 11th ed.}
%\color{green!40!blue}
\end{frame}


\begin{frame}{linear, homogeneous, constant-coefficient}

\begin{itemize}
\item recall from \S4.1 slides that \underline{linear} DEs which are \underline{homogeneous} and \underline{constant-coefficient} always have exponential solutions
    \begin{itemize}
    \item thus you can always find at least one solution $y=e^{mx}$
    \item but the \underline{underlined} words are all important to this fact
    \end{itemize}
\item \emph{example 1}: solve the ODE IVP
    $$y'' -2 y' - 4 y = 0, \quad y(-1)=4, \, y'(-1)=0$$

\vspace{30mm}
\end{itemize}
\end{frame}


\begin{frame}{example 1, finished}

\vspace{35mm}

\hfill \includegraphics[width=0.5\textwidth]{figs/expodeivp2nd}
\end{frame}

\begin{frame}[fragile]
\frametitle{example 1: how I did it}

\begin{itemize}
\item here is how I solved for the constants and made the figure using \Matlab:

\bigskip
\begin{Verbatim}[fontsize=\small]
w = 1-sqrt(5);  z = 1+sqrt(5);
A = [exp(-w), exp(-z); w*exp(-w), z*exp(-z)];
b = [4; 0];
c = A \ b           % get: c(1)=0.8409, c(2)=28.119

x = -2:.01:1;
y = c(1) * exp(w*x) + c(2) * exp(z*x); 
plot(x,y),  grid on,  xlabel x,  ylabel y
axis([-2 0.5 0 50])
hold on,  plot(-1,4,'ko','markersize',12),  hold off
\end{Verbatim}

\bigskip
\item I am committed to helping you use a computer for math!

\end{itemize}
\end{frame}


\begin{frame}{example 2}

\begin{itemize}
\item \emph{example 2}: find the general solution of the ODE
    $$y'' + y = 0$$
\end{itemize}

\vspace{55mm}
\end{frame}


\begin{frame}{Euler's helpful identity}

\begin{itemize}
\item Euler recognized the connection between imaginary numbers and trig functions:
    $$e^{i\theta} = \cos\theta + i \sin\theta$$
\item \emph{exercise}: Explain \emph{Euler's identity} above using the Taylor series of $e^x,\cos x,\sin x$ at basepoint $x_0=0$.  Also draw a picture.
\end{itemize}

\vspace{50mm}
\end{frame}


\begin{frame}{example 3}

\begin{itemize}
\item from Euler's identity we also know
    $$e^{a+ib} = e^a(\cos b + i \sin b) \hspace{50mm}$$
\item \emph{example 3}: find the general solution of the ODE
    $$y''-4y'+5y=0$$
\end{itemize}

\vspace{50mm}
\end{frame}


\begin{frame}{the major facts of \S4.3}

\begin{itemize}
\item consider constant-coefficient and homogeneous linear ODE
    $$a_n y^{(n)} + a_{n-1} y^{(n-1)} + \dots + a_1 y' + a_0 y = 0$$
\item substitution of $y=e^{mx}$ yields (polynomial) \emph{auxiliary equation}
    $$a_n m^n + a_{n-1} m^{n-1} + \dots + a_1 m + a_0 = 0$$
\item any polynomial equation has at least one \emph{complex} solution
    \begin{itemize}
    \item auxiliary eqn.~has at least $1$ and at most $n$ distinct solutions
    \item some solutions may be repeated
    \end{itemize}
\item there is a \underline{recipe (next slide!)} which generates a fundamental set of $n$ real solutions and a general solution to the ODE:
    $$y_1(x),\dots,y_n(x) \quad \implies \quad y(x) = c_1 y_1(x) + \dots + c_n y_n(x)$$
\end{itemize}
\end{frame}


\begin{frame}{main recipe of \S4.3}

find all solutions (roots) $m$ of the auxiliary equation, namely
    $$a_n m^n + a_{n-1} m^{n-1} + \dots + a_1 m + a_0 = 0,$$
and build the fundamental solution set this way:
\begin{itemize}
\item[case I]   if $m$ is a real root then
    $$e^{mx} \,\, \text{ is in the set}$$
\item[case II]  if $m$ is a real root which is repeated $k$ times then
    $$e^{mx}, x e^{mx}, \dots, x^{k-1} e^{mx} \,\, \text{ are in the set}$$
\item[case III] if $m=a\pm i b$ is a complex root then
    $$e^{ax} \cos(bx), e^{ax} \sin(bx) \,\, \text{ are in the set}$$
\end{itemize}
\end{frame}


\begin{frame}{exercise 5 in \S4.3}

\begin{itemize}
\item \emph{exercise 5}:  find the general solution of the second-order DE
    $$y''+8y'+16y=0$$
\end{itemize}

\vspace{60mm}
\end{frame}


\begin{frame}{exercise 23 in \S4.3}

\begin{itemize}
\item \emph{exercise 23}:  find the general solution of the higher-order DE
    $$y^{(4)} + y''' + y'' = 0$$
\end{itemize}

\vspace{60mm}
\end{frame}


\begin{frame}{exercise 55 in \S4.3}

\begin{itemize}
\item \emph{exercise 55}:  find a constant-coefficient, homogeneous linear DE whose general solution is
    $$y(x) = c_1 e^{-x} \cos x + c_2 e^{-x} \sin x$$
\end{itemize}

\vspace{60mm}
\end{frame}


\begin{frame}{like exercise 69 in \S4.3}

\begin{itemize}
\item \emph{like exercise 69}:  solve the ODE IVP
\begin{align*}
& 2 y^{(4)} + 13 y''' + 21 y'' + 2 y' - 8 y = 0 \\
& y(0)=-2, \, y'(0)=6, \, y''(0)=3, \, y'''(0)=\tfrac{1}{2}
\end{align*}
\emph{hint}.  you may use a computer algebra system (CAS)
\end{itemize}

\vspace{50mm}
\end{frame}


\begin{frame}[fragile]
\frametitle{exercise 69: how to do it}

\begin{Verbatim}[fontsize=\footnotesize]
>> m = roots([2,13,21,2,-8])'
m =
          -4          -2          -1         0.5
>> A = [1 1 1 1; m; m.^2; m.^3]
A =
           1           1           1           1
          -4          -2          -1         0.5
          16           4           1        0.25
         -64          -8          -1       0.125
>> b = [-2 6 3 0.5]';
>> c = A \ b
c =
    -0.48148
         5.4
     -12.222
      5.3037
\end{Verbatim}

\footnotesize \alert{\emph{conclusion}:  A computer is very effective \dots if you know where you are going.}
\end{frame}


\begin{frame}{hyperbolic functions}

\begin{itemize}
\item Euler's identity $\alert{e^{i\theta}=\cos\theta+i\sin\theta}$, for complex exponentials, has an analog for real exponentials

\noindent \begin{minipage}[t]{0.45\textwidth}
\item by definition:
\begin{align*}
\cosh x &= \frac{e^x + e^{-x}}{2} \\
\sinh x &= \frac{e^x - e^{-x}}{2}
\end{align*}

\vspace{-5mm}
    \begin{itemize}
    \item the even and odd parts of the exponential, resp.
    \item called \emph{hyperbolic} functions
    \end{itemize}
\end{minipage}
\begin{minipage}[t]{0.46\textwidth}
\vspace{5mm}

\hfill \includegraphics[width=0.9\textwidth]{figs/coshsinh}
\end{minipage}

\bigskip
\item it is easy to see that
    \begin{itemize}
    \item $\alert{e^x = \cosh x + \sinh x}$
    \item $(\cosh x)' = \sinh x$, \, $(\sinh x)' = \cosh x$
    \item $y=c_1 \cosh x + c_2 \sinh x$ is the general solution to $y''-y=0$
    \end{itemize}
\end{itemize}
\end{frame}


\begin{frame}{some nice cases}

\begin{itemize}
\item the following general solutions can all be computed by substituting $y=e^{mx}$, and getting the auxiliary equation, etc.
\item \dots but it is good to \emph{quickly} see these special cases:
%\newcommand{\genarrow}{\quad \stackrel{\text{\tiny has general solution}}{\longrightarrow} \quad}
\newcommand{\genarrow}{\qquad \longrightarrow \qquad}
\begin{align*}
 &\quad\,\, {\scriptsize \begin{matrix} \text{has general} \\ \text{solution} \end{matrix} } \\
y' = k y &\genarrow y=Ae^{kx} \\
y'' + k^2 y = 0 &\genarrow y=c_1\cos(kx) + c_2\sin(kx) \\
y'' - k^2 y = 0 &\genarrow \left[\begin{matrix} y=c_1e^{kx}+c_2e^{-kx} \\
                                      \text{or} \\
                                      y=c_1\cosh(k x) + c_2\sinh(k x)\end{matrix}\right] \\
y'' \pm ky' = 0 &\genarrow y=c_1 + c_2e^{\mp kx} \\
y'' = 0 &\genarrow y=c_1 + c_2 x
\end{align*}
\end{itemize}
\end{frame}


\begin{frame}{expectations}

\begin{itemize}
\item just watching this video is \emph{not} enough!
     \begin{itemize}
     \item see ``found online'' videos at

     \centerline{\href{https://bueler.github.io/math302/week6.html}{\tt \color{cyan} bueler.github.io/math302/week6.html}}
     \item \emph{read} section 4.3 in the textbook
         \begin{itemize}
         \item for \S4.3 you at least need to know these terms:

\medskip
             \begin{tabular}{ll}
             \emph{homogeneous} \\
             \emph{linearly (in)dependent} \\
             \emph{Wronskian} \\
             \emph{fundamental set of solutions} \\
             \emph{linear combination} \\
             \emph{general solution}
             \end{tabular}

\medskip
         \item the reasons why the repeated-roots case generates additional linearly-independent solutions via extra factors of ``$x$'' \underline{is explained in \S4.2}
         \end{itemize}
     \item \emph{do} the WebAssign exercises for section 4.3
     \end{itemize}
\end{itemize}
\end{frame}

\end{document}

