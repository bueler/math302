\documentclass{beamer}

\usepackage[utf8]{inputenc}
\usepackage{fancybox}
\usepackage{environ}


\title{\mbox{2.2 Separable Equations} \\ \phantom{foo}}

%\subtitle[short version]{}

\date{%
}

\author{\mbox{Ed Bueler}}

\institute{\scriptsize Dept.~of Mathematics and Statistics \\ UAF}

\usetheme{Pittsburgh}

%\setbeamercolor{normal text}{fg=white,bg=black}

\NewEnviron{rightframe}[1][]{%
\begin{frame}[t]
\frametitle{#1}
\begin{columns}
\begin{column}{0.4\textwidth}
%nothing
\end{column}
\begin{column}{0.6\textwidth}
\BODY
\end{column}
\end{columns}
\end{frame}
}


\begin{document}

\begin{rightframe}[]
\titlepage

\vspace{30mm}
\tiny {\color{blue!30} D. Zill, \emph{A First Course in Differential Equations with Modeling Applications}, 11th ed.}
\end{rightframe}

\begin{rightframe}[]
a \emph{separable} differential equation can be put in the form
        $$\frac{dy}{dx} = g(x) h(y)$$
\only<2-3>{
\begin{itemize}
\only<2>{
    \item \textbf{Example}.
        $$\frac{dy}{dx} = \frac{y\cos(x)}{1+y^2}$$
%PUT IT IN THIS FORM
}
\only<3>{
    \item \textbf{Not an example}.
        $$\frac{dy}{dx} = \cos(x) + y$$
    \item \textbf{Also not an example}.
        $$\frac{dy}{dx} = \sin(x + y^2)$$
}
\end{itemize}
}
\only<4->{OR in the form
        $$p(y) \frac{dy}{dx} = g(x)$$
where $p(y)=1/h(y)$
}
\end{rightframe}

\begin{rightframe}[]
how to solve separable equations?

\emph{answer}.  clear denominators in
    $$p(y) \frac{dy}{dx} = g(x)$$
and integrate both sides of 
    $$p(y)\,dy = g(x)\, dx$$
to get
    $$\int p(y)\,dy = \int g(x)\, dx$$
\only<2>{
this works because of the chain rule
}
\end{rightframe}

\begin{rightframe}[]
\begin{itemize}
\item \textbf{Example.}
    $$\frac{dy}{dx} = x y^2$$
\end{itemize}
\end{rightframe}

\begin{rightframe}[]
some familiar equations are separable
\begin{itemize}
\item \textbf{Example.}
    $$\frac{dy}{dx} = - 5 y$$
\end{itemize}
\end{rightframe}

\begin{rightframe}[]
what if there are initial conditions?
\begin{itemize}
\item \textbf{Example.}  Find $z(t)$ if
    $$z' = \frac{e^{-z}}{t}$$
and if $z(4)=1$
\end{itemize}
\end{rightframe}

\begin{rightframe}[]
you may end up only knowing the solution implicitly
\begin{itemize}
\item \textbf{Example.}
    $$\frac{dy}{dx} = \frac{x(1-x)}{y(2+y)}$$
\end{itemize}
\end{rightframe}

\begin{rightframe}[]
\emph{expectations}.  you must \emph{read} section 2.2 to learn about some pitfalls, including
\begin{itemize}
\item making sure you find all solutions
\item how to write solutions if you can't do the integrals by hand
\end{itemize}
\only<2>{and \emph{doing exercises} is essential}
\end{rightframe}

\end{document}

