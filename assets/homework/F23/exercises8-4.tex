\documentclass[12pt]{article}
\usepackage[top=1.2in, bottom=0.8in, left=0.9in, right=1.1in]{geometry}

\usepackage{graphicx,color,enumitem,fancyvrb}
\usepackage{amsmath,amsthm,amsbsy}
\usepackage{palatino}
\usepackage{mdframed}

\usepackage{tikz}
\usepackage[colorlinks=true]{hyperref}

%% Setup aproblem environment, 
%% aproblem items
%% subproblems environment
%% subproblem items
\makeatletter
\newcounter{probcount}
\newcounter{subprobcount}
\newlength\probsep
\newlength\pshrinking
\newif\iffirstprob

\newenvironment{aproblems}%
  {\ifhmode\unskip\par\fi\setcounter{probcount}{0}\probsep\parskip
  \sbox\@tempboxa{\textbf{9.}}\pshrinking\wd\@tempboxa\advance\pshrinking\labelsep
  \let\hproblem\aproblem
  \advance\linewidth -\pshrinking
  \advance\@totalleftmargin\pshrinking
  \advance\leftskip\pshrinking}%
  {\ifhmode\unskip \par\fi\advance\leftskip-\pshrinking}%

\newcommand{\aproblem}{%
  \setcounter{subprobcount}{0}%
  \stepcounter{probcount}%
  \def\@currentlabel{\arabic{probcount}}%
  \ifhmode
    \unskip \par
  \fi
%  \addpenalty{-4000}%
  \iffirstprob\else\addvspace\probsep\fi
  \firstprobfalse
  \hskip -\labelwidth\hskip -\labelsep 
  \hbox to\labelwidth{\hss\textbf{E\arabic{probcount}.}}\hskip\labelsep
}%

\newcommand{\subprob}{\item\def\@currentlabel{\arabic{probcount}\alph{\thelistlabel}}}
\newcommand{\skipproblem}{\stepcounter{probcount}}


%% The following commands put defined left and right headers on the top, and a page number
%% on the bottom of all pages beyond page 1
\usepackage{fancyhdr}
\pagestyle{fancy}
\fancyfoot[C]{\ifnum \value{page} > 1\relax\thepage\fi}
\fancyhead[L]{\ifx\@doclabel\@empty\else\@doclabel\fi}
\fancyhead[R]{\ifx\@docdate\@empty\else\@docdate\fi}
\headheight 15pt
\def\doclabel#1{\gdef\@doclabel{#1}}
\def\docdate#1{\gdef\@docdate{#1}}
\makeatother

\mdfdefinestyle{probSty}{}
\theoremstyle{definition}
\newmdtheoremenv[style=probSty, linewidth=0.5pt, topline=false, bottomline=false, %
rightline=false, leftmargin=0pt, innerleftmargin=0.4em, rightmargin=0pt, %
innerrightmargin=0pt, innertopmargin=-5pt, innerbottommargin=3pt, %
splittopskip=\topskip, splitbottomskip=0.3\topskip, %
skipabove=1.3\topsep]{prob}{Problem}

\renewcommand*{\theprob}{\arabic{prob}}

%% General formatting parameters
\parindent 0pt
\parskip 6pt plus 1pt

\newcommand{\bA}{\mathbf{A}}
\newcommand{\bC}{\mathbf{C}}
\newcommand{\bI}{\mathbf{I}}
\newcommand{\bX}{\mathbf{X}}
\newcommand{\exer}[1]{\noindent \textbf{Problem #1.} \,}
%\newcommand{\epart}[1]{\noindent \textbf{(#1)} \,}
\newcommand{\eps}{\epsilon}
\newcommand{\ds}{\displaystyle}

\doclabel{Math F302 Homework 8.4}
\docdate{5 Dec 2023}

\begin{document}
\strut
\centerline{{\Large \textbf{Homework 8.4}}}

\bigskip
\centerline{{\large due 11:59pm Monday 11 December, by Gradescope as usual}}

\bigskip
\small
\noindent \textbf{The matrix exponential.}  For a square matrix $\bA$, the matrix exponential function $e^{\bA t}$ is defined as
	$$e^{\bA t} = \bI + \bA t + \bA^2 \frac{t^2}{2} + \dots + \bA^k \frac{t^k}{k!} + \dots = \sum_{k=0}^\infty \bA^k \frac{t^k}{k!}$$
Notice that if $t=0$ then the matrix exponential is just the identity matrix:
    $$e^{\bA\,0} = \bI$$
The formula for $e^{\bA t}$ includes the ordinary (scalar) exponential as a special case: if $\bA = (\,a\,)$ is a $1\times 1$ matrix with entry $a$ then $e^{\bA t} = e^{at} = 1 + at + (at)^2/2 + (at)^3/3! + \dots$

One can only compute $e^{\bA t}$ by hand in easy cases.  First of all you need to be able to compute powers of $\bA$; you have to know how to do matrix-matrix multiplication.  Then one ``easy case'' is when $\bA$ is a diagonal matrix.  Another is when $\bA$ is a nonzero matrix for which there is a power $\bA^k$ which is the zero matrix, because then the infinite series becomes a finite sum.

The key fact about the matrix exponential, which makes it useful for differential equations, is the usual derivative rule for exponentials:
	$$\frac{d}{dt} e^{\bA t} = \bA e^{\bA t}$$

The matrix exponential allows us to solve any 1st-order, linear, constant-coefficient system of differential equations with ease.  For homogeneous systems
	$$\bX' = \bA \bX$$
the general solution is
	$$\bX(t) = e^{\bA t} \bC$$
Here $\bX(t)$ is a column vector of the solution components and $\bC$ is a column vector of constants:
    $$\bX(t) = \begin{pmatrix} x_1(t) \\ \vdots \\ x_n(t) \end{pmatrix}, \qquad \bC = \begin{pmatrix} c_1 \\ \vdots \\ c_n \end{pmatrix}$$
Notice that $\bX(0)=\bC$ in the general solution formula (because $e^{\bA 0} = \bI$).  If $\bX_0$ is a vector then the solution of the initial value problem
	$$\bX' = \bA \bX, \qquad \bX(0) = \bX_0$$
is
	$$\bX(t) = e^{\bA t} \bX_0$$
In other words, $\bC$ is the vector of initial values.

As usual, a computer can do the job!  In Matlab or Octave, $e^{\bA t}=$ \verb|expm(A * t)|.  (\emph{The command} \texttt{exp()} \emph{gives the wrong answer here.  It exponentiates entrywise.})  So, if you have entered a square matrix \texttt{A} and a vector \texttt{C}, then the solution $X(t)$ at a particular time \texttt{t} is
\begin{Verbatim}
>> X = expm(A * t) * C
\end{Verbatim}


\clearpage\newpage
\normalsize
\subsection*{Graded for Correctness}

In Problems 1 and 2 use the definition to compute $e^{\bA t}$ by hand, and simplify.

\exer{1}  $\ds \bA = \begin{pmatrix} 0 & 0 & 0 \\ 3 & 0 & 0 \\ 5 & 1 & 0 \end{pmatrix}$

\exer{2}  $\ds \bA = \begin{pmatrix} 0 & 1 \\ -1 & 0 \end{pmatrix}$

\medskip
Problems 3 and 4 require technology, presumably Matlab.  Compute the matrix $e^{\bA t}$ and the (particular) vector $\bX(t) = e^{\bA t} \bC$.

\exer{3}  $\ds \bA = \begin{pmatrix} 0 & 1 \\ -1 & 0 \end{pmatrix}$, $\ds \bC = \begin{pmatrix} 0 \\ 2 \end{pmatrix}$, $t=1$

\exer{4}  $\ds \bA = \begin{pmatrix} 0 & 1 \\ 6 & 1 \end{pmatrix}$, $\ds \bC = \begin{pmatrix} 1 \\0 \end{pmatrix}$, $t=0.5$

\medskip
Problems 5 and 6 relate the above calculations to familiar stuff.  Use the \textbf{auxiliary equation method} (\S4.3) to solve the initial value problem by hand.  You will get the same numbers as in Problems 3 and 4.  Explain this by \textbf{writing the differential equation as a 1st-order system} and \textbf{stating explicitly} what $y(t)$ corresponds to in Problems 3 and 4.

\exer{5}  Solve the initial value problem and compute $y(t)$ at $t=0.5$:
	$$y'' - y' - 6 y = 0, \qquad y(0)=1, \, y'(0)=0$$

\exer{6}  Solve the initial value problem and compute $y(t)$ at $t=1$:
	$$y'' + y = 0, \qquad y(0)=0, \, y'(0)=2$$


\subsection*{Graded for Completeness}

Use the definition to compute $e^{\bA t}$ by hand, and simplify.

\exer{7}  $\ds \bA = \begin{pmatrix} 1 & 1 & 1 \\ 1 & 1 & 1 \\ -2 & -2 & -2 \end{pmatrix}$

\medskip
Use technology to compute the matrix $e^{\bA t}$ and the vector $\bX(t) = e^{\bA t} \bC$.

\exer{8}  $\ds \bA = \begin{pmatrix} 1 & 2 & 3 \\ 4 & 0 & -1 \\ -2 & -3 & -4 \end{pmatrix}$, $\ds \bC = \begin{pmatrix} -1 \\ 9 \\ 1 \end{pmatrix}$, $t=\pi$

\end{document}
