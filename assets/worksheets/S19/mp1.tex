\documentclass[12pt]{article}
\usepackage[top=0.9in, bottom=0.9in, left=0.9in, right=1.1in]{geometry}

\usepackage{graphicx,color,enumitem}
\usepackage{amsmath,amsthm,amsbsy}
\usepackage{palatino}

\usepackage{tikz}

%% Setup aproblem environment, 
%% aproblem items
%% subproblems environment
%% subproblem items
\makeatletter
\newcounter{probcount}
\newcounter{subprobcount}
\newlength\probsep
\newlength\pshrinking
\newif\iffirstprob

\newenvironment{aproblems}%
  {\ifhmode\unskip\par\fi\setcounter{probcount}{0}\probsep\parskip
  \sbox\@tempboxa{\textbf{9.}}\pshrinking\wd\@tempboxa\advance\pshrinking\labelsep
  \let\hproblem\aproblem
  \advance\linewidth -\pshrinking
  \advance\@totalleftmargin\pshrinking
  \advance\leftskip\pshrinking}%
  {\ifhmode\unskip \par\fi\advance\leftskip-\pshrinking}%

\newcommand{\aproblem}{%
  \setcounter{subprobcount}{0}%
  \stepcounter{probcount}%
  \def\@currentlabel{\arabic{probcount}}%
  \ifhmode
    \unskip \par
  \fi
%  \addpenalty{-4000}%
  \iffirstprob\else\addvspace\probsep\fi
  \firstprobfalse
  \hskip -\labelwidth\hskip -\labelsep 
  \hbox to\labelwidth{\hss\textbf{\arabic{probcount}.}}\hskip\labelsep
}%

\newcommand{\subprob}{\item\def\@currentlabel{\arabic{probcount}\alph{\thelistlabel}}}
\newcommand{\skipproblem}{\stepcounter{probcount}}


%% The following commands put defined left and right headers on the top, and a page number
%% on the bottom of all pages beyond page 1
\usepackage{fancyhdr}
\pagestyle{fancy}
\fancyfoot[C]{\ifnum \value{page} > 1\relax\thepage\fi}
\fancyhead[L]{\ifx\@doclabel\@empty\else\@doclabel\fi}
\fancyhead[R]{\ifx\@docdate\@empty\else\@docdate\fi}
\headheight 15pt
\def\doclabel#1{\gdef\@doclabel{#1}}
\def\docdate#1{\gdef\@docdate{#1}}
\makeatother

%% General formatting parameters
\parindent 0pt
\parskip 6pt plus 1pt

\newcommand{\epart}[1]{\noindent \textbf{(#1)} \,}

\doclabel{Math F302 UX1: Mini-Project 1}
\docdate{due \emph{at 11pm} on Thursday 17 January 2019}

\begin{document}
\renewcommand{\d}{\displaystyle}

\strut
\centerline{{\Large \textsc{High Points of Calculus}}}

\medskip

\begin{quote}
This is a review of some calculus you will need for differential equations.  However, this first Mini-Project is also a test of whether you are able to successfully submit your work in this online course!

\medskip
\fbox{\begin{minipage}[t]{0.875\textwidth}
\small
\emph{Logistics.}  Please fill in the blanks below with complete and legible answers.  You may do this either by
\begin{itemize}
\item electronically-editing the PDF or
\item printing this blank version and writing your answers with pencil or pen.
\end{itemize}
In the first case, save your completed document as a PDF.  In the second case you should scan or photograph your completed document and then figure out how to save it as a PDF.  In any case \emph{you must produce a PDF for submission}.  Then submit it by uploading using the Google Form link.
\end{minipage}}
\end{quote}

\bigskip

\begin{aproblems}
\aproblem \emph{Chain rule.}  Recall the chain rule
    $$\left[f\left(g(x)\right)\right]' = f'\left(g(x)\right)\, g'(x)$$

\epart{a} Identify an outer function $f(x)$ and an inner function $g(x)$ if $f(g(x))=\sqrt{\tan x + 3 x}$.  Then compute the following derivative.
    $$\left[\sqrt{\tan x + 3 x}\right]' = \hspace{4.0in}$$

\vfill

\epart{b} Construct your own chain rule example:
\begin{align*}
f(x) &= \phantom{\sin x sdfakljdklja f sdabj adfs asdfk afdklj asdf \Big|}\\
g(x) &= \phantom{2^x + \ln x} \phantom{\Big|}\\
\left[f\left(g(x)\right)\right]' &=  \phantom{\Big|}
\end{align*}

\vspace{0.5in}
\noindent (\emph{Make it different from} \textbf{(a)} \emph{and non-trivial but not too complicated.  In particular, neither $f(x)$ nor $g(x)$ should be as simple as a linear function, i.e.~$ax+b$.  Remember that $x^k,e^x,\ln x,b^x$, $\log_b x,\sin x,\cos x,\tan x,\sec x,\arcsin x$ are common functions from calculus which you must be able to correctly differentiate!  Use this problem to practice those that are least familiar?})

\newpage
\aproblem \emph{Integration by substitution.}  Remember that the indefinite integral just means ``anti-derivative,'' so $\int f(x)\,dx = F(x) + C$ means exactly the same thing as $\left(F(x)\right)' = f(x)$.

In any case you can do some integrals just by recognizing a derivative, perhaps with some fiddling with constants.

\epart{a} Compute the indefinite integral:
    $$\int \frac{1}{x}\,dx = \hspace{4.0in}$$

\bigskip

\epart{b} Compute the indefinite integral:
    $$\int 3^{2x}\,dx = \hspace{4.0in}$$

\bigskip

Substitution is \textbf{the chain rule in reverse}.  For example, from \textbf{1 (a)} we have
\begin{align*}
\int \frac{\sec^2 x + 3}{\sqrt{\tan x + 3 x}}\,dx &= \int \frac{du}{\sqrt{u}} \hspace{1.0in} [\text{with } u=\tan x + 3x] \\
  &= 2 u^{1/2} + C = 2 \sqrt{\tan x + 3 x} + C
\end{align*}
(It is common to need to fiddle with constant factors like the ``$2$'' here.)  In general:
   $$\int f'(g(x)) g'(x)\,dx = \int f'(u)\,du = f(u) + C = f(g(x)) + C$$

\medskip

\epart{c} Turn your chain rule example in \textbf{1 (b)} into an integration by substitution.

\vfill

\newpage
\aproblem  \emph{Product rule and integration-by-parts.}  The product rule
    $$\left[u(x) v(x)\right]' = u'(x) v(x) + u(x) v'(x)$$
can be used in reverse too.  The indefinite integral of both sides of the above gives
    $$u(x) v(x) = \int u'(x) v(x)\,dx + \int u(x) v'(x)\,dx.$$
The main use of this is to exchange one of the last two integrals for the other, which is integration-by-parts:
    $$\int u(x) v'(x)\,dx = u(x) v(x) - \int u'(x) v(x)\,dx.$$
(You probably have this memorized as $\int u\, dv = u v - \int v\,du$.)

\medskip

\epart{a} Construct your own product rule example:
\begin{align*}
u(x) &= \phantom{\sin x sdfakljdklja f sdabj adfs asdfk afdklj asdf \Big|}\\
v(x) &= \phantom{2^x + \ln x} \phantom{\Big|}\\
\left[u(x) v(x)\right]' &=  \phantom{\Big|}
\end{align*}

\vspace{0.2in}
(\emph{Again, make it non-trivial but not too complicated.})

\medskip

\epart{b} Turn the above example into an integration-by-parts example.

\vfill

\newpage
\aproblem  \emph{Fundamental Theorem of Calculus (FTC).}  When you compute a definite integral by hand you usually use a form of the FTC:
    $$\int_a^b f(x)\,dx = F(b) - F(a) \qquad \text{where } F'(x) = f(x)$$
This says that doing an integral is the same as un-doing a derivative.

Recall that if you do a definite integral by substitution then you can change the limits:
    $$\int_{x=a}^{x=b} f(g(x))\,g'(x)\,dx = \int_{u=g(a)}^{u=g(b)} f(u)\,du = F(g(b)) - F(g(a))$$

\epart{a}  Compute
    $$\int_{\pi/6}^{\pi/2} \frac{\cos x}{1 + 9 \sin^2 x}\,dx = \phantom{\int_{3/2}^3 \frac{du/3}{1+u^2} =} \hspace{3.0in}$$

\vspace{0.5in}

There is another form of the FTC, often called ``FTC I.''  It says that a derivative un-does an integral:
    $$\frac{d}{dx} \left(\int_a^x f(t)\,dt\right) = f(x)$$
The integral inside the parentheses computes the area under the curve $y=f(t)$ from $t=a$ to $t=x$.  One should think of this area varying as $x$ changes, thus defining a function $g(x) = \int_a^x f(t)\,dt$.  One may answer some questions about $g(x)$ even when an antiderivative of the integrand is not known.

\epart{b}  Suppose we define
    $$g(x) = \int_2^x \sin(e^t)\,dt$$
Compute the exact value of $g(2)$.

\vspace{0.5in}

\epart{c}  For the same function $g(x)$, find $g'(x)$.

\vfill

\end{aproblems}

\end{document}
